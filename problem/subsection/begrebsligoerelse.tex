\subsection{Begrebsligørelse}

Der er indtil videre blevet  argumenteret for relevansen af det initierende problem, og det er i den sammenhæng derfor nødvendigt at redegøre for nogle vigtige emner og ord indenfor problemfeltet. Formålet med dette afsnit er at beskrive vigtige begreber samt kort at give en beskrivelse af, hvordan de forskellige ord og begreber skal forstås i den videre rapport. Begreberne som fremgår i følgende afsnit danner grundlag for forståelsen af det initierende problem. Disse begreber er følgende: Forbruger, sælger, visualisering, lys og lamper.

% \subsubsection{Køber}
% En køber er en person, der køber et produkt eller tjenesteydelser \cite{ddo_forbruger}.  Køberen står altså i denne sammenhæng i modsætning til producenterne.
% I denne rapport opfattes køberen som den person der køber lampen. Det vil altså sige, at der i dette tilfælde ikke nødvendigvis er tale om personer der til dagligt bruger eller bliver påvirket af lampen.

\subsubsection{Forbruger}
En forbruger er en privatperson som køber et produkt eller tjenesteydelser. “Forbruge” betyder at “bruge noget”, og en forbruger køber derfor produkter med henblik på at tilfredsstille nogle behov \cite{forbrugerportalen}. En bevidst forbruger, vil derfor ofte leder efter produkter der opfylder deres behov. Man antages også for at være forbruger af en varer hvis man til daglig benytter sig af eller bliver påvirket af en given lampe.
I vores rapport udvider vi definitionen af forbruger til, at en forbruger også kan være en erhvervsperson der køber en lampe til brug i virksomheden.

\subsubsection{Sælger}
En sælger er den person eller virksomhed der sælger et produkt eller tjenesteydelser. I rapporten opfattes sælgeren som værende en person eller virksomhed, der sælger lamper til forbrugeren. 

\subsubsection{Visualisering}
At visualisere noget, betyder at man er i stand til at omdanne tanker, minder og lignende til billeder \cite{ddo_visualisering}. At visualisere noget betyder altså, at man er i stand til at forestille sig noget visuelt. Visualisering kan blandt andet foregå gennem grafer, filmklip og lignende.  
Et redskab til visualisering er prototyper. Prototyper er modeller som bruges med henbik på tests og tilpasning. \cite{prototyper_pdf}.

\subsubsection{Lys}
Der er forskellige opfattelser af hvad lys det indebærer. Hvis vi tager udgangspunkt i Karsten Rottwitt, som er professor ved DTU fotonik, så påstår han at lys er:

“Lys er andet end synligt lys. For mig er lys et elektromagnetisk felt, som har en høj frekvens”
- Karsten Rottwitt\cite{def_lys}.

Han mener også, at der er en hårdfin grænse for hvornår lys kan betegnes som lys, denne grænse er dog først i spil når vi snakker om UV-lys og infrarødt lys \cite{def_lys}. 
Andre er ikke enige med Karsten Rottwitt om hvordan definitionen af lys er. Tager vi nu udgangspunkt i Britannica \cite{britannica_lys}, så betegnes lys, som magnetiske stråler, som det menneskelige øje kan opfange - Hvilket vil sige, lys med en bølgelængde mellem 380 og 750 nanometer, også kaldet synligt lys. 
Dette er denne definition, som rapporten vil tage udgangspunkt i. Dette er valgt, da det er oplagt at kombinere synligt lys og lamper \cite{def_lys}.


Det lys som kommer fra en lampe, er selvfølgelig af forskellig kvalitet. Kvalitet kan ligesom lys, betegnes på mange måder, herunder kan vi snakke om hvorvidt en lyskilde er af god kvalitet, hvis den er energivenlig eller om det kun kommer an på hvor gode de er til at eftergive farvet lys. 
Afhængigt af hvor man skal bruge lyset, kan nogle former for lys være bedre egnet end andre. Her menes der om hvorvidt lyset skal være varmt eller koldt. Integral-led er et firma med over 25 års erfaring, og har opstillet nogle foretrukne steder at bruge de forskellige typer af lys:

Varm / varm hvid = Stue, soveværelse eller gange.
Hvid / kold hvid = Køkken, studie, badeværelser, skrivebord, kontor eller butikker\cite{varm_kold}.

Ud fra disse foretrukne placeringer, kan vi antage at lys kvaliteten blandt andet afhænger af hvor pæren skal bruges. Hvis det er i et stille og roligt miljø, med hensyn på at slappe af, er det måske at foretrække det varme lys, hvorimod de steder hvor det er nødvendigt for skarpt lys, for evt. at kunne se detaljer eller koncentrere sig, er det kolde lys at foretrække \cite{varm_kold}.

Der findes mange forskellige slags pærer, men hvilken skal man vælge? Sparepærer, LED eller halogen?
Da der findes så mange pærer, er der visse ting, der er værd at overveje. En pærer har en Ra-værdi, som bruges til at bedømme hvor god en farvegengivelse pæren har. Ra-skalaen går helt op til 100, hvor det kun er sollys som har en Ra-værdi på 100, der er dog nogle typer af pærer, som næsten kan ramme de 100 Ra.\cite{halogen_paere}
En anden overvejelse er hvor energivenlig pæren skal være, da det svinger meget afhængigt af hvilken pærer der bruges. Ser man på nogle af fordelene ved LED pærer, så er de billige i drift da de har en lang levetid, på ca. 25 år, samt et lavt energiforbrug\cite{LED}, 4-5 gange så lidt, i forhold til halogenpæren, som kun har en levetid på ca. 2år\cite{vaelg_paere}. 
Der findes pærer, som eftergiver bedre end andre, og blandt toppen findes Halogen pæren, som kan komme op på 99 Ra, hvilket næsten giver perfekt lys\cite{halogen_paere}. 
Fælles for alle typer af pærer, kan kvaliteten svinge afhængig af hvilken producent. Men hvilken pærer der er bedst, er svært at sige. De har alle sine fordele og ulemper, men går man efter levetid er LED pæren bedst, samt der er mange penge at spare i løbet af de år. Halogen pæren er rigtig god til at eftergive farve, da den har en kelvin på ca. 2500-3000, samt en høj Ra-værdi. Er det en god grundbelysning, samt rimelig billig i indkøb samt drift, så er sparepæren en god løsning, undtagen hvis det er til udendørs brug, da pæren mister lys og levetid ved -20 grader\cite{sparepaerer}.

Det kan konkluderes udfra ovenstående, at kvaliteten af en lyskilde, afhænger af hvor lyset skal bruges, for de forskellige pærer er alle gode, afhængigt af hvor den placeres. 


\subsubsection{Lamper}
Formålet med dette afsnit er at afgrænse definitionen af hvad en lampe er, og hvordan begrebet skal forstås i rapporten.
Der findes mange forskellige definitioner på hvad en lampe faktisk er, og det viser sig ifølge American Heritage® Dictionary of the English Language \cite{american_heritage}, at begrebet ’lampe’ faktisk dækker over mange forskellige ting. American Heritage definerer en lampe som værende én eller flere af følgende:

En af flere forskellige enheder, der genererer lys og ofte varme, især:
\begin{enumerate}
    \item En elektrisk anordning, der har en sokkel til en pære, især et fritstående stykke møbel.
    \item En anordning, der afgiver ultravoilet, infrarød, eller anden stråling, som kan anvendes til terapeutiske formål.
    \item En pære: en projektør/et spot(light), udstyret med metalhalogenlampe.
    \item En lanterne eller armatur, der afgiver lys ved afbrænding af gas, ofte ved brug af en kappe.
\end{enumerate}

Idet der er så mange forskellige definitioner på en lampe, er vi, i konteksten af vores projekt, nødsaget til at afgrænse begrebet til noget mere specifikt. Da vi vil hjælpe forbrugeren, med at visualisere lampen i et givet rum, tager vi udgangspunkt i en mere normal lamp. Hvis man kigger på de tidligere definitioner af en lampe, kan man forestille sig utroligt mange apparaturer, som kan kaldes for en lampe. Lige fra ultraviolette lamper, der bruges i natklubber med fluoserende formål, til infrarøde lamper, der kan bruges i medicinske/terapeutiske sammenhænge, fx til at løsne og afspænde musklerne \cite{lys_terapi}. Der findes også lamper, der afgiver lys og varme ved afbrænding af fx gas, såsom en lanterne. I det videre arbejde med rapporten skal en lampe derfor opfattes som indendørs og udendørs lamper med pære og eventuelt lampeskærm til dagligsdagsbrug. Disse dagligdagslamper kunne fx være bordlamper, gulvlamper, loftlamper eller væglamper. 

\subsubsection{Afrunding}
Der er nu skabt kendskab til begreberne: Forbruger, sælger, lys og lampe. Disse begreber er relevante, da de er essentielle i forhold til det initierende problem samt problemfeltet. På baggrund af de redegjorte begreber er det muligt at skabe den forståelse, at det er forbrugeren der har et problem med at visualisere hvordan lys udbreder sig fra en lampe. Det er nu muligt at forsætte arbejdet med problemanalysen.




