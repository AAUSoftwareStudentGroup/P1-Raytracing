\subsection{Begrebsligørelse}
Der er indtil videre blevet  argumenteret for relevansen af det initierende problem, og det er i den sammenhæng derfor nødvendigt, at redegøre for nogle vigtige emner og ord indenfor problemfeltet. 

Formålet med dette afsnit er at beskrive vigtige begreber samt kort at give en beskrivelse af, hvordan de forskellige ord og begreber skal forstås i den videre rapport. Begreberne forbruger, visualisering, lys, pære og lampe, som fremgår i følgende afsnit, danner grundlag for forståelsen af det initierende problem.


% \subsubsection{Køber}
% En køber er en person, der køber et produkt eller tjenesteydelser \cite{ddo_forbruger}.  Køberen står altså i denne sammenhæng i modsætning til producenterne.
% I denne rapport opfattes køberen som den person der køber lampen. Det vil sige, at der i dette tilfælde ikke nødvendigvis er tale om personer, der til dagligt bruger eller bliver påvirket af lampen.
\subsubsection{Forbruger}
En forbruger er en privatperson, som køber et produkt eller tjenesteydelser. At “Forbruge” betyder at “bruge noget”, og en forbruger køber og/eller benytter derfor produkter med henblik på at tilfredsstille nogle behov \cite{forbrugerportalen}. En kunde hører også til under begrebet forbruger, da det er kunderne, som køber lamperne. En bevidst forbruger vil derfor ofte lede efter produkter, der opfylder deres behov. Man antages også for at være forbruger af en vare hvis man til daglig benytter sig af, eller bliver påvirket af en given lampe. 

I denne rapport udvider vi definitionen af forbruger til, at en forbruger også kan være en erhvervsperson, der køber en lampe til brug i virksomheden. 
 

%\subsubsection{Sælger}
%En sælger er den person eller virksomhed, der sælger et produkt eller %tjenesteydelser. I rapporten opfattes sælgeren som værende en person %eller virksomhed, der sælger lamper til forbrugeren. 

\subsubsection{Visualisering}
At visualisere betyder at skabe et billede på baggrund af noget \cite{ddo_visualisering}. Dette kan til dels være tanker, som omsættes til billeder for det indre øje. Det kan også være en række data, som omsættes til billeder, så de er nemmere at forstå.
Visualisering kan være et værktøj til at skabe en forståelse for det der visualiseres. Dette kan f.eks være prototyper af lamper, der kan give en forståelse for hvordan lyset udbreder sig fra en lampe. Derudover er der inden for computergrafik metoder til at skabe billeder på baggrund af 3D-modeller, så man f.eks. kan lave et delvist realistisk billede af en lampe. Dette billede kan hjælpe med at få en forståelse for, hvordan lampen ser ud i virkeligheden og hvordan dens lys udbredes. Forskellige teknologier til visualisering er uddybet senere i rapporten under afsnit \ref{sec:teknologianalyse}. 

\subsubsection{Lys}
Lys er en vigtig faktor i vores projekt, da alle lamper udsender en hvis form for lys og da en lampe, der ikke udsender lys ikke er specielt problematisk at visualisere.


Der er forskellige opfattelser af hvad lys indebærer. Hvis vi tager udgangspunkt i Karsten Rottwitt, som er professor ved DTU fotonik, så definerer han lys som:


“Lys er andet end synligt lys. For mig er lys et elektromagnetisk felt, som har en høj frekvens”
- Karsten Rottwitt\cite{def_lys}.

Han mener også, at der er en hårdfin grænse for hvornår lys kan betegnes som lys, denne grænse er dog først i spil når vi snakker om UV-lys og infrarødt lys \cite{def_lys}. 
Andre er ikke enige med Karsten Rottwitt om hvordan definitionen af lys er. Tager vi nu udgangspunkt i Britannica \cite{britannica_lys}, så betegnes lys, som magnetiske stråler, som det menneskelige øje kan opfange, også kaldet synligt lys. 


Det er denne definition, som rapporten vil tage udgangspunkt i. Dette er valgt, da det er oplagt at kombinere synligt lys og lamper.

Det lys, som kommer fra en lampe, er selvfølgelig af forskellig kvalitet. Kvalitet kan, ligesom lys, betegnes på mange måder, herunder kan vi snakke om hvorvidt en lyskilde er af god kvalitet, hvis den er energivenlig eller om det kun kommer an på hvor god den er til at eftergive farvet lys. 
Afhængigt af hvor man skal bruge lyset, kan nogle former for lys være bedre egnet end andre. Her menes der hvorvidt lyset skal være varmt eller koldt. Integral-LED er et firma med over 25 års erfaring\cite{integral_led}, og har opstillet nogle foretrukne steder at bruge de forskellige typer af lys:

Varm / varm hvid = Stue, soveværelse eller gange.
Hvid / kold hvid = Køkken, studie, badeværelser, skrivebord, kontor eller butikker\cite{varm_kold}.

Ud fra disse foretrukne placeringer, opstillet af integral-LED, kan vi antage at lyskvaliteten blandt andet afhænger af, hvor lyset skal bruges. Hvis det er i et stille og roligt miljø, med henblik på at slappe af, er det måske bedst at foretrække det varme lys, hvorimod de steder hvor det er nødvendigt at have skarpt lys for evt. at kunne se detaljer eller koncentrere sig, er det kolde lys at foretrække.


\subsubsection{Pærer}
I denne rapport forstås en pære som en enhed, der ved hjælp af elektricitet udsender lys. Herunder er der forskellige typer pærer, som vil blive beskrevet i følgende afsnit.
Da der findes så mange pærer, er der visse ting, der er værd at overveje. En pære har en Ra-værdi, som bruges til at bedømme hvor god en farvegengivelse pæren har. Ra-skalaen går helt op til 100, hvor det kun er sollys, som har en Ra-værdi på 100, der er dog nogle typer af pærer, som næsten kan ramme de 100 Ra \cite{halogen_paere}. Derudover kan farvetemperaturen af en lampe måles. Den viser noget om lysets farve, og om det er varmt eller koldt lys. Lysets farve er varmere, jo lavere temperatur det har, hvilket er målt i kelvin\cite{farvetemperatur}.

En anden overvejelse er hvor energivenlig pæren skal være, da det svinger meget afhængigt af hvilken pærer der bruges. Ser man på nogle af fordelene ved LED-pærer, så er de billige i drift, da de har en lang levetid på ca. 25 år, samt et lavt energiforbrug\cite{LED}, 4-5 gange så lidt, i forhold til halogenpæren, som kun har en levetid på ca. 2år\cite{vaelg_paere}. 
Der findes pærer, som eftergiver lys bedre end andre, og blandt toppen findes halogenpæren, som kan komme op på 99 Ra, hvilket næsten giver perfekt gengivelse af sollys\cite{halogen_paere}. 

Kvaliteten af de forskellige typer af pærer kan svinge alt afhængig af hvilken producent pærerne kommer fra, og det kan derfor være svært at sige hvilken type af pære som er bedst. De har alle sine fordele og ulemper, men går man efter levetid er LED-pæren bedst, derudover er der mange penge at spare i løbet af de år. Halogenpæren er rigtig god til at eftergive farve, da den har en meget høj Ra-værdi. Derudover har halogenpæren en farvetemperatur på ca. 2500-3000, som er målt i kelvin, hvilket viser, at halogenpæren udsender varme farver \cite{farvetemperatur}. 

Leder man derimod efter en pære med god grundbelysning til en rimelig pris, så er spærepæren en god løsning til indendørs brug, men ikke til udendørs brug, da pæren mister lys og levetid ved -20 grader.\cite{sparepaerer}.

Det kan konkluderes udfra ovenstående, at kvaliteten af en lyskilde, afhænger af hvor lyset skal bruges, for de forskellige pærer er alle gode, afhængigt af hvor de placeres. Udover kvaliteten af pæren, kan det antages at de forskellige pærer afgiver lys på forskellige måder og dermed kan det være svært at forudse hvordan lampen og lyset kommer til at se ud.  


\subsubsection{Lamper}
Formålet med dette afsnit er at afgrænse definitionen af hvad en lampe er i vores kontekst og hvordan begrebet skal forstås i rapporten.
Der findes mange forskellige definitioner på hvad en lampe er, og det viser sig ifølge American Heritage® Dictionary of the English Language \cite{american_heritage}, at begrebet ’lampe’ dækker over mange forskellige ting. 

American Heritage definerer en lampe som værende én eller flere af følgende:

En af flere forskellige enheder, der genererer lys og ofte varme, især:
\begin{enumerate}
    \item En elektrisk anordning, der har en sokkel til en pære, især et fritstående stykke møbel.
    \item En anordning, der afgiver ultravoilet, infrarød, eller anden stråling, som kan anvendes til terapeutiske formål.
    \item En pære: en projektør/et spot(light), udstyret med metalhalogenlampe.
    \item En lanterne eller armatur, der afgiver lys ved afbrænding af gas, ofte ved brug af en kappe.
\end{enumerate}

Idet der er så mange forskellige definitioner på en lampe, er vi, i konteksten af vores projekt nødsaget til at afgrænse begrebet til noget mere specifikt. Da vi vil hjælpe forbrugeren med at visualisere lampen i et givet rum, tages der udgangspunkt i en mere normal lampe. Hvis man kigger på de tidligere definitioner af en lampe, kan man forestille sig utroligt mange apparaturer, som kan kaldes for en lampe. Lige fra ultraviolette lamper, der bruges i natklubber med fluoserende formål, til infrarøde lamper, der kan bruges i medicinske/terapeutiske sammenhænge, f.eks. til at løsne og afspænde musklerne \cite{lys_terapi}. Der findes også lamper, der afgiver lys og varme ved afbrænding af f.eks. gas, såsom en lanterne. For at afgrænse alle disse definitioner vil en lampe i det videre arbejde med rapporten opfattes som en indendørs anordning, hvori der kan isættes en pære, som kan udsende lys, der evt. afskærmes af anordningen. Årsagen til at der afgrænses til at lamperne udelukkende skal være indendørs lamper skyldes, at der er langt flere typer af indendørs lamper, som specifikt  bruges til forskellige ting f.eks. skal en læselampe udsende et lys som gør, at lyset er behageligt at læse i.


% Skjuler opsumering fra indholdsfortegnelsen
% \addtocontents{toc}{\protect\setcounter{tocdepth}{2}}
\subsubsection*{Opsummering}
% \addtocontents{toc}{\protect\setcounter{tocdepth}{3}}
Ud fra de ovenstående afsnit i begrebsliggørelsen, kan der nu kortfattes, at der senere i denne rapport anvendes de omtalte begreber med følgende betydning:
\begin{enumerate}
	\item Forbruger: En person, der køber en lampe med henblik på brug i hjemmet eller i en virksomhed.
	%\item Sælger: En person eller virksomhed, der sælger produkter.
	\item Visualisering: Skabelsen af et billede på baggrund af noget, der evt. ønskes lettere forståeligt.
	\item Lys: Den elektromagnetiske stråling, der er synligt for øjet (Synligt lys).
	\item Pære: En enhed, der ved hjælp af elektricitet udsender lys.
	\item Lampe: En indendørs anordning hvor der kan isættes en pære, som udsender lys der evt. afskærmes af anordningen.
\end{enumerate}
Ud fra de ovenstående begreber skulle der nu være en entydig forståelse af det initierende problemet, som gør at problemet nu kan analyseres videre i de kommende afsnit.





