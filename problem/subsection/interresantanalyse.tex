\subsection{Interresantanalyse}
Indledning
\subsubsection{Hvem rammes af problemet?}
Dem der bliver ramt af problemet er brugerne, da det er dem der benytter sig af lampen til daglig. Brugeren er til dels også køberen i nogle tilfælde, men som nævnt før kan en lampe godt være købt af en anden person end den der bruger den.
Sælgeren bliver også indirekte ramt, da hvis mange kunder henvender sig fordi de gerne vil have en lampe returneret og byttet, så kræver det resourcer fra sælgeren side. Derfor vil det også være til sælgeren fordel at en kunde vil være i stand til at købe “den rigtige” lampe første gang.
 
\subsubsection{Hvem ønsker vi at løse problemet for?}
Vi vil løse problemet for enten sælgeren eller køberen, da det er der vi kan gøre mest. Ved sælgeren vil det kræve engagement fra den gældende virksomhed, da det vil kræve nogle ressourcer for at få dette implementeret enten på deres egen hjemmeside og så det ind i den fysiske butik på en anden måde. Dette vil også kræve at der bliver brugt ordentligt tid på at vise købere hvordan værktøjet fungerer og skabe opmærksomhed omkring den givne løsning til problemet.
Ved køberen ses det samme problem, her skal man også bruge tid på at informerer dem om brugen af værktøjet og i det hele taget sørge for at vi kan finde frem til værktøjet og ved det eksisterer.
 
\subsubsection{Hvordan påvirker det?}

Irritation?
 
Spørgeskema?
 
\subsubsection{Relationen mellem de forskellige interessenter}
For det meste er sælgeren hovedmål at sælge køberen en lampe, de går måske ikke så meget op i hvordan den kommer til at give lys hjemme ved køberen. Dog er sælgeren og køberen direkte forbundne da den ene ikke kunne eksistere uden den anden.  På samme måde vil brugeren heller ikke kunne eksistere uden køberen og derved sælgeren.
 
Kom kort ind på returret og hvilken rolle det spiller
Det der er problemet med returret er at for at finde ud af om lampen reelt set passe ind og giver det rigtige i forhold til boligen bliver man nødt til at hænge den op. Idet man hænger den op er man nødsaget til at tage lampen ud af original emballage og hænge den op og tænde den, og så vil lampen ikke længere være ubrugt, og derfor vil den ikke kunne returneres da den nu er blevet brugt selvom det kun var én gang og man stadig har kvitteringen \cite{ikea_returret}.
