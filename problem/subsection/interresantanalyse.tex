\subsection{Interresantanalyse}

 Dette afsnit vil undersøge hvilke interresanter der er, da det er vigtigt at få en forståelse for de involverede parter. Dette afsnit vil omhandle dem der bliver ramt af problemet og hvem de er, det vil også omhandle hvordan dette problem påvirker dem samt relationen mellem forbruger og sælger. I rapporten opfattes sælgeren som værende en person eller virksomhed, der sælger lamper til forbrugeren. 

\subsubsection{Hvem bliver ramt af problemet?}
Dem der bliver ramt af dette problemet er brugerne, da det er dem der benytter sig af lampen til daglig. En bruger kan i nogle tilfælde også være personen der har købt lampen men som nævnt før kan en lampe godt være købt af en anden person end den der bruger den.

Sælgeren bliver også indirekte ramt af dette problem. Hvis mange kunder er utilfredse med deres lampekøb vil sælgeren have dårlig kundetilfredshed, og dette kan påvirke det samlede lampesalg. Det vil derfor være til sælgerens fordel hvis en kunde vil være i stand til at købe "den rigtige" lampe første gang, da dette ville medføre tilfredse kunder.

% Sælgeren bliver også indirekte ramt, hvis mange kunder henvender sig fordi de gerne vil have en lampe returneret og byttet, så kræver det resourcer fra sælgerens side. Derfor vil det også være til sælgerens fordel, at en kunde vil være i stand til at købe “den rigtige” lampe første gang eller en lampe, der er lidt dyrere fordi den giver den ønskede effekt.
 
\subsubsection{Målgruppen}
Som beskrevet i forrige afsnit, er det både sælgere og brugere, der rammes af problemet. Det er nu relavant at afgøre hvem problemløsningen retter sig mod, da dette danner grundlag for hvordan løsningen skal udvikles. 

Hvis man retter problemløsningen mod sælgeren, og laver en løsning der gør det muligt for kunder at visualisere lamperne bedre, så vil kunderne højst sandsynligt være mere tilfredse med deres lamper, da de har mulighed for at se lampens belysning inden købet. For sælgeren vil dette betyde, at kunden ikke vil returnere ligeså mange lamper, og dette vil bidrage til øget kundetilfredshed, som i sidste ende gavner både sælgeren og kunderne.

% Skjuler opsumering fra indholdsfortegnelsen
\addtocontents{toc}{\protect\setcounter{tocdepth}{1}}
\subsubsection{Opsummering}
\addtocontents{toc}{\protect\setcounter{tocdepth}{2}}
I dette afsnit er der blevet argumenteret for, at det er sælgere og brugere der bliver ramt af problemet, men at det vil være mest fordelagtigt at rette løsningen mod sælgeren, da dette også løser problemet for brugerne.

Dette afsnit er relevant i forhold til den senere problemformulering, da der er blevet argumenteret for hvem det er som har problemet, samt hvilken målgruppe det senere produkt skal udvikles til.
