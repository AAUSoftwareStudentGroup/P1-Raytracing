\subsection{Interresantanalyse}

 Dette afsnit vil undersøge hvilke interresanter der er, da det er vigtigt at få en forståelse for de involverede parter. Dette afsnit vil omhandle dem der bliver ramt af problemet og hvem de er, det vil også omhandle hvordan dette problem påvirker dem samt relationen mellem forbruger og sælger. Da det i dette afsnit er relevant at tale om relationen mellem sælger og forbruger, så vil sælgeren kort blive introduceret i dette afsnit.

 \subsubsection{Sælger}
En sælger er den person eller virksomhed der sælger et produkt eller tjenesteydelser. I rapporten opfattes sælgeren som værende en person eller virksomhed, der sælger lamper til forbrugeren. 

\subsubsection{Hvem bliver ramt af problemet?}
Dem der bliver ramt af dette problemet er brugerne, da det er dem der benytter sig af lampen til daglig. Brugeren er til dels også køberen i nogle tilfælde, men som nævnt før kan en lampe godt være købt af en anden person end den der bruger den.

Sælgeren bliver også indirekte ramt, da hvis mange kunder henvender sig fordi de gerne vil have en lampe returneret og byttet, så kræver det resourcer fra sælgeren side. Derfor vil det også være til sælgeren fordel at en kunde vil være i stand til at købe “den rigtige” lampe første gang eller en lampe, der er lidt dyrere fordi den giver den ønskede effekt.
 
\subsubsection{Målgruppen}
Vi vil løse problemet for sælgeren, da det er der vi kan gøre mest. Vi kan gøre mest ved sælgeren da en virksomhed kan have flere hundrede kunder, og hvis man derfor gav løsningen til sælgeren i stedet for forbrugeren så kan man slå flere flyer med et smæk. Dette vil kræve at der bliver brugt ordentligt tid på at lave gode forklaringer eller guides til hvordan værktøjet fungerer og skabe opmærksomhed til løsningen så den bliver brugt.
 
