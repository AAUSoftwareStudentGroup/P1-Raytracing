\subsection{Interessentanalyse}
Dette afsnit vil undersøge hvilke interessenter der er, da det er vigtigt, at få en forståelse for de involverede parter. Dette afsnit vil omhandle dem, der bliver ramt af problemet og hvem de er, det vil også omhandle hvordan dette problem påvirker dem, samt relationen mellem forbruger og sælger. I rapporten opfattes sælgeren som værende en person eller virksomhed, der sælger lamper til forbrugeren. 

\subsubsection{Hvem bliver påvirket af problemet?}


Sælgeren bliver også indirekte ramt af dette problem. Hvis mange kunder er utilfredse med deres lampekøb, vil sælgeren have dårlig kundetilfredshed, og dette kan påvirke det samlede lampesalg. Det vil derfor være til sælgerens fordel hvis en kunde ville være i stand til at købe "den rigtige" lampe første gang, da dette ville medføre tilfredse kunder.

% Sælgeren bliver også indirekte ramt, hvis mange kunder henvender sig fordi de gerne vil have en lampe returneret og byttet, så kræver det ressourcer fra sælgerens side. Derfor vil det også være til sælgerens fordel, at en kunde vil være i stand til at købe “den rigtige” lampe første gang eller en lampe, der er lidt dyrere fordi den giver den ønskede effekt.

\subsubsection{Konteksten}
[SKRIV PARAGRAF OM KOMMUNIKATION MED KONTEKSTEN]()
\subsubsection{Designere}
[SKRIV PARAGRAF OM KOMMUNIKATION MED DESIGNERE]()
\subsubsection{Producenter}
[SKRIV PARAGRAF OM KOMMUNIKATION MED PRODUCENTER]()
\subsubsection{Sælgere}
[SKRIV PARAGRAF OM KOMMUNIKATION MED SÆLGERE]()
\subsubsection{Kunden}

\subsubsection{Brugere}
Det er brugerne der i sidste ende benytter sig af lamperne i deres hjem eller på deres arbejde. Dette gør brugerne til den gruppe af interessenter, som påvirkes direkte af problemet, da de må leve med konsekvenserne, som belysningen fra en lampe kan medføre[HENVIS TIL AFSNIT OM KONSEKVENSER VED LYS]. Dette kan bl.a. være kontorarbejdere i en virksomhed, som påvirkes, hvis lamperne på deres kontor ikke passer sammen med den indretning der. Et andet eksempel på brugere, er hjemme i privaten, hvor der kan være mange forskellige typer lamper, som skal passe ind i hjemmet. Hvis en person i hjemmet køber en lampe, som har en belysning, der ikke passer ind i hjemmet pga. manglende visualisering ved købet, så vil dette påvirke brugerne i hjemmet. 

\subsubsection{Målgruppen}
Som beskrevet i forrige afsnit, er det både sælgere og brugere, der rammes af problemet. Det er nu relevant at afgøre hvem problemløsningen retter sig mod, da dette danner grundlag for hvordan løsningen skal udvikles og hvem der kan indrages i løsningen og udarbejdelsen af løsningsforslaget. 


Hvis man retter problemløsningen mod sælgeren, og laver en løsning der gør det muligt for kunder at visualisere lamperne bedre, så vil kunderne højst sandsynligt være mere tilfredse med deres lamper, da de har mulighed for at se lampens belysning inden købet. For sælgeren vil dette betyde, at kunden ikke vil returnere lige så mange lamper, og dette vil bidrage til øget kundetilfredshed, som i sidste ende gavner både sælgeren og kunderne.

\subsubsection*{Opsummering}
I afsnittet er der blevet argumenteret for, at det er sælgere og brugere der bliver ramt af problemet, men at det vil være mest fordelagtigt at rette løsningen mod sælgeren, da dette også løser problemet for brugerne.

Dette afsnit er relevant i forhold til den senere problemformulering, da der er blevet argumenteret for hvem det er, som har problemet, samt hvilken målgruppe det senere produkt skal udvikles til.
