\subsection{Interresantanalyse}

 Dette afsnit vil undersøge hvilke interresanter der er, da det er vigtigt at få en forståelse for de involverede parter. Dette afsnit vil omhandle dem der bliver ramt af problemet og hvem de er, det vil også omhandle hvordan dette problem påvirker dem samt relationen mellem forbruger og sælger.

\subsubsection{De ramte}
Dem der bliver ramt af dette problemet er brugerne, da det er dem der benytter sig af lampen til daglig. Brugeren er til dels også køberen i nogle tilfælde, men som nævnt før kan en lampe godt være købt af en anden person end den der bruger den.

Sælgeren bliver også indirekte ramt, da hvis mange kunder henvender sig fordi de gerne vil have en lampe returneret og byttet, så kræver det resourcer fra sælgeren side. Derfor vil det også være til sælgeren fordel at en kunde vil være i stand til at købe “den rigtige” lampe første gang.
 
\subsubsection{Målgruppen}
Vi vil løse problemet for enten sælgeren eller køberen, da det er der vi kan gøre mest. Ved sælgeren vil det kræve engagement fra den gældende virksomhed, da det vil kræve nogle ressourcer for at få en løsning implementeret enten på deres egen hjemmeside og så det ind i den fysiske butik på en anden måde. Dette vil også kræve at der bliver brugt ordentligt tid på at vise købere hvordan værktøjet fungerer og skabe opmærksomhed omkring den givne løsning til problemet.
Ved køberen ses det samme problem, her skal man også bruge tid på at informerer dem om brugen af værktøjet og i det hele taget sørge for at vi kan finde frem til værktøjet og ved det eksisterer.
 
\subsubsection{Hvordan påvirker det?}

Irritation?
 
Spørgeskema?
 
\subsubsection{Relationen mellem de forskellige interessenter}
For det meste er sælgerens hovedmål at sælge en kunde en genstand, hvilket i vores tilfælde er en lampe. De går måske ikke så meget op i hvordan den kommer til at give lys hjemme ved forbrugeren, da de ikke ved hvordan forbrugerens hjem ser ud, derfor giver den bedste vejledning ud fra den information de har. Sælgeren og forbrugeren er direkte forbundne da den ene ikke kunne eksistere uden den anden. 
