\subsection{Loesevejledning}
Rapporten er skrevet med en rød tråd, hvilket vil sige, at nogle afsnit er sat før andre, så man kan få en bedre forforståelse for en hvis ting, inden man læser videre. Det er dog ikke nødvendigt at læse hele rapporten, da hvert afsnit har sin egen indledning og opsummering hhv. først og sidst i afsnittet.


\subsubsection{Kildehenvisning}
Rapportens brug af kildehenvisninger er baseret på nummer-metoden. I nummer-metoden anføres kilderne i fortløbende nummerorden, svarende til hvilket nummer, de har i teksten. To identiske kilder har samme nummer. Herunder ses et eksempel, på en kildehenvisning:

Eksempel[1].

[1] Titel på emne, hjemmesidenavn, set DD-MM-YYYY.
	URL på webside.

Figurhenvisning foregår på samme måde som med andre kilder, dog med en forklaring under selve figuren. Hvis en figur ingen kilde har, er figuren fremstillet af gruppen.


\subsubsection{Visning af kode}
Flere steder i rapporten, vil der blive vist dele af gruppens kode. Et eksempel på, hvordan dette vil blive vist er herunder:

\begin{lstlisting}[language=C, caption=Kodeeksempel i C]
#include <stdio.h>

int main(void){
   printf("Hello, world!\n");
   return 0;
}
\end{lstlisting}