\subsubsection{3D Vektorer}

<Beskrivelse af 3D Vektorer>


\subsubsection{Rotationsmatricer}
Hvis vi vil roterer et punkt eller en vektor omkring nul-punktet i et koordinatsystem kan vi bruge en rotationsmatrix \cite(rotationsmatricer). En rotationsmatrix i 2D består af:
[cos θ, -sin θ
 sin θ,  cos θ]
indsætter vi mængden af radianer vi vil dreje vores vektor og ganger dem sammen, ser vi at vektoren bliver drejet omkring nul-punktet, med netop den mængde radianer. 
Vil vi roterer en vektor/punkt i 3D bliver det lidt mere komplekst. Vi vil så roterer 2 af punkterne i et plan omkring det sidste punkt, som dermed også virker som normalvektor til planet. Formlerne for at gøre dette er udviklet og frit tilgængelige på blandt andet Wikipedia's side:
<indsæt 3D_vektor_rotationsmatrix.png her og henvis til wikipedia artiklen det er lånt fra: https://en.wikipedia.org/wiki/Rotation_matrix>
Her skal vi, ligesom med 2D rotationsmatricer, bare indsætte vinklen og gange dem sammen med vektoren vi vil dreje.