\subsection{Problemets placering}
I dette afsnit undersøges det hvor og i hvilke situationer problemet opstår for målgruppen. Da problemet tager udgangspunkt i købet af lampen, beskrives der, hvordan købssituationen er ved hhv. handel i fysiske butikker (Detail-handel) og handel i internetbaserede butikker (e-handel). Ud fra disse undersøgelser sammenlignes detail-handel og e-handel, for at afgøre hvor problemet er størst, og derudfra vælge den hvilken af type handel, som denne rapport ønsker at løse problemet indenfor.

\subsubsection{Detailhandel}
En fysisk butik er et sted hvor kunderne selv skal komme hen, når de vil købe eller kigge på butikkens varer. En fysisk butik har et ansat personale, som tager sig af butikkens kunder, og kan besvare deres mulige spørgsmål til butikkens varer. En fysisk butik er grundet det ansatte personale mm. dyr i omkostning, men kunderne foretrækker de fysiske butikker, da man kan se den virkelige varere selv og få direkte assistance om varen gennem en medarbejder\cite{fysisk_kontra_online}. Det kræver dog stadig at kunden selv skal ud til butikken, og være sikker på dens åbningstider og varelager.

I vores kontekst snakker vi om en hvilken som helst fysisk butik, der har med lampesalg at gøre. Den lampeinteresserede kunde, kommer ud i butikken, og leder fx efter en ny væglampe til stuen. Problemet heri kan opstå, ved at der er adskillige forskellige lamper at vælge imellem, men ikke alle lamperne er tilsluttet, så man kan se hvordan lyset falder. 
\newline Ved køb af lamper i den fysiske butik, er det ikke altid, at de lamper, som stilles frem til udstilling giver et realistisk billede af, hvordan lyset udbreder sig, da der ofte er mange lamper og lyskilder tæt samlet. Et andet problem er returretten. Returretten er ikke obligatorisk at have for fysiske butikker, så det er altså op til den enkelte butik/butikskæde, om de vælger at gøre det muligt at returnere en vare, selvom den er uåbnet og stadig i original emballage\cite{fortrydelsesret}. Hvis kunden beslutter sig for en lampe, som ikke er tilsluttet, men regner med at den vil se godt ud på væggen i stuen, hvorefter det så viser sig, at lyset falder helt forkert og er alt for skarpt og blændende er det for sent, da lampen er pakket ud, og ledningen er blevet pillet ved. Lampen kan altså ikke byttes, og er ikke optimal i forhold til kundens stue. Der er mange butikker og butikskæder, som vælger at gøre det muligt for kunden at bytte en vare, hvis den stadig er i original emballage og med kvittering \cite{ikea_returret}, men det er ikke noget, som butikker er tvunget til at gøre, og med en lampe som eksempel, kan man ikke tage den med hjem og ’afprøve’ den, uden at annullere sin returret, da den bliver pakket ud og ledningerne bliver pillet ved.


\subsubsection{E-handel}
\label{sec:ehandel}
E-handel er elektronisk handel via internettet\cite{ddo_ehandel}. På internettet kan sælgere inden for e-handel have såkaldte e-butikker, hvor kunder kan købe varer\cite{ddo_ebutik}. E-butikker er ofte udformet således at kunden kan se billeder og informationer omkring sælgerens varer og derudfra kan kunden vælge at lægge varerne i en virtuel indkøbskurv, hvor kunden til sidst indtaster de nødvendige oplysninger for at købe og modtage varerne.


Blandt de mange forskellige varer, der sælges via e-butikker, er det her relevant at tale om e-handel med lamper. Nedenstående figur \ref{fig:e_handel_med_lamper} illustrerer princippet bag en lampesælgers salg af lampe til en kunde via en e-butik.
\begin{figure}[H]
	\centering
	\def\svgwidth{\columnwidth}
	\input{./graphics/e_handel_med_lampe.pdf_tex}
	\caption{Princippet bag handel af en lampe via en e-butik.}
    \label{fig:e_handel_med_lamper}
\end{figure}

På figur \ref{fig:e_handel_med_lamper} er det vist hvordan e-handlen starter med at kunden får et udvalg af lamper fra e-butikken. Kunden sender så en bestilling, som via e-butikken sendes videre til lampesælgeren, og til sidst sendes lampen til kunden. Dog ender handlen ikke nødvendigvis her, da kunden kan sende lampen retur såfremt at gældende lovgivning og købsbetingelser muliggører dette. For at undersøge lovgivningen nærmere kan man tage udgangspunkt i den danske lov om forbrugeraftaler\cite{retsinformationen}.

I lovens kapitel 1, § 1, stk. 2, nr. 1, fremgår der at lovens bestemmelser for fortrydelsesret gælder for aftaler, som er indgået ved fjernsalg. For en  fjernsalgsaftale gælder der, at aftalen om varer, er indgået gennem fjernkommunikation, hvor den erhvervsdrivende og forbrugeren ikke mødes fysisk (jf. kap. 1, § 3, nr. 1).

Ser man nu på loven i forbindelse med e-handel, foregår fjernkommunikationen gennem internettet via e-butikken, hvor fjernsalgsaftalen udføres i form af brugerens bestilling af f.eks. en lampe. Dette gør at fortrydelsesretten gælder ved e-handel.

Fortrydelsesretten er en forbrugers mulighed for at melde sig ud af en aftale, herunder køb af lamper ved e-handel. Hvis en en forbruger eksempelvis køber en lampe via en e-butik, har forbrugeren mulighed for at fortryde købet inden 14 dage ved at meddele dette til den erhvervsdrievende (jf. kap. 4, § 19). Herefter har forbrugeren 14 dage til at returnere varen (jf. kap. 4, § 24). Hvis varens værdi er forringet som følge af forbrugerens unødvendige håndtering af varen for at inspicere denne, så hæfter forbrugeren for værdiforringelsen (jf. kap. 4, § 24, stk. 5). Dvs. at hvis en bruger installerer og bruger lampen, hvor der f.eks. tilpasses ledninger, så kan lampens værdi forringes og forbrugeren skal hæfte for dette. 

\subsubsection{Sammenligning af detail- og e-handel}
Ud fra ovenstående redegørelse af de to typer for handel, analyseres disse nu med henblik på at finde ligheder og forskelle, hvoraf det kan afgøres i hvilken af de to typer af handel, at problemet er størst. 

Da det initierende problem er at forbrugeren ikke kan visualisere lampen uden at købe den, er det derfor relevant at se på i hvor høj grad dette er tilfældet ved de to typer handler.

Fordelen ved detail-handel, er at forbrugeren ofte kan se lampen i butikken, og ud fra dette, vurdere hvilken lampe der opfylder de behov som forbrugeren har. Dog er problemet stadigvæk at forbrugeren ikke ser lampen i den rette kontekst, dvs. i sit eget hjem. Dette kan gøre at forbrugeren får et godt indtryk af lampen i den kontekst, som butikken præsenterer den i, men at den ikke passer ind i den kontekst, som forbrugeren køber lampen til.

Ved e-handel har forbrugeren ikke muligheden for at se en fysisk udgave af lampen, men ofte kun billeder. Dette gør at forbrugeren alene kan tage valg ud fra de billeder og informationer, som e-butikken præsenterer. Problemet er så, at billederne til dels ikke er interaktive, dvs. brugeren ikke kan se lampen fra flere vinkler end dem som billederne er taget i, samt at billederne ikke er taget af lampen i den kontekst, som forbrugeren ønsker at købe lampen til. 

Med hensyn til konteksten er fordelen ved e-handel, at forbrugeren kan sidde derhjemme, i den kontekst, hvor lampen skal indgå, og sammenligne med de informationer, der er tilgængelige på e-butikken. I modsætning til dette er detail-butikker, hvor forbrugeren står i butikken, og måske har problemer med at huske eller blot forestille sig alle detaljerne ved den kontekst, som lampen skal indgå i.

Ud fra denne sammenligning, er der på den ene side detail-handel, hvor det er svært at visualisere konteksten, men hvor man kan se lampen. På den anden side er e-handel, hvor man kan sidde derhjemme i konteksten, men har svært ved at visualisere lampen. 

For at afgøre hvilken type handel denne rapport vil fokusere på, skal der derfor svares på om det er mangel på visualisering af lampen i kontekst ved detail-handel eller mangel på visualisering af lampen ved (e-handel), som er det største problem.

Da forbrugeren omgås og ser den kontekst, som lampen skal indgå i f.eks. et kontor, køkken og bad, så må man kunne antage at forbrugeren har en forestilling om, hvordan denne kontekst ser ud selvom forbrugeren ikke står i den når der handles i en fysisk butik. Derfor er dette ikke et lige så stort problem, som hvis forbrugeren ikke kan visualisere lampen når der handles via e-handel. Derfor vil fokuset i denne rapport være at forbedre forbrugerens evne til at visualisere lamper under e-handel.

\subsubsection{Opsummering}
Trods at problemet opstår hos forbrugeren, altså at de har problemer med at visualisere den givne lampe fra alle vinkler og i det rette miljø, er der ingen måde, hvorpå forbrugeren selv kan løse dette problem på en let og effektiv måde. Vi vil derfor fokusere på hvordan  problemet kan blive løst, allerede før varen når ud til forbrugeren. Derfor er vi nødt til at fokusere på sælgeren (e-handelsbutikker), når det kommer til løsningen af viasuleringsproblemet. Her kunne det tænkes, at der kunne udarbejdes et værktøj, der ville forbedre og optimere forbrugerens visualisering af den givne lampe.





