\subsection{Teori}
\label{sec:teori}

Dette afsnit omhandler de teorier samt videnskabelige principper som er relevante i forhold til at lave den løsning som er beskrevet i afsnit \ref{sec:losning}. Der vil derfor i afsnittet være fokus på de principper, formler og algoritmer som der bruges i udviklingsfasen. Teoriafsnittet vil primært fokusere på at beskrive og undersøge tekniske og matematiske principper som ellers ville være svære at forstå og arbejde med. Ud fra kravene til løsningen i afsnit \ref{sec:krav} har gruppen valgt at undersøge og diskutere følgende fænomener og principper: Rotationsmatricer i afsnit \ref{sec:rot_matricer}, konvertering af farvetemperatur til RGB-værdier i afsnit \ref{sec:temptilrgb}, fra 3D-model til billede i afsnit \ref{sec:fra_model_til_billede} herunder 3D-model, kamera, perspektiv projektion, backwards raytracing, phong-modellen og til sidst skæring mellem linje og trekant i rummet i afsnit \ref{sec:triangle_intersection}. I slutningen af afsnittet vil der fremgå, hvordan de enkelte teorier kan bidrage til udviklingen af løsningen.

Hvis vi vil roterer et punkt eller en vektor omkring nul-punktet i et koordinatsystem kan vi bruge en rotationsmatrix\cite{rotationsmatricer}.
En rotationsmatrix er en matrix der, hvis ganget sammen med en anden matrix, roterer en vektor eller et punkt i et koordinatsystem. 

\begin{align}\label{eu_eqn}
R_x(\theta) = 
\begin{bmatrix}
1 & 0 & 0\\ 
0 & cos \theta & - sin \theta\\ 
0 & sin \theta & cos \theta
\end{bmatrix}\\
R_y(\theta) = 
\begin{bmatrix}
cos \theta  & 0 & sin \theta\\ 
0           & 1 & 0\\ 
-sin \theta & 0 & cos \theta
\end{bmatrix}\\
R_z(\theta) = 
\begin{bmatrix}
cos \theta & - sin \theta & 0\\ 
sin \theta & cos \theta & 0\\
0 & 0 & 1
\end{bmatrix}
\end{align}

Hvis vi vil rotere en vektor/et punkt i 3D bliver det lidt mere komplekst. Vi vil så roterer 2 af punkterne i et plan omkring det sidste punkt, som dermed også virker som normalvektor til planet. Formlerne for at gøre dette er udviklet og frit tilgængelige på blandt andet Wikipedias side om rotationsmatricer:

Indsætter vi mængden af radianer vi vil dreje vores vektor og ganger dem sammen, ser vi at vektoren bliver drejet omkring nul-punktet med netop den mængde radianer.
[Indsæt eksempel]

\subsubsection{Konvertering fra farvetemperatur til RGB}
\label{sec:temptilrgb}
Farvetemperatur, som også er beskrevet nederst i \ref{sec:lys}, er temperaturen af et udsendt lys og måles i kelvin. Denne temperatur kan bruges til at finde ud af, om et lys er varmt eller koldt. 
RGB-værdien er en værdi for en given farves indhold af rød, grøn og blå. Værdien angives normalt ved et tal mellem 0 og 255, altså er 0 128 255 ingen rød, en del grøn og fuld blå, hvilket, blandet sammen, giver en blålig farve.
Der findes ingen direkte og 100\% præcis formel for at ’oversætte’ en kelvintemperaturværdi til en RGB-værdi, derfor har rapporten taget udgangspunkt i en algoritme, som er lavet ud fra 400 målinger, men som stadig ikke er præcis nok til videnskabelig brug.
Måden hvorpå algoritmen er lavet, er ved at tage disse 400 målinger, og lave en funktion ud fra dem. Der er lavet én måling per 100 kelvin, der starter ved 1000 kelvin og slutter ved 40.000 kelvin. Ved at kigge på funktionen \cite{tanner_helland_chart} har Tanner Helland kunne konkludere  tre ting:

\begin{itemize}
\item Røde værdier under 6600 kelvin er altid 255.
\item Blå værdier under 2000 kelvin er altid 0.
\item Blå værdier over 6500 kelvin er altid 255.
\end{itemize}

Disse tre, forholdsvis simple, konklusioner har hjulpet med at gøre algoritmen meget kortere og mere simpel. Herunder kan udregningerne for hhv\. rød-, grøn- og blåværdierne ses matematisk, før de er skrevet om til kode. Matematikken er vist gennem gaffelfunktioner, altså funktioner med forskellige funktionsudtryk for bestemte intervaller.


\begin{displaymath}
   RED = \left\{
     \begin{array}{lr}
       255: & 1000 <= \text{k and k} <= 6600\\
       329.698727446*(k-60^{-0.1332047592}): & 6600< \text{k and k} <= 40000
     \end{array}
   \right.
\end{displaymath} 

\begin{displaymath}
   GREEN = \left\{
     \begin{array}{lr}
       99.4708025861*\ln(k)-161.1195681661: & 1000 <= \text{k and k} <= 6600\\
       288.1221695283*(k-60^{-0.0755148492}): & 6600< \text{k and k} <= 40000
     \end{array}
   \right.
\end{displaymath} 

\begin{displaymath}
   BLUE = \left\{
     \begin{array}{lr}
       255: & 6600 <= \text{k and k} <= 40000\\
       0: & 1000 <= \text{k and k} <= 1900\\
       138.5177312231 * \ln(k-10) - 305.0447927307: & 1900 < \text{k and k} < 6600
     \end{array}
   \right.
\end{displaymath} 

Grunden til at der oversættes fra farvetemperatur til RGB-værdi, er for at kunne visualisere farverne på en computer. Et billede vist på en computer består af pixels, som alle har en RGB-værdi, derfor kan lyset fra en lampe med en given farvetemperatur visualiseres, hvis farvetemperaturen oversættes til RGB-værdi.













\subsubsection{Fra 3D-model til billede}
\label{sec:Fra_model_til_billede}
I dette afsnit er det vist, hvordan der kan udledes en model, der beskriver en billeddannelsen af objekter i rummet, også kaldt rendering. Dette er essentielt da billeddannelsen danner grundlag for, hvordan 3D-modellen for en lampe omdannes til et billede, der kan vises for kunderne på e-butikken. Til sidst i afsnittet udledes en model for, hvordan belysningen fra en lampe kan simuleres og visualiseres vha. raytracing. 

\paragraph{3D-model}
En 3D-model er en matematisk beskrivelse af et tre dimensionelt objekt. For at beskrive et 3D-objekt opdeler man ofte objektet i trekanter. Dette er illustreret på nedenstående figur.

\begin{figure}[H]
\label{fig:kanin}
    \centering
    \includegraphics[width=5cm]{kanin}
    \caption{Eksempel hvor trekanter bruges til at repræsentere et objekt: http://www.cs.rpi.edu}
\end{figure}

\paragraph{Kamera}
[BESKRIV HVORDAN KAMERAET ER EN MODEL FOR HVORDAN DER LAVES ET BILLEDE AF ET KAMERA]
For at beskrive kameraet er det nødvendigt at fastlægge dets position og orientering i rummet[KILDE]. Hvordan dette kan modelleres er vist på figur [INDSÆT REF]

[LAV FIGUR DER VISER VEKTORER DER BESKRIVER KAMERAET]
\begin{figure}[H]
  \label{fig:kamera}
  \centering
  \tdplotsetmaincoords{60}{130}
\begin{tikzpicture}[tdplot_main_coords]
\path[fill=gray!20, draw=gray!40] (-2,-4,-2) -- (-2,-4,2) -- (2,-4,2) -- (2,-4,-2) -- (-2,-4,-2);
\draw (0,0,0) -- (0,-4,0);
\draw[blue!50, thick, -{Stealth[width=2mm, length=2mm]}] (0,0,0) -- (0,-2,0);
\draw[blue!50, thick, -{Stealth[width=2mm, length=3mm]}] (0,-4,0) -- (-2,-4,0);
\draw[blue!50, thick, -{Stealth[width=2mm, length=2mm]}] (0,-4,0) -- (0,-4,2);
\draw plot [mark=*, mark size=2] coordinates{(0,0,0) }; 
\node [above right] at (0,0,0) {$C$};
\node [above right] at (0,-1,0) {$\vv{f}$};
\node [above] at (-1,-4,0) {$\vv{r}$};
\node [left] at (0,-4,1) {$\vv{u}$};
\end{tikzpicture}
  \caption{Viser hvordan kameraet kan beskrives ved tre vektorer $\protect\vv{r}$, $\protect\vv{u}$ og $\protect\vv{f}$, der repræsentere hhv. op, højre og frem retninger for kameraet.}
\end{figure}
[FIGUR TILFØJ VEKTORER DER BESKRIVER PUNKTET B, SOM LINEAR KOMBINATION. TILFØJ NORMALVEKTOR FRA BILLEDPLANEN TIL KAMERAET]

\paragraph{Perspektiv projektion}
For at udlede en model for billeddannelsen, tages der udgangspunkt i en perspektiv projektion. Perspektiv projektion er en måde at danne et billede af 3D-objekter ved at projektere objekterne hen på et plan mod et kameraes position\cite{fig:perspective_projection}. Princippet bag perspektiv projektion er vist på figur \ref{fig:perspektiv_projektion}.

\begin{figure}[H]
  \label{fig:perspektiv_projektion}
  \centering
  \tdplotsetmaincoords{60}{130}
\begin{tikzpicture}[tdplot_main_coords]
\path[fill=blue!50, draw=gray!20] (2,-8,2) -- (-2,-8,2) -- (0,-8,0) -- (2,-8,2);
\draw (0,0,0) -- (2,-8,2);
\path[fill=gray!20, draw=gray!40] (-2,-4,-2) -- (-2,-4,2) -- (2,-4,2) -- (2,-4,-2) -- (-2,-4,-2);
\path[fill=blue!50, draw=gray!20] (1,-4,1) -- (-1,-4,1) -- (0,-4,0) -- (1,-4,1);
\draw (0,0,0) -- (1,-4,1);

\draw plot [mark=*, mark size=2] coordinates{(2,-8,2) } ; 
\draw plot [mark=*, mark size=2] coordinates{(1,-4,1) }; 
\draw plot [mark=*, mark size=2] coordinates{(0,0,0) }; 
\node [above left] at (2,-8,2) {$P$};
\node [above left] at (1,-4,1) {$B$};
\node [above right] at (0,0,0) {$C$};
\end{tikzpicture}
  \caption{Viser princippet bag perspektiv projektion af et punkt på et billedplan.}
\end{figure}

[TEGN TO PUNKTER MERE PÅ FIGUREN OVENOVER OG TAG UDGANGSPUNKT I AFBILDNING AF EN TREKANT I STEDET FOR ET PUNKT.]

Som vist på figur \ref{fig:perspektiv_projektion} kan et punkt $P\in \mathbb{R}^3$ projekteres ned på billedplanen $\alpha$ ved at finde skæringspunktet $B$ mellem billedplanen $\alpha$ og en lysstråle $L$, som går fra punktet $P$ mod kameraets position $C$. Gør man nu dette for alle punkter på et objekt i rummet, og tegner skæringspunkterne på billedplanen, dannes et billede af objektet. For at omdanne 3D-objekter i rummet til et billede, er det nødvendigt at have en model for kameraet der danner billedet. 

Udfordringen er så at afgøre hvilken farve punkterne på billedplanen skal have, da dette afhænger af objektets egenskaber, samt hvilket udefrakommende lys der rammer objektet. 

For at løse denne udfordring, benytter vi i dette projekt raytracing, der som beskrevet under afsnit \ref{sec:computergrafik}, bygger på at simulere lysstrålers interaktion med forskellige objekter i rummet. Hvordan dette fungere er beskrevet i næste afsnit, hvor der er beskrevet en model for backwards raytracing.

\paragraph{Backwards raytracing}
I modsætning til en perspektiv projektion af et punkt på et plan, er backwards raytracing, hvor man i stedet for punktet i rummet, tager udgangspunkt i de lysstråler der danner billedet. Ved backwards raytracing følger man lysstrålerne baglæns og ser på, hvor stor en lysintensitet, den pågældende lysstråle har efter den har interageret med objekterne i rummet. Ud fra dette farves det tilhørende punkt på billedet, og på den måde kan man rendere et helt billede. På figur \ref{fig:raytracing_skitse} er det vist hvordan man kan konstruere en lysstråle ud fra et bestemt punkt på billedplanen, hvor lysstrålen er beskrevet ved en retningsvektor og et startpunkt.

\begin{figure}[H]
  \label{fig:raytracing_skitse}
  \centering
  \tdplotsetmaincoords{60}{130}
  \begin{tikzpicture}[tdplot_main_coords]
  \path[fill=blue!50, draw=gray!20] (2,-8,2) -- (-2,-8,2) -- (0,-8,0) -- (2,-8,2);
\draw (0,0,0) -- (0,-8,1);
\path[fill=gray!20, draw=gray!40] (-2,-4,-2) -- (-2,-4,2) -- (2,-4,2) -- (2,-4,-2) -- (-2,-4,-2);
\draw (0,0,0) -- (1,-4,1);

\draw plot [mark=*, mark size=2] coordinates{(1,-4,1) }; 
\draw plot [mark=*, mark size=2] coordinates{(0,0,0) }; 
\node [above right] at (1,-4,1) {$B$};
\node [above right] at (0,0,0) {$C$};
\draw [blue!50, thick, -{Stealth[width=3mm, length=3mm]}] (0,0,0) -- (1,-4,1);
\node [above right] at (0.5,-2,0.5) {$\vv{r}$};
\end{tikzpicture}
  \caption{Viser hvordan en der kan opstilles retningsvektor mellem kameraets position $C$ og punktet $P$ på billedplanen, som sammen med startpunktet $C$ beskriver lysstrålen i omvendt retning.}
\end{figure}

[TEGN TO PUNKTER MERE PÅ FIGUREN OVENOVER OG TAG UDGANGS PUNKT I AFBILDNING AF EN TREKANT I STEDET FOR ET PUNKT HVOR DEN SKÆRES I MIDTEN]

Retningsvektoren $\vv{r}$ for lysstrålen kan heraf beskrives som følgende.

$$ \vv{r} = \vv{B} - \vv{C} $$

Hvor $\vv{B}$ og $\vv{C}$ er stedvektorer for hhv. punktet på billedplanen $B$ og kameraets position $C$.

Lysstrålen kan på den måde beskrives ved følgende vektorfunktion.

$$ \vv{l}(t) = \vv{r} t + \vv{C}$$

Hvor $t$ er en skalar i $\mathbb{R}$.

For at finde ud af hvilken farve punktet på billedplanen $B$ skal have, ser man hvordan lysstrålen rammer de forskellige objekter der skal renderes.

Der findes flere forskellige modeller for hvordan lysintensiteten for en lysstråle beregnes. En simpel model, er Phong-modellen, som opdeler lys i forskellige kategorier: ambient, diffuse og specular.

\paragraph{Phong-modellen}
Phong-modellen er en såkaldt \textit{shading} funktion, som beskriver lyset fra punkter på et objekt på baggrund af lyskilden, objektet og kameraets synsvinkel\cite{phong_paper}. Der findes flere forskellige variationer af phong-modellen. Da vi som vist i afsnit \ref{sec:temptilrgb} kan arbejde med farvetemperaturer via rgb-værdier, så har vi valgt en variation af phong-modellen, som beskriver lys via rgb-værdier. 

Modellen bygger på følgende \textit{shading} funktion\cite{stanford_phong}.
\begin{equation} \label{eq:phong}
  \rho = \rho_a + \sum\limits_{lights} (\rho_d + \rho_s)
\end{equation}
Hvor $\rho_a$, $\rho_l$ og $\rho_s$ er hhv. \textit{ambient}, \textit{diffuse} og \textit{specular} lys beregnet som følgende\cite{stanford_phong}.
\begin{align}
	\rho_a &= m_a CA  \\
	\rho_l &= m_l CI max(\vv{I}\bullet\vv{n}, 0) \\
	\rho_s &= m_s S I max(-\vv{r}\bullet\vv{u},0)^{m_{sp}} \\
	S &= m_{sm} C + (1 - m_sm)(1,1,1)
\end{align}








På figur \ref{fig:phong_skitse} er 

\begin{figure}[H]
  \label{fig:phong_skitse}
  \centering
  \tdplotsetmaincoords{60}{130}
  \begin{tikzpicture}[tdplot_main_coords]
\draw (0,0,0) -> (2,-8,2);
\path[fill=gray!20, draw=gray!40] (-2,-4,-2) -- (-2,-4,2) -- (2,-4,2) -- (2,-4,-2) -- (-2,-4,-2);
\draw (0,0,0) -- (0,-4,0.5);
\path[fill=blue!50, draw=gray!20] (1,-4,1) -- (-1,-4,1) -- (0,-4,0) -- (1,-4,1);
\draw plot [mark=*, mark size=2] coordinates{(1,-4,1) }; 
\draw plot [mark=*, mark size=2] coordinates{(0,0,0) }; 
\node [above right] at (1,-4,1) {$B$};
\node [above right] at (0,0,0) {$C$};

\draw [blue!50, thick, -{Stealth[width=3mm, length=3mm]}] (0,0,0) -- (0,-4,0.5);
\node [above right] at (0.5,-2,0.5) {$\vv{r}$};
\end{tikzpicture}
  \caption{Viser de forskellige vektorer der indgår i beregning af lysintensiteten fra et punkt $p$ på et objekt.}
\end{figure}




%\begin{enumerate}

%  \item Omgivende lys (eng. \textit{ambient light}): En konstant belysning, som bliver tilføjet til alle objekter i scenen. Formålet med dette, er at undgå at skyggerne bliver helt sorte. Da der i et normalt rum altid vil være en lille lysmængde ved alle overflader. 
%  \item Spredt lys (eng. \textit{diffuse light}): Lyset, som rammer en overflade, og bliver spredt i alle retninger. Intensiteten af dette lys afhænger af vinklen mellem lysvektoren og overfladenormalen.
%  \item Spejlende lys (eng. \textit{specular light}): Lyset, som bliver reflekteret i en bestemt retning. Grundet dette, er det spejlende lys afhænig af vinklen mellem kameravektoren og lysvektoren. Det spejlende lys hjælper med at få et objekt til at have højglanspunkter.

%\end{enumerate}

%http://www.gameprogrammer.net/delphi3dArchive/phongfordummies.htm 



%\begin{equation}
%  S = (m_{sm})C + (1-m_{sm})(1,1,1)
%\end{equation}










\subsubsection{Skæring mellem linje og trekant i rummet}
\label{sec:triangle_intersection}
For at finde ud af om der er en skæring mellem en ray og en trekant, findes en metode, først at finde strålens skæring med trekantens plan, og derefter at bestemme om punktet er indenfor trekantens tre sider.

Hvis rayen er parallel med planen, skærer det enten ikke planen noget sted, ellers ligger alle punkter i planet. En ray er parallel med en plan, hvis rayens retning er 90 grader relativt til planets normalvektor, dette er sandt hvis ligning \ref{eq:parralel} er opfyldt.\cite{ray_triangle_intersection}

\begin{equation}
  \label{eq:parralel}
  \vv{r} \bullet \vv{n} = 0
\end{equation}

Planets normalvektor findes ved at krydse to vektorer fra en af trekantens hjørner til de to andre (se ligning \ref{eq:triangle_normal}).

\begin{equation}
  \label{eq:triangle_normal}
  \vv{n} = (B - A) \times (C - A)
\end{equation}

\begin{figure}[H]
  \centering
  \tdplotsetmaincoords{60}{130}
  \begin{tikzpicture}[tdplot_main_coords]
    % dashed line through plane
    \draw [black, thick] (0,0,0) -- (4,-8,4);
    % triangle and plane
    \path[fill=gray!10, draw=gray!20] (0,-4,0) -- (4,-4,0) -- (4,-4,4) -- (0,-4,4) -- (0,-4,0);
    \path[fill=gray!30, draw=gray!60] (1,-4,2) -- (2,-4,3) -- (3,-4,1) -- (1,-4,2);
    % triangle
    \draw [blue!50, thick, -{Stealth[width=3mm, length=3mm]}] (2,-4,3) -- (3,-4,1);
    \draw [blue!50, thick, -{Stealth[width=3mm, length=3mm]}] (2,-4,3) -- (1,-4,2);
    \draw [blue!50, thick, -{Stealth[width=3mm, length=3mm]}] (2,-4,3) -- (2,-2,3);
    % \draw [black,   thick, -{Stealth[width=2mm, length=2mm]}] (2,-4,3) -- (2,-4,2);
    \node [above] at (2,-3,3) {$\vv{n}$};
    \node [above] at (2,-4,3) {$A$};
    \draw plot [mark=*, mark size=1] coordinates{(2,-4,3) }; 
    \node [left] at (3,-4,1) {$B$};
    \draw plot [mark=*, mark size=1] coordinates{(3,-4,1) }; 
    \node [below] at (1,-4,2) {$C$};
    \draw plot [mark=*, mark size=1] coordinates{(1,-4,2) };
    
    % ray
    \node [above right] at (0,0,0) {$P_0$};
    \draw plot [mark=*, mark size=1] coordinates{(0,0,0) };
    \node [above right] at (0.25,-0.5,0.25) {$\vv{r}$};
    \draw [black, thick] (0,0,0) -- (2,-4,2);
    \draw [blue!50, thick, |-|] (0.1,0,-0.1) -- (2.1,-4,1.9);
    \node [below left] at (1,-2,1) {$t_A$};
    \node [black, above left] at (2.5,-4,2) {$P(t_A)$};
    \draw [black, thick, -{Stealth[width=3mm, length=3mm]}] (0,0,0) -- (0.5,-1,0.5);
    \draw plot [mark=*, mark size=1] coordinates{(2,-4,2) };
  \end{tikzpicture}
  \caption{Viser princippet bag perspektiv projektion af et punkt på et billedplan.}
  \label{fig:perspektiv_projektion}
\end{figure}

Hvis strålen ikke er parallel med planet kan der nu findes et punkt i planet, som strålen skærer. Alle vektorer i planet er ortogonale på normalvektoren, dermed kan der opstilles en ligning \ref{eq:vektor_in_plane} til at beskrive betingelsen for et punkt strålen som også ligger i planet. Ved at substituere linjens ligning (ligning \ref{eq:point_on_ray}) ind og isolere $t_A$ kan vi bruge til at finde punktet på strålens linje og er $t_A > 0$ så er punktet også foran strålens udgangspunkt $P_0$ \cite{ray_triangle_intersection}

\begin{align}
  \label{eq:point_on_ray}
  P(t) &= \vv{r} \cdotp t + \vv{P_0} \\
  \label{eq:vektor_in_plane}
  ( P(t_A) - A ) \bullet \vv{n} &= 0 \\
  \label{eq:test3}
  (\vv{r} \cdotp t_A + \vv{P_0} - A) \bullet \vv{n} &= 0 \\
  \label{eq:test4}
  t_A \cdotp \vv{r} \bullet \vv{n} + (\vv{P_0} - A) \bullet \vv{n} &= 0 \\
  \label{eq:test5}
  t_A &= -\frac{(\vv{A} - \vv{P_0})\bullet \vv{n}}{\vv{r} \bullet \vv{n}}
\end{align}

\begin{figure}[H]
  \centering
  \tdplotsetmaincoords{60}{130}
  \begin{tikzpicture}[tdplot_main_coords,thick,scale=2]
    \coordinate (P) at (2,-4,2);
    \coordinate (P') at (3,-4,3);
    \coordinate (A) at (2,-4,3);
    \coordinate (B) at (3,-4,1);
    \coordinate (C) at (1,-4,2);
    % dashed line through plane
    % triangle and plane
    \path[fill=gray!10, draw=gray!20] (0.5,-4,0.5) -- (3.5,-4,0.5) -- (3.5,-4,3.5) -- (0.5,-4,3.5) -- (0.5,-4,0.5);
    \path[fill=red!30, draw=red!20] (3.25,-4,0.5) -- (3.5,-4,0.5) -- (3.5,-4,3.5) -- (1.75,-4,3.5) -- (3.25,-4,0.5);
    \path[fill=gray!30, draw=gray!60] (A) -- (B) -- (C) -- (A);
    % triangle
    \draw [blue!50, thick, -{Stealth[width=3mm, length=3mm]}] (C) -- (1,-3,2);
    \draw [blue!50, thick, -{Stealth[width=3mm, length=3mm]}] (A) -- (B);
    \draw [blue!50, thick, -{Stealth[width=2mm, length=3mm]}] (A) -- (P);
    \draw [blue!50, thick, -{Stealth[width=3mm, length=3mm]}] (A) -- (2,-3,3);
    \draw [blue!50, thick, -{Stealth[width=2mm, length=3mm]}] (A) -- (P');
    \draw [blue!50, thick, -{Stealth[width=3mm, length=3mm]}] (A) -- (2,-5,3);
    \node [above] at (A) {$A$};
    \draw plot [mark=*, mark size=1] coordinates{(A) }; 
    \node [left] at (B) {$B$};
    \draw plot [mark=*, mark size=1] coordinates{(B) }; 
    \node [below] at (C) {$C$};
    \draw plot [mark=*, mark size=1] coordinates{(C) };
    \node [right] at (1,-3,2) {$\vv{AB \times AC}$};
    \node [right] at (2,-3,3) {$\vv{AB \times AP}$};
    \node [left] at (2,-5,3) {$\vv{AB \times AP'}$};
    
    % ray
    \node [below left] at (P) {$P$};
    \draw plot [mark=*, mark size=1] coordinates{(P) };
    \node [below left] at (P') {$P'$};
    \draw plot [mark=*, mark size=1] coordinates{(P') };
  \end{tikzpicture}
  \caption{Viser krydsproduktet $\protect\vv{n_a}$ af en af trekantens sider $\protect\vv{a}$ og $\protect\vv{AP}$.}
  \label{fig:perspektiv_projektion}
\end{figure}

For at et punkt i trekantens plan også er indenfor trekanten skal det ligge på den rigtige side af hver af trekantens sider $\{\vv{AB}, \vv{BC}, \vv{CA}\}$. Krydsproduktet af en side og en vektor til punktet i planen fra en af trekantens hjørner $\vv{AB} \times \vv{AP}$ vil være parralel med trekantens normal vektor (enten i samme retning eller direkte modsat). Afhængigt af hvilken side af $\vv{AB}$ punktet $P$ befinder sig bestemmer retningen af krydsproduktsvektoren, Så derved kan vi vælge et andet punkt på den ønskede side af $\vv{AB}$ at sammenligne med. Derved skal retningen af $\vv{AB} \times \vv{AP}$ være lig $\vv{AB} \times \vv{AC}$. Samme metode anvendt på de to andre sider af trekanten ($\vv{BC}$, $\vv{CA}$) og hvis punktet $P$ er på 'indersiden' af dem alle, da er $P$ i trekanten.


\subsubsection{Kd-træer}
\label{sec:kdtree}

Kd-træer er binære træer, hvilket betyder at hver node maksimalt har to under-noder. K'et i kd-træ er antallet af dimensioner, men i stedet for 3-d træ, siger man et 3-dimensionelt kd-træ. Kd-træer bruges til at minimere antallet af søgninger, som et givent program skal gennemføre, hvilket får det til at være mindre ressourcekrævende og hurtigere. 

Et kd-træ opbygges ved at have en mængde af værdier, som er usorterede. Disse værdier kan inddeles i et træ, ved at finde et taktisk splittepunkt, som er roden. Herefter går to noder ud fra roden, den venstre har den nedre halvdel af værdierne og den højre har den øvre halvdel. Herefter splittes disse to noder hver for sig igen, og sådan fortsætter træet indtil visse krav nås, hvorefter man så kalder den sidste række af noder for bladene. 

En simpel forklaring af et to-dimensionelt kd-træ kan være et program til at finde tallet 100 i en tilfældig mængde fra 1-1000. Roden deles på et smart udtænkt punkt, hvilket i dette tilfælde er midtpå ved 500, og derfor vil den venstre under-node til 'rod-noden' indeholde værdier, der er lig med eller under 500 og det højre barn til indeholde værdier, der er højere end eller lig med 500. Sådan fortsætter programmet med at halvere alle muligheder efter hvert niveau indtil den ønskede værdi er fundet. Denne metode er mindre ressourcekrævende og hurtigere end hvis man bad et program om at finde et givent tal mellem 1 og 1000, da den, i værste tilfælde, skulle tjekke 999 tal igennem, før den fandt det ønskede nummer.

\subsubsection*{Opsummering}

Ud fra teorien kan der nu kortfattets hvordan teorierne bidrager til programudviklingen. 

Rotationsmatricer giver den viden som skal til for at roterer vektorer i rummet med en vilkårlig vinkel. Dette er relevant når man skal flytte kameraets position, som hører til programkravet om at lampens belysning skal kunne visualiseres fra flere forskellige vinkler.

Teorien om konvertering fra farvetemperatur til RGB giver os den algoritme som skal til for, at lave en farvetemperatur i kelvin om til en RGB-værdi. Dette er relevant i forhold til programkravet om at kunne ændre pærens farvetemperatur. 

Teorien fra 3D-model til billede omhandler matematiske modeller og beskrivelser af relevante teknologier og algoritmer. Teorien i dette er relevant for, at forstå nogle af grundprincipperne i en raytracer heriblandt backwards raytracing og Phong-modellen.

Teorien om ray-triangle intersection omhandler en matematisk model over, hvordan man finder skæringen mellem en ray og en trekant i planen. Dette knytter sig til teorien om en 3D-model, da alle objekter består af trekanter. 

Som afslutning på teoridiskussionen opstilles følgende nye krav til programudviklingen:
\begin{enumerate}
    \item Programmet skal kunne modtage følgende input fra brugeren: 3D-fil af lampen og dens kontekst, billedets opløsning, pærens farvetemperatur og hvilken synsvinkel man ønsker at se billedet fra (kameraets position).
    \item Programmet skal ved brug af raytracing kunne fremstille et billede af lampen og dens belysning på baggrund af brugerinputtet. Derudover skal raytracerens renderingstid optimeres ved brug af KD-træer.
    \item Programmet skal kunne gemme det renderede billede.
\end{enumerate}