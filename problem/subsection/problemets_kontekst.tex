\subsection{Problemets kontekst}
\lipsum[1-2]


\subsubsection{Detailhandel}
En fysisk butik er et sted hvor kunderne selv skal komme hen, når de vil købe eller kigge på butikkens varer. En fysisk butik kræver lokaler, og derfor er der oftest faste udgifter til eje eller leje af grunden samt andre faste udgifter som fx elektricitet. Hvis butikken er stor nok, skal der også være ansat personale, til at vedligeholde den, og til at tage sig af de besøgende kunder og deres mulige spørgsmål til butikkens varer. En fysisk butik er altså dyr i omkostning, men kunderne foretrækker de fysiske butikker, da man kan se vareren selv og få direkte assistance om varen gennem en medarbejder. Det kræver dog stadig at kunden selv skal ud til butikken, og være sikker på dens åbningstider og varelager. Derudover kan det være en dyr sag, at ændre i sit brand, som fx da Superbest og Eurospar blev til Meny.
I vores kontekst snakker vi om en hvilken som helst fysisk butik, der har med lampesalg at gøre. Den lampeinteresserede kunde, kommer ud i butikken, og leder fx efter en ny væglampe til stuen. Problemet heri kan opstå, ved at der er adskillige forskellige lamper at vælge imellem, men ikke alle lamperne er tilsluttet, så man kan se hvordan lyset falder. Hvis kunden beslutter sig for en lampe, som ikke er tilsluttet, men regner med at den vil se godt ud på væggen i stuen, hvorefter det så viser sig, at lyset falder helt forkert og er alt for skarpt og blændende er det for sent, da lampen er pakket ud, og ledningen er blevet pillet ved. Lampen kan altså ikke byttes, og er ikke optimal i forhold til kundens stue.

\subsubsection{E-handel}
E-handel er elektronisk handel via internettet\cite{ddo_ehandel}. På internettet kan sælgere inden for e-handel have såkaldte e-butikker, hvor kunder kan købe varer\cite{ddo_ebutik}. E-butikker er ofte udformet således at kunden kan se billeder og informationer omkring sælgerens varer og derudfra kan kunden vælge at lægge varerne i en virtuel indkøbskurv, hvor kunden til sidst indtaster de nødvendige oplysninger for at købe og modtage varerne.


Blandt de mange forskellige varer, der sælges via e-butikker, er det her relevant at tale om e-handel med lamper. Nedenstående figur \ref{fig:e_handel_med_lamper} illustrerer princippet bag en lampesælgers salg af lampe til en kunde via en e-butik.
\begin{figure}[H]
	\centering
	\def\svgwidth{\columnwidth}
	\input{./graphics/e_handel_med_lampe.pdf_tex}
	\caption{Princippet bag handel af en lampe via en e-butik.}
    \label{fig:e_handel_med_lamper}
\end{figure}

På figur \ref{fig:e_handel_med_lamper} er det vist hvordan e-handlen starter med at kunden får et udvalg af lamper fra e-butikken. Kunden sender så en bestilling, som via e-butikken sendes videre til lampesælgeren, og til sidst sendes lampen til kunden. Dog ender handlen ikke nødvendigvis her, da kunden kan sende lampen retur såfremt at gældende lovgivning og købsbetingelser muliggører dette. For at undersøge lovgivningen nærmere kan man tage udgangspunkt i den danske lov om forbrugeraftaler\cite{retsinformationen}.

I lovens kapitel 1, § 1, stk. 2, nr. 1, fremgår der at lovens bestemmelser for fortrydelsesret gælder for aftaler, som er indgået ved fjernsalg. For en  fjernsalgsaftale gælder der, at aftalen om varer, er indgået gennem fjernkommunikation, hvor den erhvervsdrivende og forbrugeren ikke mødes fysisk (jf. kap. 1, § 3, nr. 1).

Ser man nu på loven i forbindelse med e-handel, foregår fjernkommunikationen gennem internettet via e-butikken, hvor fjernsalgsaftalen udføres i form af brugerens bestilling af f.eks. en lampe. Dette gør at fortrydelsesretten gælder ved e-handel.

Fortrydelsesretten er en forbrugers mulighed for at melde sig ud af en aftale, herunder køb af lamper ved e-handel. Hvis en en forbruger eksempelvis køber en lampe via en e-butik, har forbrugeren mulighed for at fortryde købet inden 14 dage ved at meddele dette til den erhvervsdrievende (jf. kap. 4, § 19). Herefter har forbrugeren 14 dage til at returnere varen (jf. kap. 4, § 24). Hvis varens værdi er forringet som følge af forbrugeren unødvendige håndtering af varen for at inspicere denne, så hæfter forbrugeren for denne værdiforringelse (jf. kap. 4, § 24, stk. 5). Dvs. at hvis en bruger installerer og bruger lampen, hvor der f.eks. tilpasses ledninger, så kan lampens værdi forringes og forbrugeren skal hæfte for dette. 







