Hvis vi vil roterer et punkt eller en vektor omkring nul-punktet i et koordinatsystem kan vi bruge en rotationsmatrix\cite{rotationsmatricer}.
En rotationsmatrix er en matrix der, hvis ganget sammen med en anden matrix, roterer en vektor eller et punkt i et koordinatsystem. 

\begin{align}\label{eu_eqn}
R_x(\theta) = 
\begin{bmatrix}
1 & 0 & 0\\ 
0 & cos \theta & - sin \theta\\ 
0 & sin \theta & cos \theta
\end{bmatrix}\\
R_y(\theta) = 
\begin{bmatrix}
cos \theta  & 0 & sin \theta\\ 
0           & 1 & 0\\ 
-sin \theta & 0 & cos \theta
\end{bmatrix}\\
R_z(\theta) = 
\begin{bmatrix}
cos \theta & - sin \theta & 0\\ 
sin \theta & cos \theta & 0\\
0 & 0 & 1
\end{bmatrix}
\end{align}

Hvis vi vil rotere en vektor/et punkt i 3D bliver det lidt mere komplekst. Vi vil så roterer 2 af punkterne i et plan omkring det sidste punkt, som dermed også virker som normalvektor til planet. Formlerne for at gøre dette er udviklet og frit tilgængelige på blandt andet Wikipedias side om rotationsmatricer:

Indsætter vi mængden af radianer vi vil dreje vores vektor og ganger dem sammen, ser vi at vektoren bliver drejet omkring nul-punktet med netop den mængde radianer.
[Indsæt eksempel]