\subsection{Relevans}
Rapporten vil i dette afsnit bestræbe sig efter at besvare om problemet er relevant og i så fald, hvorfor det er relevant. I denne sammenhæng ligges der blandt andet fokus på hvordan lyset påvirker mennesker, samt hvilke konsekvenser dårlig belysning kan medføre. 

\subsubsection{Lysets påvirkning på mennesket} 
\label{sec:konsekvenser}
Undersøgelser har vist, at lys har stor indvirkning på vores humør og trivsel\cite{videnskab_dk_paavirkning}.
Blåt/hvidt lys har den effekt, at vi føler os mere vågne og oppe på mærkerne, hvorimod rødt/gult lys har den modsatte effekt\cite{videnskab_dk_paavirkning}. Det giver god mening, når vi ser på det lys der er om dagen som ofte er lyst og blåligt pga. den blå himmel. Hvorimod lyset om aftenen og natten ofte er gult/rødt ved f.eks. solnedgang.

Vi bruger også lys til rigtig meget i hverdagen. Øjet, som virker ved hjælp af lys bruger vi til at navigere, se eventuelle farer og genkende venner.

Mange mennesker sidder i kontormiljøer i stort set al deres arbejdstid og som nævnt i indledningen, er de afhængige af, at lyset er godt. Hvis lyset ikke er godt, vil man ofte blive træt og uproduktiv. Derfor er det vigtigt at både loftslamper, skrivebordslamper og indretningen spiller godt sammen og skaber et godt lys-miljø.

\subsubsection{Når lampedesigns lysforhold ikke visualiseres}
En lampes funktion er, at afgive lys som kan bruges i forhold til brugerens ønsker. Hvis lampen ikke lever op til disse ønsker eller krav kan den blive tilset som værende dårlig og irreterende. Hvis lampen laver mærkelige skygger eller ujævn belysning vil den oftest være dårlig at læse ved, og øjnene skal derfor kompensere hvilket kan medføre at man får ondt i hovedet\cite{lys_konsekvenser}.

Én ting er lampeskærmen, men lyskilden kan give lige så mange problemer:

Lysstofrør giver ofte et flimrende lys som man i længden kan få ondt i hovedet af, men er tilgengæld billige i drift\cite{videnskab_dk_led}. Glødepærer giver et lys der er næsten tilsvarende dagslys, men har en kortere levetid hvilket medfører at den er dyrere på længere sigt, f.eks koster en glødepærer 10-15 kr og har en levetid på 1000 timer mens en LED koster 50-200 kr og har en levetid på 50000 timer. Ud fra et eksempel på bolius.dk kan man spare 954 kr om året ved at skifte glødepærer ud\cite{pris_glødepærer}. LED-pærer er en rimelig ny teknologi i lys-pære verdenen, men har potentiale, da de både kan farves, er billige i drift, billige i indkøb og holder meget længere end både glødepærer og lysstofrør\cite{videnskab_dk_led}.

\subsubsection{Hvorfor er det relevant?}
\label{sec:hvorfor_relavant}
For at svare på, hvorfor problemet er relevant, opstilles følgende antagelse: "Mennesker har svært ved at visualisere hvordan lys udbreder sig fra en lampe." Det er svært at bevise, at denne antagelse er korrekt for alle mennesker, men antagelsen er lavet efter en diskussion med Lars Peter Jensen, lektor på AAU, som netop havde erfaret, at han havde svært ved at visualisere hvordan lys fra en bestemt lampe ville se ud i hans hus, før han havde installeret lampen. Dette er en erfaring som flere i gruppen har gjort sig. Ud fra diskussionen og egne erfaringer har gruppen valgt at arbejde videre med antagelsen, da det må formodes at andre mennesker har haft lignende problem. Denne formodning ønskede gruppen at teste ved en spørgeskemaundersøgelse i IKEA, hvor tanken var at spørge de handlende om de kunne visualisere hvordan lys udbredte sig fra de lamper som de så i butikken. Gruppen henvendte sig derfor til IKEA i Aalborg for at høre om det var muligt at møde op i deres butik for at uddele spørgeskemaer, men gruppens henvendelse blev afvist, da IKEA mente, at spørgsmål som omhandler mangel på visualisering af lys fra lamper var spørgsmål som kunne sætte IKEA i dårlig lys, da kunderne ville relatere spørgsmålene til IKEAs produkter. 

På baggrund af antagelsen må vi formode, at fordi mennesker har svært ved at visualisere hvordan lys udbreder sig for en lampe, sker det at der købes lamper, som ikke lever op til de forventninger som forbrugeren havde da lampen blev købt. Lamper med dårlig belysning er lamper, der ikke lever op til de belysningsmæssige krav som forbrugeren stiller til at dække sine behov. Som nævnt i \ref{sec:konsekvenser} er konsekvenserne ved dårlig belysning blandt andet træthed og mindsket produktivitet.
Ud fra antagelsen kan det uddrages at problemet er relevant, da mangel på visualisering af lamper kan føre til fejlkøb som blandt andet kan medføre hovedpine, træthed og formindske produktiviteten af ens arbejde.

\subsubsection*{Opsummering}
I dette afsnit er der blevet argumenteret for hvilke konsekvenser dårlige lysforhold kan have på mennesker, samt ud fra en antagelse er der blevet bekræftet hvorfor problemet er relevant. 

Da problemet er relevant, kan rapporten nu beskæftige sig med at undersøge andre dele af problemfeltet. 
