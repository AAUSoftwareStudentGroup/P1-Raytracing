For at besvare det initierende problem, er det nødvendigt at researche, heraf opstilles der en række HV-spørgsmål, som vil danne grundlag for problemfeltet. Formålet med disse spørgsmål er, at komme i dybden med samt at forstå problemet.  
For at få afklaret hvor stort et problem det egentlig er, vil der blive uddelt spørgeskemaer, med formål at få en idé om omfanget af problemet. 
Spørgeskemaet vil være et redskab, som vi benytter for at få bekræftet vores antagelse om hvorvidt det er svært at visualisere lysets udbredelse fra en lampe. 

\subsection{Relevans}
Rapporten vil i dette afsnit bestræbe sig efter at besvare om problemet er relevant og i så fald, hvorfor det er relevant. I denne sammenhæng ligges der blandt andet fokus på hvordan lyset påvirker mennesker, samt hvilke konsekvenser dårlig belysning kan medføre. 

\subsubsection{Lysets påvirkning på mennesket}

Undersøgelser har vist at lys har stor invirkning på vores humør og trivsel.
Blåt/hvidt lys har den effekt at vi føler os mere vågne og oppe på mærkerne\cite{videnskab_dk_paavirkning}, hvorimod rødt/gult lys har den modsatte effekt. Vi bliver mere afslappede og dermed trætte. Det giver god mening når vi ser på det lys solen udsender om dagen(hvidtt/blå himmel), som gør os friske, og det lys vi ser når solen går ned(rød solnedgang/bål om natten) og det er tid til at gå til ro. Vi bruger også lys til rigtig meget i hverdagen. Øjet, som virker ved hjælp af lys bruger vi til at navigere, se evt. farer og genkende venner. Når vi bruger en computer er vi afhængige af at kunne se skærmen for at kunne bruge den. Mange mennesker sidder i kontormiljøer i stort set al deres arbejdstid og er afhængige af at lyset er godt. Hvis lyset ikke er godt vil man ofte blive træt og uproduktiv. Derfor er det vigtigt at både loftlamper, skrivebordslamper og indretningen spiller godt sammen og skaber et godt lys-miljø.


\subsubsection{Når lampedesigns lysforhold ikke visualiseres}

En lampes primære funktion er at afgive lys som kan bruges. Hvis lampen blænder nogen, er den dårlig og irriterende at bruge enten for en selv eller andre. Hvis lampen laver mærkelige skygger eller ujævnt lys er den dårlig at læse ved eller kigge på billeder og øjnene skal arbejde for at kompensere og man kan få ondt i hovedet. Hvis lampen afgiver et mærkeligt farvet lys er den også irriterende at bruge i længden. Lysstofrør giver ofte et dårligt flimrende lys som man i længden kan få ondt i hovedet af, men er tilgengæld billige i drift. Glødepærer giver et lys der er næsten tilsvarende dagslys, men er rigtig dyre i drift\cite{videnskab_dk_led}. LED-pærer er en rimelig ny teknologi i lys-pære verdenen, men har et kæmpe potentiale da de både kan farves, er billige i drift, billige i indkøb og holder meget længere end både glødepærer og lysstofrør. Et andet problem er designet af lygtepæle. Da LED'er er billige, giver et godt lys og kan tænde og slukke uden at bruge ekstra energi er de selvfølgelig et åbenlyst valg i lygtepæle. Men med flere resourcer til lys bliver lygtepælene stærkere og laver mere lys, som er en fordel for dem der er ude og gå eller køre, men er til stor ulempe for folk der prøver at sove eller gerne vil kigge på stjerner\cite{dr_dk_lysforurening}.


\subsubsection{Hvorfor er det så relevant?}

For at svare på hvorfor problemet er relevant, så opstilles følgende antagelse:
"Mennesker har svært ved at visualisere hvordan lys udbreder sig fra en lampe." Det er utroligt svært at argumentere for at denne antagelse er korrekt for alle mennesker, men antagelsen er lavet efter en diskussion med Lars Peter Jensen, professor på AAU, som netop havde erfaret, at han havde svært ved at visualisere hvordan lys fra en bestemt lampe ville se ud i hans hus, før han havde installeret lampen. Dette var en antagelse som flere i gruppen også havde erfaret. Ud fra diskussionen og egne erfaringer har gruppen altså valgt at arbejde videre med antagelsen, da det må formodes at andre mennesker har haft lignende problemet.
På baggrund af antagelsen må vi formode, at fordi mennesker har svært ved at visualisere hvordan lys udbreder sig for en lampe, så sker det at der købes lamper der har dårlig belysning. Lamper med dårlig belysning er lamper, der ikke lever op til de belysningsmæssige krav som forbrugeren stiller til at dække deres behov. Dette kunne f.eks være en kontorlampe der blænder, eller en lampe i køkkenet med svag belysning. Konsekvenserne ved dårlig belysning er som tidligere beskrevet blandt andet hovedpine, depression og formindsket arbejdsindsats. Ud fra antagelsen kan det altså uddrages at problemet er relevant, da mangel på visualisering i sidste ende kan medvirke til sundhedsmæssigekonsekvenser.