For at besvare det initierende problem, er det nødvendigt at researche. Heraf opstilles der en række hv-spørgsmål, som vil danne grundlag for problemfeltet. Formålet med disse spørgsmål er, at komme i dybden med samt at forstå problemet.

\subsection{Relevans}
Rapporten vil i dette afsnit bestræbe sig efter at besvare om problemet er relevant og i så fald, hvorfor det er relevant. I denne sammenhæng ligges der blandt andet fokus på hvordan lyset påvirker mennesker, samt hvilke konsekvenser dårlig belysning kan medføre. 

\subsubsection{Lysets påvirkning på mennesket}

Undersøgelser har vist, at lys har stor indvirkning på vores humør og trivsel.
Blåt/hvidt lys har den effekt, at vi føler os mere vågne og oppe på mærkerne, hvorimod rødt/gult lys har den modsatte effekt\cite{videnskab_dk_paavirkning}. Det giver god mening, når vi ser på det lys der er om dagen som ofte er lyst og blåligt pga. den blå himmel. Hvorimod lyset om aftenen og natten ofte er gult/rødt ved f.eks solnedgang.

Vi bruger også lys til rigtig meget i hverdagen. Øjet, som virker ved hjælp af lys bruger vi til at navigere, se eventuelle farer og genkende venner.

Mange mennesker sidder i kontormiljøer i stort set al deres arbejdstid og som nævnt i indledningen, er de afhængige af, at lyset er godt. Hvis lyset ikke er godt, vil man ofte blive træt og uproduktiv. Derfor er det vigtigt at både loftlamper, skrivebordslamper og indretningen spiller godt sammen og skaber et godt lys-miljø.


\subsubsection{Når lampedesigns lysforhold ikke visualiseres}

En lampes primære funktion er, at afgive lys som kan bruges. Hvis lampen blænder nogen, er den dårlig og irriterende at bruge enten for en selv eller andre. Hvis lampen laver mærkelige skygger eller ujævn belysning kan den være dårlig at læse ved, og øjnene skal arbejde for at kompensere, og man kan få ondt i hovedet\cite{lys_konsekvenser}.

Én ting er lampeskærmen, men lyskilden kan give lige så mange problemer:

Lysstofrør giver ofte et dårligt flimrende lys som man i længden kan få ondt i hovedet af, men er tilgengæld billige i drift\cite{videnskab_dk_led}. Glødepærer giver et lys der er næsten tilsvarende dagslys, men er rigtig dyre i drift\cite{videnskab_dk_led}. LED-pærer er en rimelig ny teknologi i lys-pære verdenen, men har et kæmpe potentiale, da de både kan farves, er billige i drift, billige i indkøb og holder meget længere end både glødepærer og lysstofrør\cite{videnskab_dk_led}.


\subsubsection{Hvorfor er det relevant?}
\label{sec:hvorfor_relavant}

For at svare på, hvorfor problemet er relevant, opstilles følgende antagelse: "Mennesker har svært ved at visualisere hvordan lys udbreder sig fra en lampe." Det er utroligt svært at argumentere for, at denne antagelse er korrekt for alle mennesker, men antagelsen er lavet efter en diskussion med Lars Peter Jensen, professor på AAU, som netop havde erfaret, at han havde svært ved at visualisere hvordan lys fra en bestemt lampe ville se ud i hans hus, før han havde installeret lampen. Dette var en antagelse som flere i gruppen også havde erfaret. Ud fra diskussionen og egne erfaringer har gruppen valgt at arbejde videre med antagelsen, da det må formodes at andre mennesker har haft lignende problem. 

På baggrund af antagelsen må vi formode, at fordi mennesker har svært ved at visualisere hvordan lys udbreder sig for en lampe, sker det at der købes lamper, der har dårlig belysning. Lamper med dårlig belysning er lamper, der ikke lever op til de belysningsmæssige krav som forbrugeren stiller til at dække deres behov. Dette kunne f.eks. være en kontorlampe, der blænder, eller en lampe i køkkenet med svag belysning. Dette er et problem, da undersøgelser har vist, at dårlig belysning kan få sundhedsskadelige konsekvenser \cite{lys_konsekvenser}. Ud fra antagelsen kan det  uddrages at problemet er relevant, da mangel på visualisering i sidste ende kan medføre sundhedsmæssige konsekvenser.

% Skjuler opsumering fra indholdsfortegnelsen
% \addtocontents{toc}{\protect\setcounter{tocdepth}{2}}
\subsubsection*{Opsummering}
% \addtocontents{toc}{\protect\setcounter{tocdepth}{3}}
I dette afsnit er der blevet argumenteret for hvilke konsekvenser dårlige lysforhold kan have på mennesker, samt ud fra en antagelse er der blevet bekræftet hvorfor problemet er relevant. 

Da problemet er relevant, kan rapporten nu beskæftige sig med at undersøge andre dele af problemfeltet. 
