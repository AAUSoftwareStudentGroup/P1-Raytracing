\subsection{Teknologier til visualisering}
\label{sec:tek_til_visualisering}
Vi ser en tydelig mulighed for at assistere forbrugere med at træffe et valg, når det kommer til køb af varer på nettet. For at undersøge hvilke teknologier der kan anvendes til at udvikle løsningsforslaget beskrevet i afsnit \ref{sec:losning}. Formålet med afsnittet er, at få en forståelse af hvilke teknologier der allerede eksisterer inden for visualisering, og finde ud af hvilken teknologi, der er bedst i forhold til visualisering af lamper for kunder der besøger en e-butiks hjemmeside. De enkelte teknologier vurderes på baggrund af de krav der er sat til løsningsforslaget i afsnit \ref{sec:losning}.

\subsubsection{Digitale billeder taget med et fysisk kamera}
Som beskrevet under afsnit \ref{sec:ehandel}, benytter e-butikker, sig ofte af billeder til at vise kunden deres varer over internettet. Et eksempel på dette er vist på figur \ref{fig:e_handel_lampebilleder}.

\begin{figure}[H]
    \centering
    \fbox{\rule{\textwidth}{5cm}}
    \caption{Billeder af lamper på e-butikken somelampstore.what}
    \label{fig:e_handel_lampebilleder}
\end{figure} 

I det viste tilfælde er visualiseringen skabt ved at tage billeder af lamperne med et kamera fra en bestemt vinkel, i en kontekst, der typisk hænger sammen med lampetypen. 

Fordelen ved denne type af visualisering er, at den giver et virkelighedstro billede af, hvordan lampen ser ud i den kontekst, som billedet er taget i. Ulempen er, at der ofte kun er et begrænset antal billeder til rådighed, hvilket kan medføre, at forbrugeren ikke kan se lampen fra alle vinkler og på den måde ikke kan visualisere lampen for sig. Derudover kan det være svært, at se hvordan lyset udbreder sig fra lampen, da dette til dels afhænger af hvilken vinkel man ser lampen fra. 

Herudfra kan man kortfattet sige, at visualisering af lamper gennem billeder, taget med et fysisk kamera, giver et realistisk billede af lampen, men kun i den kontekst og vinkel billedet er taget i. 


\subsubsection{Computergrafik}
\label{sec:computergrafik}
I computergrafik, er en 3D model, en beskrivelse af objekters form og materiale \cite{computergrafik_introduktion}. Computergrafiske metoder kan bruges til at simulere, hvordan lys interagere med modellen og på den måde tegne et billede af modellen. Der eksisterer forskellige computergrafiske metoder, flere af hvilke kan bruges sammen med andre for, at opnå et mere realistisk eller effektivt resultat. Der er allerede værktøjer, som kan visualisere produkter til salg på websites som f.eks.\ Cylindo \cite{Cylindo}. Vi har ikke kendskab til at andre specialiserer sig, eller markedsfører sig på nuværende tidspunkt med deres kompetencer med fokus på visualisering af lampers belysning. Derfor er der herunder beskrivelse af to af de mest anvende metoder inden for computergrafik.

\paragraph{Rasterisering}
er en metode til at visualisere miljøer med høj aktiv brugerinteraktion som f.eks.\ computerspil \cite{rastarization}. Metoden virker ved rent matematisk at projektere modellen på et billedplan som repræsentere skærmen \cite{rastarization}. Fordelen ved rasterisering er at disse projektioner, kan foretages meget hurtigt af computerens grafikkort, som er bygget specielt til formålet \cite{rastarization}. Dette kan dog mindske fleksibiliteten, og muligheden for mere avancerede visualisering, hvor der kræves refleksioner og refraktioner af lys, som ikke passer ind i den proces (graphics pipeline \cite{rastarization}, som de enkelte grafikkort danner billeder ud fra). 

\paragraph{Ray tracing} [DER MANGLER KILDER TIL PÅSTANDE I DETTE AFSNIT]() forsøger, nøjagtigt at simulere lys i et virtuelt miljø, i modsætning til rasterisering hvor hastighed er den primære faktor. Raytracing bygger fundamentalt på at følge lysstråler og bygge en model for hvordan lysstrålerne interagere med forskellige objekter og materialer \cite{raytracing_for_begyndere}. 

I forhold til rasterisering, tager det længere tid at tegne, men komplekse lysfænomener som refleksioner og lys forvrængninger igennem semitransparante medier som vand(kaldet refraktion) er simple at beskrive for en raytracing algoritme, som kan tegne disse med realistisk precision. Nogle fænomener som bløde skygger kan også beskrives men jo flere typer fænomener og jo større realisme der kræves des længere tid tager det at tegne et billede, men raytracing tillader stor fleksibilitet.

\subsubsection{Augmented Reality}
Der er nu blevet gennemgået visualisering af virkelige objekter ved hjælp af digitale billeder og virtuelle objekter ved hjælp af computergrafik. I dette afsnit beskrives augmented reality som er en teknologi der kombinerer virkelige og virtuelle objekter \cite{augmented_reality}.

Augmented reality fungerer ved at tage et billede med et normalt kamera, og herefter ændrer billedet ved at indsætte computergrafik på billedet \cite{augmented_reality}.

Et eksempel på anvendelsen af augmented reality er Artemides Augmented Reality App. Denne app gør det muligt, at visualisere udvalgte lamper i en kontekst som brugeren selv vælger \cite{artemides}. 

Fordelen ved augmented reality i forhold til løsningsforslaget er, at den muliggøre at se lampen fra flere forskellige vinkler i den kontekst som kunden ønsker. Hvis der ses bort fra tekniske udfordringer så ville det også være en fordel hvis kunden kunne visualisere en lampe og dens belysning i den kontekst som kunden ønsker at købe lampen til. \newline Ulempen ved augmented reality i forhold til løsningsforslaget er, at det er en teknisk udfordring, at få en rummelig forståelse for den virkelig kontekst så det virtuelle der indsættes får en tilstrækkelig realisme. Det vil derfor være svært at simulere lampens belysning hvis man ikke kender til dimensioner eller materialerne i den virkelige kontekst. Hvis det lykkedes, at få en forståelse for de virkelige objekter i billedet, så vil det stadigvæk være raytracing eller en anden teknik indenfor computergrafik som ville være at foretrække når man skal visualisere lampens belysning. Derfor er udfordringen ved augmented reality dobbelt, da den både skal få en rummelig forståelse for de virkelige objekter i billedet, samt lave en realistisk computergrafik som passer ind i billedet.



\subsubsection*{Opsummering}
Ud fra ovenstående afsnit er der udledt følgende fordele og ulemper for de enkelte teknologier.
\begin{table}[H]
  \centering
  
\center
    \begin{tabular}{ | p{3cm} | p{5cm} | p{5cm} |}
    
    \hline
    Teknologi & Fordele & Ulemper \\ \hline
    Kamerabilleder & Realistisk visualisering. & Tidskrævende. Begrænset kombinationer af lampen, synsvinkel, farvetemperatur og kontekst. \\ \hline
   Computergrafik & Nemt at integrere på lampebtuikker hjemmesider. Kan opnå høj realisme af lampen og dens belysning. \newline Nemt at ændre vinkel, farvetemperatur og kontekst hvorfra lampen visualiseres. & Kræver 3D-model af lampen og konteksten. Kræver meget computerkraft ved høj realisme. \\ \hline
   Augmented Reality & Lampen kan visualiseres fra flere vinkler og i den kontekst som kunden ønsker at købe lampen til & Det er en teknisk udfordring at virtuelle objekter med den virkelige kontekst, så der stadigvæk opnås en hvis realisme. \\ \hline
    \end{tabular}
  \caption{Viser fordele og ulemper ved de tre teknologier til visualisering.}
\label{tab:fordele_ulemper_teknologier}
\end{table}

På baggrund af fordele og ulemper vist i tabel \ref{tab:fordele_ulemper_teknologier} sammenlignes de tre teknologier nu med de krav, der er opstillet til løsningsforslaget i afsnit \ref{sec:losning}. 

Billeder af fysiske lamper taget med et kamera, kan bidrage til udviklingen af løsningsforslaget, da den delvist opfylder krav 1-3 og 5. Da det er muligt at tage et billede af lampens og dens belysning med en bestemt farvetemperatur og fra en bestemt vinkel. Derudover kan løsningen implementeres på en e-butiks hjemmeside, ved blot at bruge de billeder der tages med kammeraet. Ulempen er at det kan være ressourcekrævende, da der skal tages billeder med forskellige lamper, farvetemperaturer og vinkler.

Hvordan computergrafik kan biddrage med at løse kravene i afsnit \ref{sec:losning}, afhænger af hvilken teknik man benytter. Hvis man benytter rasterisering, vil det være svært at opfylde krav 1, da visualisering af lampens belysning kræver, at der anvendes lysfænomener, som ikke passer ind i modellen for rasterisering. Rasterisering vil dog være oplagt til at opfylde krav 2 og 4, da den som nævnt er bygget til at rendere billede med høj brugerinteraktion og kort renderingstid.
Raytracing, vil derimod være oplagt til at opfylde krav 1-3, da det er muligt at simulere lampen og dens belysning med forskellige farvetemperaturer og vinkler. Derudover kan raytracing implementeres i løsningen, der renderes billede, der kan vises på en e-butiks hjemmeside, hvilket gør at den opfylder krav 5. Ulempen ved raytracing, er at det kan være svært at opfylde krav 4, da simulering af lampens belysning kan være en tidskrævende proces.

Augmented reality kan bidrage bidrage til udviklingen af løsningsforslaget ved at den har potentialet for at opfylde de samme krav, som både visualisering vha. kamerabilleder og computergrafik (krav 1-3). Dog er dette en teknisk udfordring, da den rummelige forståelse af den virkelige kontekst af billedet, er nødvendig for at kunne opnå en realistisk visualisering, når computergrafikken indsættes. I forhold til krav 4 har augmented reality den samme udfordring som f.eks.\ raytracing, da simulering af lampens belysning kan være en tidskrævende proces. Derudover er det svært at implementere augmented reality på en e-butiks hjemmeside, da det kræver adgang til kundens kamera. 



\clearpage

