\subsection{Teknologier til visualisering}
Vi ser en tydelig mulighed for at assistere forbrugere med at træffe et valg, når det kommer til køb af varer på nettet. For at undersøge hvilke teknologier, der kan anvendes til visualisering, er der i dette afsnit en række teknologier og metoder, som alle er relevante i forhold til at visualisere en lampe. Teknologierne er udvalgt på baggrund af diskussion i gruppen, hvor mindre relevante teknologier, som f.eks. 3D-print blev fravalgt. Formålet med afsnittet er, at få en forståelse af hvilke teknologier der allerede eksisterer inden for visualisering, og finde ud af hvilke metoder, der er bedst i forhold til visualisering af lamper for brugere der handler via internettet. De enkelte teknologier vurderes på baggrund af teknologiens fordele og ulemper i forhold til at besvare den endelige problemformulering.

\subsubsection{Digitale billeder taget med et fysisk kamera}
Som beskrevet under afsnit \ref{sec:ehandel}, benytter e-butikker, sig ofte af billeder til at vise kunden deres varer over internettet. Et eksempel på dette er vist på figur \ref{fig:e_handel_lampebilleder}.

\begin{figure}[H]
    \centering
    \fbox{\rule{\textwidth}{5cm}}
    \caption{Billeder af lamper på e-butikken somelampstore.what}
    \label{fig:e_handel_lampebilleder}
\end{figure} 

I det viste tilfælde er visualiseringen skabt ved at tage billeder af lamperne med et kamera fra en bestemt vinkel, i en kontekst, der typisk hænger sammen med lampetypen. 

Fordelen ved denne type af visualisering er, at den giver et virkelighedstro billede af, hvordan lampen ser ud i den kontekst, som billedet er taget i. Ulempen er, at der ofte kun er et begrænset antal billeder til rådighed, hvilket kan medføre, at forbrugeren ikke kan se lampen fra alle vinkler og på den måde ikke kan visualisere lampen for sig. Derudover kan det være svært, at se hvordan lyset udbreder sig fra lampen, da dette til dels afhænger af hvilken vinkel man ser lampen fra. 

Herudfra kan man kortfattet sige, at visualisering af lamper gennem billeder, taget med et fysisk kamera, giver et realistisk billede af lampen, men kun i den kontekst og vinkel billedet er taget i. 


\subsubsection{Augmented Reality App}
Augmented Reality er en teknologi der kan sætte virtuelle objekter ind i en virkelig kontekst. Derudover har man mulighed for at interagere med objektet i real tid. 
Denne metode bruger producenten "Artemides" i deres Augmented Reality App. Appen virker ved, at man scanner et logo fra et fysisk lampekatalog, hvorefter en givet lampe vil vise sig på logoets plads. Det er herefter muligt at flytte kataloget for, at se lampen i forskellige kontekster. Derudover har man mulighed for at rotere i alle vinkler samt tænde og slukke for lampens pære. 
Fordelene ved appen er, at brugeren har mulighed for selv at vælge hvilken lampe de vil se i sin egen kontekst, samt interagere med lampen. Herved har brugeren mulighed for, at se præcist hvordan lampen ser ud. 
Ulemperne ved appen er, at den ikke visualisere lampens belysning særlig godt, da dens første priotet er at visualisere selve lampens design. Derudover er det ikke muligt at visualisere lamper i en kontekst, hvis man ikke har den nødvendige bog som indeholder logoer over de forskellige lampedesigns i 3D. En anden ulempe er, at lampeudvalget er meget begrænset da det kun er udvalgte produkter fra "Artemides" som kan visualiseres. 
Nedenstående figur viser "Artemides" brugervejledning til appen.

\begin{figure}[H]
    \centering
    \includegraphics[width=10cm]{augmented_reality_artemides}
    \caption{Brugervejledning fra Artemides augmented reality.}
    \label{fig:augmented_reality_artemides}
\end{figure} 


\subsubsection{Computergrafik}
\label{sec:computergrafik}
I computergrafik, er en 3D model, en beskrivelse af objekters form og materiale.\cite{computergrafik_introduktion} Computergrafiske metoder kan bruges til at immitere hvordan lys interaktere med modellen og på den måde tegne et billede af modellen. Der eksisterer en mængde forskellige computergrafiske metoder, flere af hvilke kan bruges sammen med andre for, at opnå et mere realistisk eller effektivt resultat. Der findes flere produkter som kan visualisere produkter til salg på websites som f.eks. Cylindo\cite{Cylindo}. Men vi har ikke kendskab til at andre specialisere sig, eller markedsføre sig på nuværende tidspunkt med deres kompetencer med fokus på visualisering af lampers belysning.

\paragraph{Rasterisering}
er en metode til at visualisere miljøer med høj aktiv brugerinteraktion som f.eks. computerspil.\cite{rastarization} Metoden virker ved rent mattematisk at projektere modellen på et billedplan som repræsentere skærmen.\cite{rastarization}. Fordelen ved resterisering er at disse projektioner, kan foretages meget hurtigt af computerens grafikkort, som bygget specielt til formålet\cite{rastarization}. Dette kan dog mindske fleksibiliteten, og muligheden for mere avancerede visualisering, hvor der kræves refleksioner og refraktioner af lys, som ikke passer ind i den proces (graphics pipeline\cite{rastarization}, som de enkelte grafikkort danner billeder ud fra. 

\paragraph{Ray tracing}\cite{raytracing_for_begyndere} forsøger, nøjagtigt at simulere lys i et virtuelt miljø, i modsætning til rasterisering hvor hastighed er den primære faktor. Raytracing bygger fundementalt på at følge stråler af lys og bygge en model for hvordan de stråler interaktere med forskællige objekter og materialer. Der skælnes mellem to typer af raytracing: Forwards raytracing og backwards raytracing.

\subparagraph{Forwards raytracing}\cite{radiosity_by_wpi,radiosity_by_uob} er oftere kaldet radiosity og vil bliver reffereret til som sådan fremhenværende. Radiosity er hvad man kunne kalde en forwards raytracing metode. Her eksistere lyskilder ikke som specielle objekter i en 3D model, hvilket er tilfældet for de andre metoder, men her som objekter uden forskel fra de andre i modellen.

I radiosity modellen er alle flader betegnet med en absorbans faktor og en energi faktor. Absorbansen beskriver hvor meget af lys der rammer fladen der bliver absorberet. Absorberet lys hæver en flades energi og som i virkeligheden, afgives noget af den energi som lys, mens andet bliver omdannet til f.eks. varme. Lyskilder er således blot flader som starter med en mængde energi.

Radiosity er i stor grad blandt de mest tidskrævende metoder eftersom at den laver beregninger som ikke nødvendigvis bliver set i et billede. Dette muliggøre dog at visualisere en scene en gang og derefter at kunne se den fra mange vinkler eftersom at de ekstra udregninger allerede er gjort. Radiosity er derimod ikke designet til at håndtere fænomener som er afhængig af hvor man ser et objekt fra, så som refleksion og refraktion. Dvs. at radiosity ikke kan håndtere metalliske overflader eller semitransparente materialer. Til gengæld er Radiosity rigtig god til at simulere matte overflader og skygger.

\subparagraph{Backwards raytracing} er hvad man i almindelighed kalder raytracing og vil bliver reffereret til som sådan fremhenværende. Raytracing forsøger at simplificerer den fysiske model af lys ved at ignorere det lys som ikke rammer vores øjne. Raytracing er dog alligevel blandt de metoder som kendes for at kunne skabe de mest fotorealistiske renderinger. Dette gøres ved såkaldt \textit{backwards raytracing}, hvorved man følger en stråle fra øjet og ud mod 3D modellen, så tjekkes der for kollisioner mellem strålen og objekterne i modellen. Ved hver kollision kan man vælge at følge yderligere stråler som kan hjælpe med at udregne refleksioner eller komplekse skygger. Denne metoder står i modsætning til hvad man kalder \textit{forwards raytracing} som er den mere fysisk korrekte metode, hvor man følger stråler af lys fra hver lyskilde.

I forhold til rasterisering, tager raytracede billeder væsentligt længere tid at tegne, men komplekse lysfænomener som refleksioner og lys forvrængninger igennem semitransparante medier som vand(kaldet refraktion) er simple at beskrive for en raytracing algoritme, som kan tegne disse med realistisk precision. Nogle fænomener som bløde skygger kan også beskrives men jo flere typer fænomener og jo større realisme der kræves des længere tid tager det at tegne et billede, men raytracing tillader stor fleksibilitet.


\subsubsection*{Opsummering}
Ud fra ovenstående afsnit er der udledt følgende fordele og ulemper for de enkelte teknologier.
\begin{table}[H]
  \centering
  
\center
    \begin{tabular}{ | p{3cm} | p{5cm} | p{5cm} |}
    
    \hline
    Teknologi & Fordele & Ulemper \\ \hline
    Kamerabilleder & Realistisk visualisering. & Tidskrævende. Begrænset kombinationer af lampen, synsvinkel, farvetemperatur og kontekst. \\ \hline
    Augmented Reality & Lampen er i konteksten & Manglende realisme af lampens belysning \\ \hline
   Computergrafik & Nemt at integrere på lampebtuikker hjemmesider. Kan opnå høj realisme af lampen og dens belysning. \newline Nemt at ændre vinkel, farvetemperatur og kontekst hvorfra lampen visualiseres. & Kræver 3D-model af lampen og konteksten. Kræver meget computerkraft ved høj realisme. \\ \hline
    \end{tabular}
  \caption{Viser fordele og ulemper ved de tre teknologier til visualisering.}
\label{tab:fordele_ulemper_teknologier}
\end{table}

På baggrund af fordele og ulemper vist i tabel \ref{tab:fordele_ulemper_teknologier} har gruppen valgt at arbejde videre med raytracing, som teknologi til visualisering. Dette skyldes at raytracing, til dels kan visualisere lamper og deres belysning, men også give muligheden for at skifte lampe,  farvetemperatur, synsvinkel og kontekst uden at hverken producenten, e-butikken eller brugeren skal udføre ekstra arbejde. På trods af at computergrafik kræver meget computerkraft, så er det den teknologi som er mest fordelagtig at bruge i forhold til problemformuleringen, da man ved brug af computergrafik kan opnå høj realisme samt integrere løsningen på lampebutikkers hjemmesider. Derudover kan der ved brug af matematiskemodeller implementeres features som ændring af vinkel, farvetemperatur og kontekst. 
