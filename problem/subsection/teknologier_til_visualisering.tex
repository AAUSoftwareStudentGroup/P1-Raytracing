\subsection{Teknologier til visualisering}
\label{sec:tek_til_visualisering}
Vi ser en tydelig mulighed for at assistere forbrugere med at træffe et valg, når det kommer til køb af varer på nettet. For at undersøge hvilke teknologier der kan anvendes til at udvikle løsningsforslaget beskrevet i afsnit \ref{sec:losning}. Formålet med afsnittet er, at få en forståelse af hvilke teknologier der allerede eksisterer inden for visualisering, og finde ud af hvilken teknologi, der er bedst i forhold til visualisering af lamper for kunder der besøger en e-butiks hjemmeside. De enkelte teknologier vurderes på baggrund af de krav der er sat til løsningsforslaget i afsnit \ref{sec:losning}.

\subsubsection{Digitale billeder taget med et fysisk kamera}
Som beskrevet under afsnit \ref{sec:ehandel}, benytter e-butikker, sig ofte af billeder til at vise kunden deres varer over internettet. Et eksempel på dette er vist på figur \ref{fig:e_handel_lampebilleder}.

\begin{figure}[H]
    \centering
    \fbox{\rule{\textwidth}{5cm}}
    \caption{Billeder af lamper på e-butikken somelampstore.what}
    \label{fig:e_handel_lampebilleder}
\end{figure} 

I det viste tilfælde er visualiseringen skabt ved at tage billeder af lamperne med et kamera fra en bestemt vinkel, i en kontekst, der typisk hænger sammen med lampetypen. 

Fordelen ved denne type af visualisering er, at den giver et virkelighedstro billede af, hvordan lampen ser ud i den kontekst, som billedet er taget i. Ulempen er, at der ofte kun er et begrænset antal billeder til rådighed, hvilket kan medføre, at forbrugeren ikke kan se lampen fra alle vinkler og på den måde ikke kan visualisere lampen for sig. Derudover kan det være svært, at se hvordan lyset udbreder sig fra lampen, da dette til dels afhænger af hvilken vinkel man ser lampen fra. 

Herudfra kan man kortfattet sige, at visualisering af lamper gennem billeder, taget med et fysisk kamera, giver et realistisk billede af lampen, men kun i den kontekst og vinkel billedet er taget i. 


\subsubsection{Computergrafik}
\label{sec:computergrafik}
I computergrafik, er en 3D model, en beskrivelse af objekters form og materiale.\cite{computergrafik_introduktion} Computergrafiske metoder kan bruges til at immitere hvordan lys interaktere med modellen og på den måde tegne et billede af modellen. Der eksisterer en mængde forskellige computergrafiske metoder, flere af hvilke kan bruges sammen med andre for, at opnå et mere realistisk eller effektivt resultat. Der findes flere produkter som kan visualisere produkter til salg på websites som f.eks. Cylindo\cite{Cylindo}. Men vi har ikke kendskab til at andre specialisere sig, eller markedsføre sig på nuværende tidspunkt med deres kompetencer med fokus på visualisering af lampers belysning.

\paragraph{Rasterisering}
er en metode til at visualisere miljøer med høj aktiv brugerinteraktion som f.eks. computerspil.\cite{rastarization} Metoden virker ved rent mattematisk at projektere modellen på et billedplan som repræsentere skærmen.\cite{rastarization}. Fordelen ved resterisering er at disse projektioner, kan foretages meget hurtigt af computerens grafikkort, som bygget specielt til formålet\cite{rastarization}. Dette kan dog mindske fleksibiliteten, og muligheden for mere avancerede visualisering, hvor der kræves refleksioner og refraktioner af lys, som ikke passer ind i den proces (graphics pipeline\cite{rastarization}, som de enkelte grafikkort danner billeder ud fra. 

\paragraph{Ray tracing}\cite{raytracing_for_begyndere} forsøger, nøjagtigt at simulere lys i et virtuelt miljø, i modsætning til rasterisering hvor hastighed er den primære faktor. Raytracing bygger fundementalt på at følge stråler af lys og bygge en model for hvordan de stråler interaktere med forskællige objekter og materialer. Der skælnes mellem to typer af raytracing: Forwards raytracing og backwards raytracing.

\subparagraph{Forwards raytracing}\cite{radiosity_by_wpi,radiosity_by_uob} er oftere kaldet radiosity og vil bliver reffereret til som sådan fremhenværende. Radiosity er hvad man kunne kalde en forwards raytracing metode. Her eksistere lyskilder ikke som specielle objekter i en 3D model, hvilket er tilfældet for de andre metoder, men her som objekter uden forskel fra de andre i modellen.

I radiosity modellen er alle flader betegnet med en absorbans faktor og en energi faktor. Absorbansen beskriver hvor meget af lys der rammer fladen der bliver absorberet. Absorberet lys hæver en flades energi og som i virkeligheden, afgives noget af den energi som lys, mens andet bliver omdannet til f.eks. varme. Lyskilder er således blot flader som starter med en mængde energi.

Radiosity er i stor grad blandt de mest tidskrævende metoder eftersom at den laver beregninger som ikke nødvendigvis bliver set i et billede. Dette muliggøre dog at visualisere en scene en gang og derefter at kunne se den fra mange vinkler eftersom at de ekstra udregninger allerede er gjort. Radiosity er derimod ikke designet til at håndtere fænomener som er afhængig af hvor man ser et objekt fra, så som refleksion og refraktion. Dvs. at radiosity ikke kan håndtere metalliske overflader eller semitransparente materialer. Til gengæld er Radiosity rigtig god til at simulere matte overflader og skygger.

\subparagraph{Backwards raytracing} er hvad man i almindelighed kalder raytracing og vil bliver reffereret til som sådan fremhenværende. Raytracing forsøger at simplificerer den fysiske model af lys ved at ignorere det lys som ikke rammer vores øjne. Raytracing er dog alligevel blandt de metoder som kendes for at kunne skabe de mest fotorealistiske renderinger. Dette gøres ved såkaldt \textit{backwards raytracing}, hvorved man følger en stråle fra øjet og ud mod 3D modellen, så tjekkes der for kollisioner mellem strålen og objekterne i modellen. Ved hver kollision kan man vælge at følge yderligere stråler som kan hjælpe med at udregne refleksioner eller komplekse skygger. Denne metoder står i modsætning til hvad man kalder \textit{forwards raytracing} som er den mere fysisk korrekte metode, hvor man følger stråler af lys fra hver lyskilde.

I forhold til rasterisering, tager raytracede billeder væsentligt længere tid at tegne, men komplekse lysfænomener som refleksioner og lys forvrængninger igennem semitransparante medier som vand(kaldet refraktion) er simple at beskrive for en raytracing algoritme, som kan tegne disse med realistisk precision. Nogle fænomener som bløde skygger kan også beskrives men jo flere typer fænomener og jo større realisme der kræves des længere tid tager det at tegne et billede, men raytracing tillader stor fleksibilitet.


\subsubsection{Augmented Reality}
Augmented Reality er en teknologi der kan sætte virtuelle objekter ind i en virkelig kontekst. Derudover har man mulighed for at interagere med objektet i real tid.
Augmented Reality kan direkte oversættes til "Fordrejet virkelighed". Augmented reality bruges i blandt andet jægerfly \cite{hud} hvor en "HUD" (Head-Up Display) bruges til at fortælle piloten informationer om højde, hastighed (se figur) men kan også bruges til at vise en landingsbane hvis det f.eks. er tåget og man ikke kan se landingsbanen. 
Vi ser også Augmented Reality i hverdagen hvor f.eks. redigerede billeder på Instagram er en form for augmented reality. I takt med at billig computerkraft er blevet hvermands-eje er Augmented Reality også på vej ind i spil-verdenen hvor en kedelig skov eller mark kan blive til en skydebane hvor man kan spille mod sine venner.

\begin{figure}[H]
    \centering
    \includegraphics[width=10cm]{hud.jpg}
    \caption{Heads-Up Display fra et "F/A-18 Hornet" jægerfly}
    \label{fig:hud}
\end{figure} 


\subsubsection*{Opsummering}
Ud fra ovenstående afsnit er der udledt følgende fordele og ulemper for de enkelte teknologier.
\begin{table}[H]
  \centering
  
\center
    \begin{tabular}{ | p{3cm} | p{5cm} | p{5cm} |}
    
    \hline
    Teknologi & Fordele & Ulemper \\ \hline
    Kamerabilleder & Realistisk visualisering. & Tidskrævende. Begrænset kombinationer af lampen, synsvinkel, farvetemperatur og kontekst. \\ \hline
   Computergrafik & Nemt at integrere på lampebtuikker hjemmesider. Kan opnå høj realisme af lampen og dens belysning. \newline Nemt at ændre vinkel, farvetemperatur og kontekst hvorfra lampen visualiseres. & Kræver 3D-model af lampen og konteksten. Kræver meget computerkraft ved høj realisme. \\ \hline
   Augmented Reality & Lampen kan visualiseres fra flere vinkler og i den kontekst som kunden ønsker at købe lampen til & Det er en teknisk udfordring at virtuelle objekter med den virkelige kontekst, så der stadigvæk opnås en hvis realisme. \\ \hline
    \end{tabular}
  \caption{Viser fordele og ulemper ved de tre teknologier til visualisering.}
\label{tab:fordele_ulemper_teknologier}
\end{table}

På baggrund af fordele og ulemper vist i tabel \ref{tab:fordele_ulemper_teknologier} sammenlignes de tre teknologier nu med de krav, der er opstillet til løsningsforslaget i afsnit \ref{sec:losning}. 

Billeder af fysiske lamper taget med et kamera, kan bidrage til udviklingen af løsningsforslaget, da den delvist opfylder krav 1-3 og 5. Da det er muligt at tage et billede af lampens og dens belysning med en bestemt farvetemperatur og fra en bestemt vinkel. Derudover kan løsningen implementeres på en e-butiks hjemmeside, ved blot at bruge de billeder der tages med kammeraet. Ulempen er at det kan være ressourcekrævende, da der skal tages billeder med forskellige lamper, farvetemperaturer og vinkler.

Hvordan computergrafik kan biddrage med at løse kravene i afsnit \ref{sec:losning}, afhænger af hvilken teknik man benytter. Hvis man benytter rasterisering, vil det være svært at opfylde krav 1, da visualisering af lampens belysning kræver, at der anvendes lysfænomener, som ikke passer ind i modellen for rasterisering. Rasterisering vil dog være oplagt til at opfylde krav 2 og 4, da den som nævnt er bygget til at rendere billede med høj brugerinteraktion og kort renderingstid.
Raytracing, vil derimod være oplagt til at opfylde krav 1-3, da det er muligt at simulere lampen og dens belysning med forskellige farvetemperaturer og vinkler. Derudover kan raytracing implementeres i løsningen, der renderes billede, der kan vises på en e-butiks hjemmeside, hvilket gør at den opfylder krav 5. Ulempen ved raytracing, er at det kan være svært at opfylde krav 4, da simulering af lampens belysning kan være en tidskrævende proces.

Augmented reality kan bidrage bidrage til udviklingen af løsningsforslaget ved at den har potentialet for at opfylde de samme krav, som både visualisering vha. kamerabilleder og computergrafik (krav 1-3). Dog er dette en teknisk udfordring, da den rummelige forståelse af den virkelige kontekst af billedet, er nødvendig for at kunne opnå en realistisk visualisering, når computergrafikken indsættes. I forhold til krav 4 har augmented reality den samme udfordring som f.eks.\ raytracing, da simulering af lampens belysning kan være en tidskrævende proces. Derudover er det svært at implementere augmented reality på en e-butiks hjemmeside, da det kræver adgang til kundens kamera. 



\clearpage

