\subsection{Teknologianalyse}

Vi ser en tydelig mulighed for at assistere forbrugere med at træffe et valg når det kommer til (køb af vare på nettet | bestemmelse af optimale lysforhold i hjemmet | visualisering af et tilkøbt element i forbrugernes dagligdag/hjem). Dette vil sandsynligvis kunne løses ved hjælp af bedre købsvejledning eller værktøjer til at assistere forbrugeren i en købssituation hvor en prøve ikke kan stilles til rådighed eller at returnere varen er umuligt eller for omfattende en process.

% // redegørelse
Blot at vælge en lampe fra et katalog er problematisk hvis der ikke er billeder af lampen som 
\begin{enumerate}
    \item Fremviser lampen som møbel, rent visuelt, det fysiske design og 
    \item Viser hvordan lys kastes af lampen. En god løsning vil være at have en fysisk model placeret i en kontekst hvor man kan komme og se lampen og se lyset i sammenspil med anden indretning, sålledes som f.eks. Ikea gør.
\end{enumerate}
\begin{figure}[H]
    \centering
    \fbox{\rule{\textwidth}{5cm}}%
    \caption{Ikea hus billede}
    % https://www.pinterest.com/pin/6685099420243693/
\end{figure}

Man vil også kunne skabe billige prototyper af lamper vha. 3D printer tekniker. Disse ville eventuelt være mulige at tage med hjem for at teste hvordan en lampe passer ind i det rum den egentligt er købt til, men fordi plastik vejer mindre end metal, glas og andre tunge materialer som lamper kan være produceret af, kunne man forestille sig ophængs metoder der ikke nødvendiggør at bore huller i væge før man har set om lampen passer ind i rummet.
En tredje metode kunne være at konstruere en 3D model af lampen og køre en simulation af hvordan den kaster lys, dette koncept vil også kunne udvides til at en forbruger kan modellere deres eget hjem og placere lampen i den model, eller det kan anvendes af sælgere som et værktøj til at vejlede forbrugeren til at gøre det rigtige køb.

