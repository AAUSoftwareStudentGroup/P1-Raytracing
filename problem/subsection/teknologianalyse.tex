\subsection{Teknologianalyse}

Vi ser en tydelig mulighed for at assistere forbrugere med at træffe et valg når det kommer til (køb af vare på nettet | bestemmelse af optimale lysforhold i hjemmet | visualisering af et tilkøbt element i forbrugernes dagligdag/hjem). Dette vil sandsynligvis kunne løses ved hjælp af bedre købsvejledning eller værktøjer til at assistere forbrugeren i en købssituation hvor en prøve ikke kan stilles til rådighed eller at returnere varen er umuligt eller for omfattende en process.

% // redegørelse
Blot at vælge en lampe fra et katalog er problematisk hvis der ikke er billeder af lampen som 
\begin{enumerate}
    \item Fremviser lampen som møbel, rent visuelt, det fysiske design og 
    \item Viser hvordan lys kastes af lampen. En god løsning vil være at have en fysisk model placeret i en kontekst hvor man kan komme og se lampen og se lyset i sammenspil med anden indretning, sålledes som f.eks. Ikea gør.
\end{enumerate}
\begin{figure}[H]
    \centering
    \fbox{\rule{\textwidth}{5cm}}%
    \caption{Ikea hus billede}
    % https://www.pinterest.com/pin/6685099420243693/
\end{figure} 

Man vil også kunne skabe billige prototyper af lamper vha. 3D printer tekniker. Disse ville eventuelt være mulige at tage med hjem for at teste hvordan en lampe passer ind i det rum den egentligt er købt til, men fordi plastik vejer mindre end metal, glas og andre tunge materialer som lamper kan være produceret af, kunne man forestille sig ophængs metoder der ikke nødvendiggør at bore huller i væge før man har set om lampen passer ind i rummet.
En tredje metode kunne være at konstruere en 3D model af lampen og køre en simulation af hvordan den kaster lys, dette koncept vil også kunne udvides til at en forbruger kan modellere deres eget hjem og placere lampen i den model, eller det kan anvendes af sælgere som et værktøj til at vejlede forbrugeren til at gøre det rigtige køb.

% Indledning over

\subsubsection{Teknologier til visualisering}
For at undersøge hvilke teknologier der kan anvendes til visualisering, er der i dette afsnit en række teknologier og metoder, som alle er relevante i forhold til at visualisere en lampe. Formålet med afsnittet er at få en forståelse af hvilke teknologier der allerede eksisterer inden for visualisering, og finde ud af hvilke metoder der er bedst i forhold til visualisering af lamper for forbrugere der handler via internettet.

\paragraph{3D print}
En teknologi som sælgeren vil kunne være i stand til at bruge er 3D printere, **hvad er 3D printere?**. Sælgeren vil kunne lave en demostrations vare som forbrugeren ville kunne være i stand til at tage med hjem, men da vi fokuserer på sælgere inden for e-handel vil dette ikke være en mulighed da e-handel som sagt ikke er en fysisk butik. I stedet kan sælgeren give forbrugeren en fil, så personen selv vil være i stand til at lave en 3D print af en bestemt lampe, dette vil dog kræve at forbrugeren har en 3D printer, som i en meget simpelt forstand er en printer der kan lave 3D figurer af et materiale, som for det meste er plastik. Disse 3D printere varierer rigtigt meget i pris og funktionalitet, dog koster nogle af de gode 3D printer over titusinde kroner \cite{3D_printer}. 
Dette vil dog ikke være en fuldstændig løsning da forbrugeren stadig vil skulle hænge lampen op for at se lysets udbredelse. Desuden vil det være en dårlig ide for sælgere at forvente at deres kunder har en 3D printer derhjemme og det kan heller ikke forventes at forbrugerne investerer så mange penge på noget som de måske kun kommer til at bruge til at lave en lampe. Et andet problem er at sælgeren også kommer i et dilemma, da sælgeren skal bestemme om man kan få disse tegninger inden man har købt lampen eller om forbrugeren skal betale en form for depositum.

\paragraph{Computergrafik}
\lipsum[3]
\subparagraph{Rasterization}
\lipsum[4]
\subparagraph{Radiosity}
\lipsum[5]
\subparagraph{Raytracing}
\lipsum[6]