\subsection{indledning}

Selvom vi ikke tænker på det så ofte, så er lamper en stor del af vores hverdag. De står i vores hjem, på vores gader, på vores arbejdsplads - ja de er stort set overalt. Men hvorfor er lamper enlig så vigtige? Det er de, fordi lamper bliver brugt til at skabe lys. Belysning kan bidrage til mange ting som at læse, arbejde mere koncentret eller til at skabe hygge og stemning i et rum, der ellers ville have været koldt og kedeligt. Lamper findes i mange forskellige typer og mange forskellige steder. Der er læselamper, arbejdslamper, loftlamper, udendørslamper osv. og de tjener allesammen forskellige formål, men fælles for dem er, at de skaber lys steder hvor der ellers ikke ville have været lys. 
I dag findes der utroligt mange forskellige lampedesigns, og de er ikke allesammen lige gode. Heraf menes der at nogle lamper har meget dårlig belysning. Dette er et problem, da undersøgelser har vist at dårlig belysning kan føre til blandt andet øjenskader, hovedpine samt ondt i nakke og hals \cite{lys_konsekvenser}. Andre problemer kan også opstå, hvis en lampe har en for kraftig belysning, og derfor er blændende eller hvis f.eks en udendørs lampe ikke lyser tilstrækkeligt, og man derfor vælter, fordi man ikke kan se noget. 
Så selvom lamper spiller en stor rolle i vores hverdag, så er det vigtigt at kunne visualisere hvordan lys udbreder sig fra en lampe, så man kan undgå dårlig belysning. 
Fordi at lamper i sig selv er et stort emne, der dækker over ting som design, indretning, belysning etc. så har vi valgt at den videre rapport skal beskæftige sig med visualisering af lys fra lamper. 


