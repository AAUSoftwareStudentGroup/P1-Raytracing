\section{Indledning}
Selvom vi ikke tænker på det så ofte, er lamper en stor del af vores hverdag. De står i vores hjem, på vores gade og på vores arbejdspladse - ja de er stort set overalt. Men hvorfor er lamper så udbredte? Det er de, fordi lamper bliver brugt til at skabe lys. Belysning kan bidrage til mange ting som at læse, arbejde mere koncentreret eller til at skabe hygge og stemning i et rum, der ellers ville have været koldt og kedeligt. 

Lamper findes i mange forskellige typer og mange forskellige steder. Der er læselamper, arbejdslamper, loftlamper, udendørslamper osv. og de tjener allesammen forskellige formål, men fælles for dem er, at de skaber lys, hvor der ellers ikke ville have været lys. 

Da belysning fra lamper fylder en stor del i rigtig mange menneskers hverdag, er det derfor interessant at undersøge de konsekvenser, der kan opstå, når en lampe ikke passer ind der hvor den bruges. Dette kan være irritation over en blændende lampe, men endnu værre kan der opstå sundhedsmæssige konsekvenser, af belysning, fra lamper der ikke er optimale i forhold til konteksten (Omtalt i afsnit \ref{sec:hvorfor_relavant}).

Da vi kan spore disse konsekvenser mere eller mindre tilbage til købet af lampen, så er er det netop købet af lampen, som er et godt udgangspunkt for nærmere undersøgelse.

Med dette udgangspunkt, har vi, som studerende på AAU Software, taget kontakt til en række lampebutikker(se bilag xxx), hvor vi afgjorde, at der er mangel på visualisering af lys fra lamper, som er et stort problem i forhold til købet, da dette netop afgør om forbrugeren kan se hvordan lyset breder sig fra en lampe, og derfor er vigtigt, når brugeren skal vælge en lampe. På baggrund af vores viden om lys fra lamper, samt de erfaringer og diskussioner vi har foretaget os som gruppe, har vi valgt at opstille følgende initierende problem:

\subsection{Initierende problem}
\textit{Forbrugeren kan ikke visualisere, hvordan lyset udbreder sig fra en lampe uden først at købe og installere lampen.}

\clearpage