\section{Indledning}
Selvom vi ikke tænker på det særligt ofte, så er lamper en stor del af vores hverdag. De står i vores hjem, på vores gade og på vores arbejdspladser - ja de er stort set overalt. Men hvorfor er lamper så udbredte? Det er de, fordi lamper bliver brugt til at skabe lys. Belysning kan bidrage til mange ting som at læse i mørke omgivelser, arbejde mere koncentreret eller til at skabe hygge og stemning i et rum, der ellers ville have været koldt og kedeligt. 

Lamper findes i mange forskellige typer og mange forskellige steder. Der er læselamper, arbejdslamper, loftlamper, udendørslamper m.fl.\ og de tjener allesammen forskellige formål, men fælles for dem er, at de skaber lys, hvor der ellers ikke ville have været lys. 

Da belysning fra lamper fylder en stor del i rigtig mange menneskers hverdag, er det derfor interessant at undersøge de konsekvenser, der kan opstå, når en lampe ikke passer ind der, hvor den bruges, samt at udvikle en løsning på dette problem. Dette kan være irritation over en blændende lampe, men endnu værre kan der opstå sundhedsmæssige konsekvenser, af belysning, fra lamper der ikke er optimale i forhold til konteksten (Omtalt i afsnit \ref{sec:hvorfor_relavant}).

Da vi, mere eller mindre, kan spore disse konsekvenser tilbage til købet af lampen, er det netop købet af lampen, som er udgangspunktet for vores undersøgelse og løsning.

Med dette udgangspunkt, har vi, som studerende på AAU Software, taget kontakt til 10 lampebutikker (Bilag \ref{sec:mailbelysning}) for, at få mere viden inden for området. Herefter afgjorde vi, at der er mangel på visualisering af lys fra lamper, som er et stort problem i forhold til købet, da dette netop afgør om forbrugeren kan se hvordan lyset breder sig fra en lampe, og derfor er vigtigt, når brugeren skal vælge en lampe. På baggrund af vores viden om lys fra lamper, samt de erfaringer og diskussioner vi har foretaget os som gruppe, har vi valgt følgende initierende problem:

\subsection{Initierende problem}
\textit{Kunden kan ikke visualisere, hvordan lyset udbreder sig fra en lampe uden først at købe og installere lampen.}

\clearpage
\subsection{Rapportens struktur og metodiske overvejelser}
Rapportstrukturen afspejler den metodiske tilgang til projektet. Rapportens afsnit er struktureret i kronologisk rækkefølge fra indledning til perspektiveringen (punkt 2 - 9). 
En kort gennemgang af rapportens afsnit:
\begin{enumerate}
\setcounter{enumi}{2}
\item Problemanalyse: En analyse af problemfeltet for at få en dybere forståelse for det initierende problem. Afsnittet indeholder følgende underafsnit:
  \begin{itemize}
    \item Relevans: Argumenterer for relevansen af det initierende problem.
    \item Begrebsliggørelse: Redegører for begreber til forståelse af det initierende problem.
    \item Interessentanalyse: Undersøger interessenterne og hvilken målgruppe rapporten vil arbejde videre med. I afsnittet har gruppen anvendt den kvalitative undersøgelse til bl.a.\ at samarbejde med erhvervslivet heraf designere, lampebutikker og belysningskonsulenter. Formålet med interessentanalysen er at finde ud af hvilke interessenter, denne rapport vil løse problemet for.
    \item Problemets placering: Undersøger det initierende problems placering, samt hvor rapporten vil fokusere på at løse problemet.
  \end{itemize}
\item Problemformulering: Formulerer det problem, som skal løses.
\item Løsningsdesign: Gennemgår gruppens løsningsforslag af problemet samt hvilke teknologier der er til visualisering. Afsnittet indeholder følgende underafsnit: 
  \begin{itemize}
    \item Løsningsforslag: Skitserer et løsningsforslag og opstiller krav til løsning.
    \item Teknologier til visualisering: Beskriver hvilke teknologier, der er til at visualisere en lampe og dens belysning.
  \end{itemize}
\item Problemløsning: Undersøger hvilke teorier, der skal til for at løse problemet, samt implementationen og test af løsningen. Afsnittet indeholder følgende underafsnit:
  \begin{itemize}
    \item Teori: Undersøger den teori, som skal til for at lave løsningen.
    \item Udvikling: Dokumenterer udviklingen af løsningen.
    \item Test af programmet: Tester om løsningen lever op til de krav som blev sat til løsningsforslaget.
  \end{itemize}
\item Diskussion: Diskuterer resultaterne og udarbejdelsen af rapporten. Afsnittet indeholder følgende underafsnit:
  \begin{itemize}
    \item Løsningen i forhold til løsningsforslaget: Diskuterer i hvor høj grad løsningen lever op til de krav som blev stillet i løsningsforslaget.
    \item Løsningen i forhold til problemformuleringen: Diskuterer i hvor høj grad løsningen lever op til besvarelsen af den endelige problemformulering.
    \item Diskussion af det initierende problem: Diskuterer de valg og antagelser, som blev foretaget i forbindelse med udarbejdelsen af problemformuleringen.
  \end{itemize}
\item Konklusion: Konkluderer i hvor høj grad løsningen løser problemet opstillet i problemformuleringen, samt hvilke mangler der er.
\item Perspektivering: En diskussion af hvad der skal til for at arbejde videre med løsningen, samt alternative anvendelsesmuligheder for løsningen.
\end{enumerate}

\clearpage