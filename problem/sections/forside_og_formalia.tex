\begin{titlepage}
  %\toprule[2pt]
  %\midrule
  \vspace{0.2cm}
  \begin{center}
    \Huge{\textbf{Visualisering af lampers belysning}}
  \end{center}
  \vspace{0.2cm}
  %\midrule
  %\toprule{2pt}
  \begin{center}
    \Large{\textbf{Gruppe B2-28}}\\
	Morten Rask Andersen\\
	Christian Mønsted Grünberg\\
	Anton Christensen\\
	Mathias Ibsen\\
	Lasse Fribo Gadegaard\\
	Mathias Rohde Pihl
  \end{center}
  \vfill
  \begin{center}
 	\today\\
    Aalborg Universitet\\
    Software, 1. semester
  \end{center}
\end{titlepage}



\section{Forord}
Denne rapport er udarbejdet af gruppen B2-28, bestående af software-studerende, som P1-rapport på Aalborg Universitet.

Rapporten tager udgangspunkt i Aalborg-modellen for problembaseret læring. Denne læringsproces har givet gruppen mulighed for at undersøge en given problemstilling og derudfra tilegne sig viden, og på baggrund af denne viden udarbejde en problemanalyse. Derudover gør rapporten også brug af den kvalitative metode til korrespondance med interessanterne. Den kvalitative metode er fordelagtige at bruge, når man vil undesøge forhold som er svære at iagtage eller måle \cite{kvalitativ_metode}. Problemanalysen har herefter, gennem et samarbejde med erhvervslivet, hjulpet med at afgrænse og danne fundamentet for problemløsningen. I forbindelse med samarbejdet med erhvervslivet har vi taget kontakt til en række lampebutikker. Disse butikker og deres kommentarer vil efter eget ønske fremgå anonymt. 

Tak til vejledere Benjamin Bjerre Krogh og Annette Grunwald samt de medvirkende belysningskonsulenter, designere og lampebutikker.

\subsection{Læsevejledning}
Rapporten er skrevet med en rød tråd, hvilket vil sige, at afsnittene er struktureret således, at der gerne skulle skabes en sammenhæng mellem afsnittene og herefter en helhed. Det er dog ikke nødvendigt at læse hele rapporten, da hvert afsnit har sin egen indledning og opsummering hhv. først og sidst i afsnittet.

Programmet er uploadet til Digital Eksamen og er derfor elektronisk bilag. 

\subsubsection{Kildehenvisning}
Rapportens brug af kildehenvisninger er baseret på nummermetoden \cite{nummermetoden}. I nummermetoden anføres kilderne i fortløbende nummerorden, svarende til hvilket nummer, de har i teksten. To identiske kilder har samme nummer. Herunder ses et eksempel på henvisninger til hhv.\ internetkilder, artikler og bøger:

Interneteksempel med kilde[1].

[1] Titel på emne eller kort forklaring på emnet, hjemmesidenavn. Set DD-MM-YYYY. URL på hjemmeside.


Bogeksempel med kilde[2].

[2] Titel på bog, udgavenummer, forfatter(e), udgivelsesår. ISBN/ISSN-nummer.


Hvis en kilde har yderligere relevante informationer (såsom sidetal, ophavsret mm.\ angives disse også i kilden.


Figurhenvisning foregår på samme måde som med andre kilder, dog med en forklaring under selve figuren. Hvis en figur ingen kilde har, er figuren fremstillet af gruppen.


\subsubsection{Kodeuddrag}
Flere steder i rapporten, vil der blive vist dele af gruppens kode. Et eksempel på hvordan dette vil blive vist er herunder:

\begin{lstlisting}[style=Cstyle, caption=Kodeeksempel i C]
#include <stdio.h>

int main(void){
   printf("Hello, world!\n");
   return 0;
}
\end{lstlisting}








\clearpage