\section{Konklusion}

Der vil nu kort blive opsummeret den løsning der er opnået i projektet samt de mangler som programmet har i forhold til problemformuleringen. 

I projektet er der udviklet et program der kan følgende:
\begin{itemize}
\item Visualisere en lampe og dens belysning som ved brug af raytracing rendere et billede ud fra en 3D-fil der indeholder lampen, lyskilder og kontekst.
\item Skifte vinklen hvorfra lampen og dens belysning visualiseres ved brug af rotationsmatricer. 
\item Skifte farvetemperaturen for pæren i lampen ud fra en algortitme der omdanner farvetemperaturen i kelvin til RGB-værdier. 
\end{itemize}

Programmet har følgende mangler i forhold til at kunne implementeres på en e-butiks hjemmeside:
\begin{itemize}
\item Manglende realisme af refleksion og refraktion.
\item For høj renderingstid til at løsningen er praktisk anvendelig.
\item Manglende brugerflade til visning af billeder.
\end{itemize}

Ud fra ovensående kan vi konkludere at løsningen delvist besvarer problemformuleringen, men at der stadigvæk er mangler i forholdd til implementeringen af løsningen på en e-butiks hjemmeside. Derfor mangler der stadigvæk at blive testet om løsningsforslaget vil gøre det nemmere for kunden at visualisere en lampe og dens belysning på en e-butiks hjemmeside.  

\clearpage