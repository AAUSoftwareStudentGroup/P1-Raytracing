\section{Konklusion}
Formålet med dette afsnit er, at opsummere hvordan den endelige problemformulering blev udarbejdet. Derudover konkluderes der hvorvidt den udviklede løsning besvarer den endelige problemformulering.

I problemanalysen, blev der argumenteret for det initierende problems relevans. I denne forbindelse blev der antaget, at det initierende problem eksisterede, på baggrund af egne erfaringer, diskussion med lektor Lars Peter Jensen og en udtagelse fra en belysningskonsulent for en dansk lampebutik. Ønsket var at bekræfte antagelsen ved at udføre en spørgeskemaundersøgelse. Da denne undersøgelse aldrig blev fuldført, er det ikke bevist, at det initierende problem eksisterer.
Efter argumentationen for problemets relevant blev begreber, interessenter og placering af det initierende problem undersøgt. Resultatet af disse undersøgelser dannede grundlaget for den endelige problemformulering.

Det første underspørgsmål i problemformuleringen var: \textit{"Hvordan visualiseres lyset fra en given indendørslampe?"}. Dette underspørgsmål blev besvaret ved at repræsentere lampen, som en 3D-model, som sammen med brugerinput om den ønskede visualisering, indlæses af programmet og renderes vha.\ raytracing med brug af phong-modellen beskrevet i afsnit \ref{sec:teori}.

Det andet underspørgsmål var: \textit{"Hvordan kan lampen og dens belysning visualiseres fra flere vinkler?"}. Dette underspørgsmål blev besvaret ved at anvende teorien om rotationsmatricer i afsnit \ref{sec:teori}, til at positionere det virtuelle kamera, som billedet renderes ud fra.

Det tredje underspørgsmål var: \textit{"Hvordan visualiseres forskellige pærers lys?"}. Her blev der anvendt en algoritme, som konverterer en farvetemperatur for pæren til en RGB-værdi, som kunne benyttes i phong-modellen.

Problemformuleringens overordnede spørgsmål var \textit{Hvordan kan vi lave et værktøj til e-butikker, som visualiserer belysningen fra indendørslamper for kunderne?"}. Ud fra ovenstående kan det nu konkluderes, at problemformuleringen er delvist løst, da der er udviklet et værktøj, der kan visualisere en lampe og dens belysning med forskellige synsvinkler og farvetemperaturer. Problemformuleringen er kun delvist løst, da der stadigvæk er mangler i forhold til implementeringen af værktøjet på en e-butiks hjemmeside. Programmet mangler følgende:

\begin{itemize}
\item Bedre overenstemmelse af farvetemperaturen for pæren i den rigtige lampe i forhold farvetemperaturen på det renderede billede.
\item Lavere renderingstid for, at løsningen er praktisk anvendelig.
\item Brugerflade til visning af billeder.
\end{itemize}

\clearpage