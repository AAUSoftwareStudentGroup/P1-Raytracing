\section{Appendiks}
\section*{Projektforslag}
\section*{IT og læring i folkeskolen}
Mange folkeskoler bruger i dag computeren i undervisningen. Hvordan kan man anvende computeren til at fremme indlæring i folkeskolen?

\subsection*{Problemstilling}
Folkeskolen er grundlaget for den videre uddannelse af den nye generation. Da mange folkeskoler anvender computere i undervisning, er det oplagt at undersøge hvordan der kan udvikles et værktøj til at styrke indlæringen i undervisningen ved brug af computere. \\
Hvordan udvikles et værktøj til computeren, der kombinerer IT og læring i folkeskolen?

\subsection*{Mål}
Målet er at udvikle et program, der kombinere IT og læring. Er der nogle problemer ved nuværende læringsværktøjer i folkeskolens undervisning, som kan løses ved hjælp af en datalogisk løsning?

\subsection*{Ekspempler på datalogiske problemstillinger}
Datastrukturer, algoritmer, udvikling af grafisk brugerflade (GUI).

\subsection*{Ekspempler på kontekstuelle problemstillinger}
Hvem er relevante at kontakte, når der skal udvikles et nyt læringsværktøj? Hvem skal læringsværktøjet udvikles til? Hvilke krav er der til et læringsværktøj? 


\subsection*{Forslagsstiller}
Gruppe B2-28 (SW1b2-28@student.aau.dk)
\clearpage