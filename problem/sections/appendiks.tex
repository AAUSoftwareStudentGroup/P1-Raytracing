\section{Appendiks}
\section*{Projektforslag}
\section*{Produktvisualisering ved e-handel}
I takt med flere og flere computere i individers hjem, er adgangen til internettet blevet større. Grudet dette er e-handelbranchen boomet og har gjort indkøb af forskellige varer meget nemmere, men den er stadig ikke optimal. Et af de største problemer ved e-handel er manglende visualisering af de udbudte produkter. Hvis man leder efter en vare, kan man som regel nemt finde den på en internetside ved en hurtig søgning på internettet, men beskrivelsen er kort, målene er svære at visualisere og man kan ikke forestille sig varen i en given sammenhæng.

\subsection*{Problemstilling}
Hvis køberen ser et produkt, som umiddelbart ser godt ud, og derfor køber det, men grundet manglende visualisering, viser det sig at produktet ikke passer kunden, kan dette skabe en situation, hvor køberen er nødt til at returnere produktet. Dette stiller både sælger, køber og kunde utilfredse. Hvordan kan man hjælpe køberen med at købe et produkt fra en e-handelsbutik, så både kunden og sælgeren opnår den største tilfredshed?

\subsection*{Mål}
Målet er at definere en model, som kan optimere e-handelsbrancen, så f.eks\. fejlkøb undgås, ved hjælp af en eller flere af understående datalogiske problemstillinger.

\subsection*{Ekspempler på datalogiske problemstillinger}
Raytracing, rasterisering, augmented reality.

\subsection*{Ekspempler på kontekstuelle problemstillinger}
Hvilke forskellige metoder er der til visualisering, og hvilke er de mest optimale til at visualisere et givent produkt fra en e-handelsbutik?

\subsection*{Forslagsstiller}
Gruppe B2-28 (SW1b2-28@student.aau.dk)
\clearpage