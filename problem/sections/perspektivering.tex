\section{Perspektivering}
I dette afsnit forklares der hvad der skal til for at arbejde videre på løsningen samt alternative anvendelsesmuligheder. 

For at arbejde videre på løsningen kræver det følgende:

\begin{itemize}
\item En brugerflade til visning af billeder på e-butiks hjemmeside.
\item Optimering af raytraceren så renderingstiden formindskes. 
\item Viderudvikling af raytaceren med henblik på højere realisme.
\item Højere overenstemmelse med reel farvetemperatur for pærer.  
\item At lave en backend for kommunikationen mellem serveren hvor billederne renderes og en e-butiks hjemmeside.
\end{itemize}

Selvom løsningsforslaget i denne rapport er målrettet mod kunder som handler lamper på en e-butiks hjemmeside, så er der andre anvendelsesmuligheder for løsningsforslaget. Det kunne tænkes at løsningsforslaget også kunne visualisere lamper og deres belysning for kunder der handler i en detailbutik ved at have en computer eller tablet som har implementeret løsningen. 

I rapporten er det beskrevet, hvordan man kan konstruere et program der renderer et billede på baggrund af en 3D-fil. Denne løsningsmodel kan nødvendigvis ikke kun bruges på lamper, men er også anvendelig til visualisering af andre produkter, som f.eks.\ elektronikprodukter, møbler mm.\ 

\clearpage