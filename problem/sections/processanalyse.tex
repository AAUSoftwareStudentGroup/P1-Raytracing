\clearpage

\documentclass[oneside,a4paper,titlepage]{article}
\usepackage{blindtext}
\usepackage[utf8]{inputenc}
\usepackage{pdfpages}
\usepackage{graphicx}
\usepackage{geometry}
\usepackage{float}


\begin{document}

\section{Beskrivelse - Hvordan forløb processen i P1}
% I skal beskrive jeres P1 projektproces så detaljeret som muligt. I må gerne komme ind på alle de aspekter I finder relevante. I skal komme ind på følgende områder:

Lige efter gruppedannelsen skete der ikke så meget i projektet, da vi var meget i tvivl om vores projektforslag. Projektforslaget var nemlig ikke fremstillet på samme måde som de andre. Under et af lektionerne til PV skulle vi i gruppen udfylde en række spørgsmål til f.eks.\ hvordan gruppen vil håndtere konflikter og hvordan vores samarbejdsaftale så ud, disse spørgsmål kan ses i bilag \ref{sec:samarbejdsaftale}. Der var også en lektion i PV, der handlede om at få styr på projektet, denne omhandler planlægning, værktøjer og roller \ref{sec:styr_paa_projektet}.\newline\newline
I gruppen benytter vi os meget af "Peer-learning". Grunden til dette er at gruppens medlemmer har forskellige kompetencer og dette kan vi udnytte vi at hjælpe hinanden. Vi har hjulpet hinanden ved at et gruppemedlem laver en kort fremlæggelse på tavlen, så det andre gruppemedlemmer om ikke andet kan få en nogenlunde forståelse og et godt udgangspunkt til viderearbejde.
Til konflikter som f.eks.\ overskredet deadlines eller at komme for sent til gruppearbejde har vi været meget large. Enten er deadlines blevet rykket eller man har givet kage hvis man er kommet meget for sent i en periode. \newline\newline
Til disse spørgsmål havde vi til samarbejdsaftalen aftalt at møde 8:15 og derefter arbejde fra 8:30 til 16:15. Efterhånden som projektet forløb blev lavet om til at møde 8:30 og derefter til vi ikke rigtigt gad mere og havde nået det vi ville på dagen. Vi har ikke udarbejdet en fast tidsplan, men der er derimod en consensus om at der er frokost klokken 12. Der er dage hvor frokosten rykker sig, hvis vi glemmer tiden når vi har været meget fokuseret på arbejdet. Vores skriftlige samarbejdsaftale er derfor ikke særlig lang, vi har dog mange mundtlige aftaler og gensidig respekt for hinanden. \newline\newline
Vi lavede en Belbin rolletest, hvor vi på baggrund af denne, bestemte gruppens kontaktperson og hvem der skulle træde til som leder hvis det blev nødvendigt. Vi har derfor ikke haft en fast projektleder, da vi mente det var unødvendigt og formentlig fordi vi på daværende tidspunkt syntes det ville være lidt akavet at skulle agere chef over for de andre i gruppen.\newline\newline
Til samarbejde med vores vejledere, gav begge vejledere os en liste over hvilke krav de havde til samarbejdet. Disse lister har fungeret som vores samarbejdsaftale med vejlederne. De omhandlede bl.a.\ i hvilken format arbejdsblade skulle sendes i, hvor lang tid de skulle have til at læse dem og at kun referentens pc skulle være åben til vejledermøderne. Inden et vejledermøde havde vi lavet en agenda, som på forhånd blev sendt til vejlederen. Vi havde to former for vejledermøder: Den ene var at vi sendte vores arbejdsblade hvorefter vi vil få feedback i form af kommentarer over mail. Den anden var det mere traditionelle vejledermøde hvor vejlederen havde udprintet rapport og kommentarerne blev diskuteret face-to-face. \newline\newline
Da vi skulle til at lave den initierende problemstilling lavede vi en overordnet tidsplan til hvornår vi skulle have de store emner gjort færdig. Ud fra denne sørgede vi at give os selv opgaver og deadlines løbende, så dette mål ville blive opfyldt. \newline
Når vi skulle uddele opgaver til os selv, havde vi skrevet opgaverne op og derefter kunne man selv sætte sig på noget eller blive sat på noget. Motivering i gruppen kom løbende, vi motiverede hinanden og hjalp til, så vi ikke kørte død i skrivearbejdet. Jokes, små spil og sociale pauser hjalp meget med dette. 
Hver morgen lavede vi dagsordner til hvad vi vil lave den dag. Dette hjalp med at overholde tidsplanen. Til at organisere vores arbejde har gruppen benyttet sig af Trello, da det er meget nemt at lave nye kort med opgaver og man kan hurtigt få oversigt over dem. Som det kan ses på figur \ref{fig:trello} kan man se hvem der er på hvilke opgaver og hvor langt opgaven er fra at være færdig.

\begin{figure}[H]
    \centering
    \includegraphics[width=10cm]{../graphics/trello}
    \caption{Et udklip fra vores løsningsboard på Trello}
    \label{fig:trello}
\end{figure} 
Vi brugte Git til vores version-control således at alle hurtigt kunne den hente den nyeste version. Git blev også brugt til at holde styr på koden. I forbindelse med projektet kontaktede vi flere personer og virksomheder, vi skrev til 10 lampebutikker, to designere og IKEA. Ansvar for denne kommunikationen lå på kontaktpersonen, som var ansigtet ud ad til. \newline\newline
Opgaverne uddeles for de meste med først til mølle, men der bliver stadigt taget højde for omfanget af opgaven, så det ikke ender med at en person har en 30 minutters opgave mens en anden har en to timers opgave. Enten vil personen med den lille opgave få noget mere at lave eller den store opgave opdeles og laves med en mere person. Vi holder korte og uformelle møder her og der, hvor vi kommer med en status på mangler og lignende i rapport eller koden. Da møderne i sig selv er så uformelle, er der ingen mødeleder eller runde om bordet. Hvis man har noget at byde ind med byder man ind. \newline\newline
Vi har ikke haft en seriøs snak om hvad vi forventer af hinanden, det ligger meget implicit når vi arbejder. Der er dog nogle forventninger, der er blevet sagt f.eks.\ at vi møder til tiden og at man laver det man er blevet bedt om. Ambitionen ligger selvfølgelig højt, vi vil gerne lave noget vi kan stå inde for og gøre vores bedste. Vi forventer ikke at gruppemedlemmer tager med på bar eller andet, hvis de ikke har lyst. Det er noget man selv bestemmer om man vil eller ej, vi vil dog stadig spørge efter det og prøve at få en person til komme med, da vi kun har kendt hinanden i under et halvt år. \newline\newline
Vi hjælper hinanden med gruppeopgaver til det forskellige kurser. I programmering foregik dette ved at vi fælles skrev koden og viste den på en projektor. I matematik lavede vi opgaver selv så meget som muligt, men man kunne frit slå sig sammen med en anden eller spørge. Vi sørgede for at alle havde en forståelse for hvordan opgaven skulle løses. Hvis én forstod noget som nogle andre ikke gjorde holdte denne person ofte en kort fremlæggelse på tavlen, så de andre havde en god nok forståelse til at kunne lave opgaverne. \newline\newline
En af de arbejdsprocesser som vi havde i gruppen, var at vi hver morgen lavede en dagsorden for dagen. Vi vil i dette afsnit kort forklare og illustrere nogle af de dagsplaner som vi havde under P1. 
Her er et par eksempler på dagsordner i løbet af projektet:
\paragraph{Dagsorden 9/10/15:}
\begin{itemize}
  \item Brainstorm.
  \item Diskussion af spørgsmål til styringsgruppe mødet.
  \item Agenda.
  \item Find de programmer vi vil bruge til projektarbejde, heraf undersøg Trello, og Github.
  \item Opsummering og afrunding  af dagen.
\end{itemize}
Her ses en typisk dagsorden for starten af projektet, hvor vi brugte lang tid på at diskutere hvilket emne vi ville arbejde med, samt at få styr på praktiske ting som programmer til fildeling.
Et andet eksempel på en dagsorden kunne se således ud:
\paragraph{Dagsorden 20/11/15:}
\begin{itemize}
  \item Agenda til Annette.
  \item Planlæg.
\end{itemize}
I slutningen af projektet var der fokus på at samle trådene og en dagsorden så derfor ofte således ud:

\paragraph{Dagsorden 14/12/15:}
\begin{itemize}
  \item Revidere afgrænsning af løsningsforslag.
  \item Skrive afsnit om rotationsmatricer i udviklingsafsnittet.
  \item Fordele og ulemper ved augmented reality.
  \item Skitse til Phong.
  \item Ret sekvensdiagram.
  \item Skriv testafsnit færdigt.
\end{itemize}
Her ses der en dagsorden lavet nogle dage før projektafleveringen, og her er der fokus på at rette eventuelle mangler.

%læreprocessen (herunder problemorienteringens røde tråd)

%Det er vigtigt at I venter med at analysere jeres erfaringer til efter at I er færdige med at beskrive dem ! 

\section{Vurdering - Hvordan gik det}
\begin{itemize}
  \item Opgaveuddeling ved hjælp af Trello gik godt i starten og knap så godt i slutningen af projektet.
  \item Git som version-control var rigtigt god til at holde styr på koden, men knap så god til latex.
  \item Dagsordnerne gik godt, men nogle gange var de meget korte og upræcise.
  \item Grupperolletesten er slet ikke blevet brugt udover at give os selv kendskab til hvilke roller vi har.
  \item Gruppesamarbejdet gik okay, der var nogle gode ting og nogle dårlige ting. 
  \item Vejledersamarbejdet gik godt.
  \item Samarbejdet med erhvervslivet hjalp meget med god viden til viden rapporten, selvom vi blev afvist af IKEA.
  \item kommunikationen i gruppen er hovedsageligt god og samtaler kan godt afsluttes når vi skal arbejde. 
  \item Nogle i gruppen kan være ufokuseret og følger ikke med om morgenen når dagsordenen laves, hvilket ofte ender i at den person ikke laver noget.
  \item Gruppens konflikthåndtering har været meget dårlig.
  \item Vi har været dårlige til at stå op og "skælde" en person ud hvis de ikke følger med.
  \item Gennemgang af rettelser på projekter var godt.
  \item Fællesregning af gruppeopgaver fra kurser var godt.
\end{itemize}

% Trello er godt med mindre boards, godt til hjemmeopgaver
% Git er godt til koden, men måske ikke fan til latex, to forskellige repository
% Mere konkrete dagsordner, som bliver tjekket op på. evt. navne på opgaven
% Har ikke brugt grupperolletesten, andet end få et kendskab til hvilke roller der er, og hvilke vi er. 
% Gruppesamarbejde gik okay
% Vejledersamarbejde gik godt
% Samarbejde med erhvervsliv gik godt 

Hvordan er kommunikationen i jeres gruppe ? Er der nogle der taler hele tiden ? Er der nogen der aldrig siger noget ? Bruger gruppen uforholdsvis lang tid på diskussionerne ? Hvorfor ?

%Når I er færdige med at beskrive hvad I gjorde, skal I vurdere hvordan det gik. Med andre ord: Hvad gik godt i P1? Hvad gik dårligt i P1? 

\section{Analyse – Hvorfor gik det som det gik?}
%Dernæst skal I analysere jeres arbejdsprocesser og få klarlagt hvorfor noget gik godt mens andet gik dårligt. Med andre ord: Hvad er det for faktorer, som har indvirket på arbejdsprocesserne? 

\section{Syntese – Gode råd til P2}
% samarbejdsaftale, snakke sammen om konflikter og engagement

%Hvis jeres vurdering og analyse skal bidrage til at forbedre jeres evne til at håndtere det
%problemorienterede og projektorganiserede gruppearbejde, skal I til slut konkretisere jeres erfaringer i nogle ’Gode råd’ til jer selv og jeres medstuderende. En god måde at formulere sådanne gode råd på er som en *start-stop-fortsæt*-liste, dvs. en liste med følgende tre sektioner:
%– Dette vil vi begynde at gøre i P2, som vi ikke gjorde i P1 
%– Dette vil vi ikke gøre i P2, som vi gjorde i P1
%– Dette vil vi fortsætte med at gøre (gerne anderledes og bedre) i P2, som vi også gjorde i P1
%Det er en god idé at tage ét område ad gangen og gøre det færdigt. De ’Gode råd’ skal være konkrete og operationelle, så de fører til reelle forbedringer i P2. 


Spørgsmål til inspiration
Når I skal skrive P1-procesanalysen kan I lade jer inspirere af spørgsmålene herunder, men I må
gerne medtage andre emner, som I mener har haft betydning for jeres arbejdsprocesser.
Projektplanlægning
Har alle i gruppen samme opfattelse af hvad projektplanlægning er ? Find ud af det.

Hvad vil I foreslå til planlægning og styring af et P2 projekt ?

Samarbejdet i gruppen

Hvilke forventninger har I til samarbejdet i P2 ? Hvordan skal de blive opfyldt ?

Samarbejdet med vejlederne
Hvilken type respons ønsker I fra vejlederen ?
Hvilken type vejledning har I modtaget ? Var det hvad I ønskede jer ?
Hvilke forventninger vil I stille til jeres vejledere i P2 ?

Læreprocesserne
Hvordan hjælper I hinanden med at løse opgaver i kurserne ?
Hvad gør I hvis I ikke forstår det der bliver sagt/eller står i bogen ?
Hvordan lærer du bedst ?
Hvordan har I brugt resultaterne af jeres individuelle læringstest ?
Hvordan hjælper/stimulerer vejlederen jeres læreprocesser ?
Hvilken læringsstrategi er bedst til kurser ?
Hvilken læringsstrategi er bedst til projektarbejde ?


\clearpage
\section{Bilag}

\subsection{Belbins teamroller}
\label{sec:styr_paa_projektet}
\section*{PV – Få styr på projektet}
\subsection*{Planlægning}
\begin{itemize}
  \item Initierende problem – 15 / 10
  \item Problemanalyse \& problemformulering – 14 / 11
  \item Løsning (metode) – 14 / 11
  \item Udvikling – 7 / 12
  \item Dokumentation – 12 / 12
  \item Konklusion – 14 / 12
  \item Afslutning og aflevering – 18 / 12
  \item Procesanalyse – 22 / 12
\end{itemize}

\subsection*{Værktøjer}
\begin{itemize}
  \item Trello
  \item Github
  \item Latex
  \item Google docs
\end{itemize}
Trello er til opgaveadministration.
Git er til version-control.
Docs er til fællesrettelser.
Latex er det system vi skriver rapporten i.
\subsection*{Roller}
\begin{itemize}
  \item Kontaktperson: Lasse
  \item Referent: Skiftende rolle
  \item Ordstyrer: Skiftende rolle
  \item Leder: Anton og Morten
\end{itemize}
Gruppen har taget en rollemodels test: \newline
Kristian Træhold: Formidler og Specialist. \newline
Christian Grunberg: Formidler og Organisator.\newline
Mathias Ibsen: Formidler og Specialist.\newline
Lasse Gadegaard: Organisator, kontaktskaber og formidler.\newline
Mathias Pihl: Organisator og specialist.\newline
Morten Rask: Organisator, koordinator, opstarter, afslutter, specialist.\newline
Anton Christensen: Koordinator, opstarter.\newline

\subsection*{Rollebeskrivelser}
Formidler: Social personlighed som er meget diplomatisk og løser konflikter.
Specialist: Har en stor faglig viden indenfor et område. Fokuserer på sit arbejde.
Organisator: Sørger for at tingene er i orden, og at vi har en tidsplan.
Kontaktskaber: Social person der holder styr på kommunikation, og alt det formelle.
Koordinator: Leder som tager beslutninger, og uddelegerer opgaver
Opstarter: Er god til at sætte folk i gang, og vil hele tiden lave noget
Afslutter: En der kan samle trådene, og ikke lader et projekt løbe ud af tangenter
Idémand: En person som altid ser nye muligheder og er en god debatstarter.

\begin{figure}[H]
   \centering
   \includegraphics[width=10cm]{../graphics/rolletest}
   \caption{Resultatet af rolletesten}.
\end{figure}

\subsection*{Kommentar til skema:}
Vi kan se, at der generelt er en fin fordeling af roller, hvor den eneste som ikke er grøn er ”afslutter”, men her ses der at mange gruppemedlemmer har denne som 4 og 5 prioritet. Det er derfor oplagt at tage denne i fællesskab. Derudover har vi 3 roller der er i SURPLUS. Her skal vi være opmærksom på, at der ikke bliver lavet for meget dobbeltarbejde, og at vi alle går i samme retning.

\subsection{Samarbejdsaftale mm.}
\label{sec:samarbejdsaftale}

\section*{Hvordan benytter gruppen resultaterne fra læringsstiltestene til at gruppemedlemmerne lærer bedre ”til og fra hinanden” (Peer læring)?}
Vi har fået en forståelse for at folk lærer og opfatter tingene på forskellige måder. Vi skal derfor være åbne for forskellige forslag, og have en accept for at folk lærer på forskelligt. Derudover skal vi inkorporere de forskellige læringsmetoder i vores gruppearbejde, hvor personer der f.eks. lærer bedst ved ”forklaring” – også får en mulighed for at forklare hvad personen har lært for andre.
\section*{Hvad vil i gøre i gruppen, hvis der opstår konflikter, som kræver en løsning?}
Vi skal finde frem til kernen af konflikten. Vi diskutere efterfølgende problemet, hvor vi har en ordstyrer som styrer slagets gang, og alle får derefter lov til at udtrykke deres mening/holdning om problemet. Derefter kan man finde eventuelle ligheder og forskelle, og derudfra finde en løsning på konflikten.
\section*{Hvordan ser gruppens samarbejdskontrakt ud?}
Vi møder hverdag kl 8:15 og begynder gruppearbejdet 8:30. Hvilket vil sige, at man har 15min til ”fri leg” – og sociale samtaler.
Normal arbejdsdag: 8:30 – 16:15 med mulighed for ændringer hvis ALLE er enige.
Vi vil lave et mål for hver dag. Altså en dagsorden for hvad der skal nås.
Man SKAL informere ”inden” mødetid hvis man ikke kan møde til tiden.
\section*{Vejleder:}
At vi foreslår, at vores arbejdspapirer bliver delt over Google Docs. og vejlederen har derefter mulighed for at kommentere i Google Docs. Vi laver en redigeringsaftale med vejlederen.
Lav en Agenda til hver møde
\clearpage
\subsection{Mail korrespondance}
Fra: XX \newline
Sendt: 6. november 2015 13:36 \newline
Til: Lasse Fribo Gadegaard\newline
Emne: SV: Semesterprojekt om lamper - AAU\newline
Hej
 
Det at se en lampe i 3D gør ikke at man ser lyset. Nogle af vores producenter laver allerede 3D modeller af deres lamper og endda sådan at man kan se lampen med lys i. Jeg har lige vedhæftet Artemides udgave af fremgangsmåden.

Men derfra til at se hvordan lyset er i et konkret rum hvor farver, højde mv. har indflydelse på lyset gør at det bliver en ekstrem kompleks størrelse der kræver komplicerede belysningsberegningsprogrammer som f.eks. DIALux. Lysberegning handler primært om lysmængde og ikke om lyseffekt.

Kunder har svært ved at forstå er hvilken lyseffekt lampen giver. Det er jo to-siddet. Dels vil de se hvordan lampen ser ud i rummet og dels vil de se hvordan lyseffekten er. Første del klarer flere producenter. Hvis man så bagefter at man har sat lampen ind som 3D model burde man lave et lag over billedet hvor producenten så har taget et billede af et rum hvor man kan se skygger mv. Eks Tom Dixon. Det er det jo ikke lampe man køber men ofte lyseffekt og lampe handler jo om at forme lyset og lave skygger. At sætte en Tom Dixon lampe ind i et rum gør ikke at du kan se skyggerne.

Det er blevet en kompliceret proces at producere en lampe ift. EU lovgivning i dag så jeg har svært ved at se at producenterne vil koste endnu flere penge til produkter til privatmarkedet som måske kun køber en lampe til 3000 kr. som ofte kun interesserer sig for den laveste pris og ikke den bedste service og rådgivning. Så producenters incitament til ligge investeringer hos privatkunder er meget begrænset. Mange laver end ikke et fritskravet billede af deres lampe. Og igen Tom Dixon anvender en klar halogenlyskilde. Hvis du sætter en klar kultrådslyskilde hvor filamentet er længere giver forsat skygger men på en blødere måde da lyset er delt ud på en større overflade. Hvis du sætter en mat lyskilde i forsvinder skyggerne næsten helt. Pludselig er løsningen bare komplekst og når en producent så har 4000 varenumre. Samme lyskilder laves så i 3 farvetemperaturer. Med vores adgang til varesortiment giver det 2 mio. billeder dokumenter som skal indhentes. Så er vi der hvor det begynder ikke at hænge sammen tidsmæssigt når nethandlen handler om at være først ift. googlesøgninger mv. Hvem har lyst til at give bedste rådgivning ift. lyseffekter hvis man ender på side 20 når folk søger på google?  

Løsningsforslag modtages derfor med kyshånd da kompleksitet er desværre nem at se. 

I må selvfølgelig også gerne vores udtagelser.  

Med venlig hilsen / best regards

XX\newline
Belysninskonsulent

\noindent\makebox[\linewidth]{\rule{\paperwidth}{0.4pt}}

Fra: Lasse Fribo Gadegaard [mailto:lgadeg15@student.aau.dk] \newline
Sendt: 6. november 2015 11:19\newline
Til: XX\newline
Emne: SV: Semesterprojekt om lamper - AAU

Hej 
 
Vi er meget glade for at I vil bidrage til projektet. Indtil videre har vi analyseret problemet: 
Forbrugeren kan ikke visualisere, hvordan lyset udbreder sig fra en lampe uden at købe og installere lampen. 

I problemanalysen har vi undersøgt interessenter, begreber, placering og teknologier til problemet. 

Ud fra dette har vi valgt at fokusere på e-butikker, der sælger indendørs lamper til brug i erhverv eller private hjem. 

Vi har netop udarbejdet den endelige problemformulering, hvor udkastet lyder som følgende:
Hvordan kan man lave et værktøj til e-butikker som vha. raytracing*, visualiserer belysningen fra indendørs lamper for kunderne? 

* En teknik til at simulere lys og lave et billede af en 3D-model. 

Vi skal nu til at udvikle en løsning til problemet, og hertil har vi lavet en simpel skitse (Se vedlagt billede) af den ide vi har på nuværende tidspunkt. 
 
Som vist på skitsen, er ideen at lave et produkt som gør det muligt for kunderne at se nogle lamper og deres belysning i et interaktivt 3D-billede på e-butikken. 

Vi tænker at forbrugeren skal kunne gøre følgende: \newline
 - Vende og dreje billede, så de kan se lampen og belysningen fra flere vinkler. \newline
 - Se lampen med forskellige pærer (evt. angive farvetemperatur i Kelvin) \newline
 - Skifte den kontekst som lampen visualiseres i (f.eks. forskellige rum/møbler)

Pga. tidsbegrænsning forventer vi ikke at lave hele løsning, som produkt, men blot implementere de mest studierelevante dele. 

Dog skal vi stadigvæk præsenterer en færdig løsning i rapporten. 

Det vi nu ønsker jeres respons på, er følgende
 - Jeres tanker omkring ideen, som løsning på problemet.\newline
 - Forslag og ønsker til forbedringer af ideen. \newline
 - Jeres accept til, at vi i rapporten må inddrage jeres udtagelser anonymt.

Med venlig hilsen,\newline
Lasse Gadegaard
 
På vegne af \newline
Gruppe B2-28\newline
Software, AAU
 
\noindent\makebox[\linewidth]{\rule{\paperwidth}{0.4pt}}

Fra: XX \newline
Sendt: 5. november 2015 15:43\newline
Til: Lasse Fribo Gadegaard\newline
Emne: SV: Semesterprojekt om lamper - AAU\newline
Hej Lasse

Det lyder til at være et meget spændende projekt.
 
Som primær detailforretning med projektafdeling lever vi af konsulentarbejde ved at give rådgivning omkring hvordan lys forandre sig ift. til lofthøjde, farver, armatur, lyskilde foruden at der er en subjektiv mening om hvad godt lys er.

Der er overraskende mange der gerne vil se lyset inden de køber lamper. Man kan dog undre sig ovre at samme kunde køber både køleskabe, vaskemaskiner mv. uden at stille krav til at prøve tingene før de køber varerne selv om disse produkter koster lige så meget som de lamper vi sælger. Kunder har åbenbart et specielt forhold til lys. 

Har i allerede valgt teori, metode og empiri? 

Jeg tror godt vi kan hjælpe jer. Eneste krav er at data herfra bliver anonymiseret og vi får et eksemplar af opgaven når den er skrevet.

God dag

Med venlig hilsen / best regards\newline
XX\newline
Belysningskonsulent

\noindent\makebox[\linewidth]{\rule{\paperwidth}{0.4pt}}

Fra: Lasse Fribo Gadegaard [mailto:lgadeg15@student.aau.dk] \newline
Sendt: 5. november 2015 15:06\newline
Til: XX\newline
Emne: Semesterprojekt om lamper - AAU
 
Hej XX
 
Vi er en gruppe på Aalborg Universitet, som er i gang med et projekt om visualisering af lamper.
 
Vi arbejder med følgende problemstilling:
"Forbrugeren kan ikke visualisere, hvordan lyset udbreder sig fra en lampe uden at købe og installere lampen." 

Vi har fokus på e-handel, og ønsker at tilbyde e-butikken et værktøj som gør det nemmere for kunderne at visualisere, hvordan lyset breder sig ud fra en lampe (f.eks. hvilke skygger, mønstre og farver som lampen udsender). 

Derfor søger vi nu e-butikker, som ønsker at bidrage med viden og informationer omkring e-handel med lamper.  

Hvis I er interesserede i at medvirke i projektet, så skriv venligst tilbage på mail: lgadeg15@student.aau.dk 

Med venlig hilsen,\newline
Lasse Gadegaard\newline
På vegne af\newline
Gruppe B2-28\newline
AAU, Software

\subsection{Brainstorm}
\begin{figure}[H]
   \centering
   \includegraphics[width=10cm]{../graphics/brainstorm_1}
   \caption{Billede af første brainstorm}.
\end{figure}

\subsection{Dagsorden}
\begin{figure}[H]
   \centering
   \includegraphics[width=10cm]{../graphics/dagsorden}
   \caption{Billede af en god dagsorden}.
   \label{fig:dagsorden}
\end{figure}

\end{document}
