\subsubsection{Lysets påvirkning på mennesket} 
\label{sec:konsekvenser}

Med udgangspunkt i bogen "Human Factors in Lighting" af Peter R. BOYCE undersøges der hvilke konsekvenser dårlig belysning kan have på mennesket. 

De fleste problemer opstår hvis man sidder for længe i lys som øjnene opfatter som værende ubehageligt eller forstyrrende. Den mest normale konsekvens af dette er overanstrengelse af øjnene. Symptomerne på overantrengelse af øjnene er: irratation af øjnene som viser sig som betændelse omkring øjne og øjenlåg. En anden symptom er forringet syn som kan opleves som dobbeltsyn og sløring af synet. Derudover kan der forekomme bivirkninger som bla. hovedpine, forstoppelse og svimmelhed s.533.

Konsekvenser ved dårlig belysning samt de efterfølgende bivirkninger ved det opfattet forskelligt fra menneske til menneske, hvoraf nogle mennesker er særligt sårbare overfor dårlig belysning, heraf den gruppe som har fotoepilepsi, her kan ubehageligt og flimrende lys forudsage anfald s.534. 

Et problem for især ældre mennesker er, at de ofte vælter da deres syn er blevet forringet. Dette har formindsket deres ballance og evne til at afstandsvurdere. God belysning kan være med til at styrke de sanser som holder kroppen i balance. Heraf kan natlamper blandt andet være en løsning om natten. 