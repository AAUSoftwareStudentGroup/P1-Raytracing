\subsubsection{Lysets påvirkning på mennesket} 
\label{sec:konsekvenser}
Undersøgelser har vist, at lys har stor indvirkning på vores humør og trivsel\cite{videnskab_dk_paavirkning}.
Blåt/hvidt lys har den effekt, at vi føler os mere vågne og oppe på mærkerne, hvorimod rødt/gult lys har den modsatte effekt\cite{videnskab_dk_paavirkning}. Det giver god mening, når vi ser på det lys der er om dagen som ofte er lyst og blåligt pga.\ den blå himmel. Hvorimod lyset om aftenen og natten ofte er gult/rødt ved f.eks.\ solnedgang.

Vi bruger også lys til rigtig meget i hverdagen. Øjet, som virker ved hjælp af lys bruger vi til at navigere, se eventuelle farer og genkende venner.

Mange mennesker sidder i kontormiljøer i stort set al deres arbejdstid og som nævnt i indledningen, er de afhængige af, at lyset er godt. Hvis lyset ikke er godt, vil man ofte blive træt og uproduktiv. Derfor er det vigtigt at både loftslamper, skrivebordslamper og indretningen spiller godt sammen og skaber et godt lys-miljø.
