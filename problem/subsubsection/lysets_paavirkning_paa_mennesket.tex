\subsubsection{Lysets påvirkning på mennesket} 
\label{sec:konsekvenser}

Med udgangspunkt i forskellige artikler og bogen "Human Factors in Lighting" af Peter R. Boyce, undersøges der hvilke konsekvenser dårlig belysning kan have på mennesket. 

De fleste problemer opstår hvis man sidder for længe i lys som øjnene opfatter som værende ubehageligt eller forstyrrende. Den mest normale konsekvens af dette er overanstrengelse af øjnene. Symptomerne på overantrengelse af øjnene er: irratation af øjnene som viser sig som betændelse omkring øjne og øjenlåg. En anden symptom er forringet syn som kan opleves som dobbeltsyn og sløring af synet. Derudover kan der forekomme bivirkninger som bla. hovedpine, forstoppelse og svimmelhed \cite{human_factors}(side 533).

Konsekvenser ved dårlig belysning samt de efterfølgende bivirkninger ved det opfattet forskelligt fra menneske til menneske, hvoraf nogle mennesker er særligt sårbare overfor dårlig belysning, heraf den gruppe som har fotoepilepsi, her kan ubehageligt og flimrende lys forudsage anfald \cite{human_factors}(side 534). 

Et problem for især ældre mennesker er, at de ofte vælter da deres syn er blevet forringet. Dette har formindsket deres ballance og evne til at afstandsvurdere. God belysning kan være med til at styrke de sanser som holder kroppen i balance. Heraf kan natlamper blandt andet være en løsning om natten. 

Dårlig belysning kan have negative konsekvenser for mange forskellige faktorer i en kontorarbejders hverdag. Det kan f.eks. føre til overanstrengelse af øjet, hovedpiner og træthed \cite{ergonomi_arbejdsplads}. Det fysiske og psykiske ubehag kan formindske kontorarbejderens produktivitet, og det er derfor i virksomhedens bedste interesse at komme dette problem til livs. Ifølge OSHA, estimerer nogle undersøgelser at op til 90\% af de 70 millioner amerikanske arbejdere, der benytter en computer i deres arbejdstid i mere end tre timer, oplever computer vision syndrome, som er betegnelsen for synsproblemer forudsaget af belysningen fra en computerskærm \cite{CVS}. Ifølge artiklen "The Ergonomics of Light" kan den rigtige arbejdslampe ift. arbejdskonteksten øge bekvemmeligheden, produktiviteten og moralen\cite{ergonomi_arbejdsplads}. 