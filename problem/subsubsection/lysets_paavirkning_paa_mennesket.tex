\subsubsection{Lysets påvirkning på mennesket} 
\label{sec:konsekvenser}

Med udgangspunkt i forskellige artikler og bogen "Human Factors in Lighting" af Peter R. Boyce, undersøges der hvilke konsekvenser dårlig belysning kan have på mennesker. 

De fleste problemer opstår, hvis man sidder for længe i lys som øjnene opfatter som værende ubehageligt eller forstyrrende. Den mest normale konsekvens af dette er overanstrengelse af øjnene. Symptomerne på overanstrengelse af øjnene er: irritation af øjnene som viser sig som betændelse omkring øjne og øjenlåg. Et anden symptom er forringet syn, som kan opleves som dobbeltsyn og sløring af synet. Derudover kan der forekomme bivirkninger som bl.a.\ hovedpine, forstoppelse og svimmelhed \cite{human_factors}(side 533).

Konsekvenser ved dårlig belysning samt de efterfølgende bivirkninger ved det opfattes forskelligt fra menneske til menneske. Nogle mennesker er særligt sårbare overfor dårlig belysning, heraf den gruppe som har fotoepilepsi, her kan ubehageligt og flimrende lys forårsage anfald \cite{human_factors}(side 534). 

Et problem for især ældre mennesker er, at de risikerer at vælte, da deres syn er blevet forringet. Dette har formindsket deres balance og evne til at vurdere afstande. God belysning kan være med til at styrke de sanser som holder kroppen i balance. Heraf kan en passende natlampe bl.a.\ være en løsning om natten \cite{human_factors}. 

Dårlig belysning kan have negative konsekvenser for mange forskellige faktorer i en kontorarbejders hverdag. Det kan f.eks.\ føre til overanstrengelse af øjet, hovedpine og træthed \cite{ergonomi_arbejdsplads}. Det fysiske og psykiske ubehag kan formindske kontorarbejderens produktivitet, og det er derfor i virksomhedens bedste interesse at komme dette problem til livs. I følge OSHA \cite{OSHA}, estimerer nogle undersøgelser, at op til 90\% af de 70 millioner amerikanske arbejdere, der benytter en computer i deres arbejdstid i mere end tre timer oplever 'computer vision syndrome', som er betegnelsen for synsproblemer forudsaget af belysningen fra en computerskærm \cite{CVS}. Ifølge artiklen "The Ergonomics of Light" kan den rigtige arbejdslampe ift.\ arbejdskonteksten øge bekvemmeligheden, produktiviteten og moralen \cite{ergonomi_arbejdsplads}. 