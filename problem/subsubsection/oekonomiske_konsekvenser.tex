\subsubsection{Økonomiske konsekvenser}
Når en kunde har valgt at investere i en lampe, uden at have haft muligheden for at visualisere lampens lys på en ordentlig måde på forhånd, kan det have økonomiske konsekvenser for én eller flere interessenter. Kunden, der køber en lampe, som de er utilfredse med, kan i de fleste situationer ikke få byttet produktet, da den originale emballage er brudt, og ledningen er pillet ved. Dette kan føre til en utilfreds kunde, som muligvis skal ud og investere i en ny lampe, og en lampebutik, med en potentiel utilfreds kunde, da de ikke har sørget for, at kunden kunne visualisere præcis hvordan lampen og dens lys ville se ud. Lampebutikken kan derfor opleve et mindsket salg af en hvis lampe, hvilket kan resultere i et mindsket antal af lamper, som de køber hjem fra producenten. Dette kan endvidere medføre et mindsket antal af producerede lamper, og som i sidste ende kan ende med at designeren også får færre penge.


Det er blevet en kompliceret proces at producere en lampe ift. EU lovgivning i dag så jeg har svært ved at se at producenterne vil koste endnu flere penge til produkter til privatmarkedet som måske kun køber en lampe til 3000 kr. som ofte kun interesserer sig for den laveste pris og ikke den bedste service og rådgivning. Så producenters incitament til ligge investeringer hos privatkunder er meget begrænset