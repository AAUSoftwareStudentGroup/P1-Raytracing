\subsubsection{Pixels}
I programmet har vi valgt at indføre en algoritme, der kan ’oversætte’ en farvetemperatur, målt i kelvin, til en RGB-værdi (rød-, grøn-, blå-værdi), som tilsammen viser en bestemt farve. Dette program kan hjælpe kunden med at visualisere den farve, som en lampe vil udsende, da det er vigtigt hvilken farvetemperatur et lys fra en lampe har (se afsnit \ref{sec:lys} om lys). Algoritmen er lånt og oversat fra pseudokoden fra følgende hjemmeside \href{http://www.tannerhelland.com/4435/convert-temperature-rgb-algorithm-code/}{http://www.tannerhelland.com/} under afsnittet 'The Algorithm'. Algoritmen er lavet ud fra at plotte Charity’s original blackbody values, som kan findes her: \href{http://www.vendian.org/mncharity/dir3/blackbody/UnstableURLs/bbr_color.html}{http://www.vendian.org/}. Disse værdier rangerer fra 1.000 til 40.000 kelvin, og det er derfor ikke sikket, at værdier ude for disse grænser er ligeså præcise i overgangen fra farvetemperatur til RGB-værdi. Copyright m.m. mht. algoritmen står i selve programmet som en kommentar i delen ’pixel.c’.


For at beregne en farve af en pixel, er det først nødvendigt at lave en struct pixel, der fortæller programmet, hvad den består af. 
\begin{lstlisting}[style=Cstyle, caption=Struct af Pixel]
typedef struct _pixel {
  double red, green, blue;
} Pixel;
\end{lstlisting}

Herefter laves der funktionen create_from_color_temperature(), som er af typen Pixel, og som tager en unsigned int kelvin som input. Ud fra de konklusioner, der blev omtalt i teoriafsnittet \ref{sec:temp_til_rgb}, som omhandlede konvertering fra farvetemperatur til RGB, har man ud fra da benyttede data kunne opstille nogle konklusioner, der har gjort algoritmen mere simpel. Det er derfor muligt at beregne en given farve på følgende måde: 
\begin{lstlisting}[style=Cstyle, caption=Beregning af rød farve]
if(kelvin <= 66){
  color.red = 255;
}
else{
  color.red = kelvin - 60;
  color.red = 329.698727446 * (pow(color.red, -0.1332047592));
  if(color.red < 0){
    color.red = 0;
  }
  if(color.red > 255){
    color.red = 255;
  }
}
\end{lstlisting}