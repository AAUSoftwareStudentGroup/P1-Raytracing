\subsubsection{Detailhandel}
En fysisk butik er et sted, hvor kunderne selv skal komme hen, når de vil købe eller kigge på butikkens varer. En fysisk butik har et personale, som tager sig af butikkens kunder, og som besvarer deres mulige spørgsmål til butikkens varer. En fysisk butik er, grundet det ansatte personale og andre udgifter, dyr i omkostning. Dog viser en amerikansk undersøgelse, at 92\% af de udspurgte 1029 personer i undersøgelsen foretrækker de fysiske butikker, da man kan se og føle de varen selv og få direkte assistance om varen gennem en medarbejder \cite{fysisk_kontra_online}. Det var ikke muligt at finde en lignende rapport, der viste noget om danskernes meninger mht.\ om de foretrækker at føle den fysiske vare, før de køber den, men grundet kulturelle ligheder mellem USA og Danmark, har vi valgt at antage at en lignende tendens er gældende i Danmark.

I vores kontekst snakker vi om en, hvilken som helst, fysisk butik, der har med lampesalg at gøre. Den lampeinteresserede kunde, kommer ud i butikken, og leder f.eks.\ efter en ny væglampe til stuen. Problemet heri kan opstå ved, at der er adskillige forskellige lamper at vælge imellem, men ikke alle lamperne er tilsluttet og man har derfor ikke mulighed for, at se lampens belysning i rummet.

\label{sec:sammenligning_af_e_og_d}
Ved køb af lamper i den fysiske butik, er det ikke altid, at de lamper, som stilles frem til udstilling giver et realistisk billede af, hvordan lyset udbreder sig, da der ofte er mange lamper og lyskilder tæt samlet. Et andet problem er returretten. Returretten er ikke obligatorisk at have for fysiske butikker, så det er altså op til den enkelte butik/butikskæde, om de vælger at gøre det muligt at returnere en vare selvom den er uåbnet og stadig i original emballage \cite{fortrydelsesret}. Hvis kunden beslutter sig for en lampe, som ikke er tilsluttet, men regner med at den vil se godt ud på væggen i stuen, hvorefter det så viser sig, at lyset falder helt forkert, er det for sent, da lampen er pakket ud, og ledningen er blevet pillet ved. Lampen kan altså ikke byttes, og er ikke optimal i forhold til kundens stue. Der er mange butikker og butikskæder, som vælger at gøre det muligt for kunden at bytte en vare, hvis den stadig er i original emballage og med kvittering \cite{ikea_returret}, men det er ikke noget, som butikker er tvunget til at gøre, og med en lampe som eksempel kan man ikke tage den med hjem og "afprøve", da dette giver afkald på returretten.