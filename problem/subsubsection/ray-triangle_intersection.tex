\subsubsection{Skæring mellem linje og trekant i rummet}
\label{sec:triangle_intersection}
For at finde om der er en skæring mellem en stråle og en trekant, kan man følge tre trin:
\begin{enumerate}
  \item Påvis at strålens linje ikke er parralel med trekantens plan
  \item Find skæringen med trekantens plan og påvis at punktet er foran strålens udgangspunkt
  \item Påvis at punktet i trekantens plan, ligger indenfor trekanten
\end{enumerate}

En stråle er paralel med et plan hvis strålens retning er 90 grader relativt til planets normalvektor, dette er sandt hvis ligning \ref{eq:parralel} er opfyldt.

\begin{equation}
  \label{eq:parralel}
  \vv{r} \bullet \vv{n} = 0
\end{equation}

Planets normalvektor findes ved at krydse to vektore fra en af trekantens hjørner til de to andre (se ligning \ref{eq:triangle_normal}).

\begin{equation}
  \label{eq:triangle_normal}
  \vv{n} = (B - A) \times (C - A)
\end{equation}

\begin{figure}[H]
  \centering
  \tdplotsetmaincoords{60}{130}
  \begin{tikzpicture}[tdplot_main_coords]
    % dashed line through plane
    \draw [black, thick] (0,0,0) -- (4,-8,4);
    % triangle and plane
    \path[fill=gray!10, draw=gray!20] (0,-4,0) -- (4,-4,0) -- (4,-4,4) -- (0,-4,4) -- (0,-4,0);
    \path[fill=gray!30, draw=gray!60] (1,-4,2) -- (2,-4,3) -- (3,-4,1) -- (1,-4,2);
    % triangle
    \draw [blue!50, thick, -{Stealth[width=3mm, length=3mm]}] (2,-4,3) -- (3,-4,1);
    \draw [blue!50, thick, -{Stealth[width=3mm, length=3mm]}] (2,-4,3) -- (1,-4,2);
    \draw [blue!50, thick, -{Stealth[width=3mm, length=3mm]}] (2,-4,3) -- (2,-2,3);
    % \draw [black,   thick, -{Stealth[width=2mm, length=2mm]}] (2,-4,3) -- (2,-4,2);
    \node [above] at (2,-3,3) {$\vv{n}$};
    \node [above] at (2,-4,3) {$A$};
    \draw plot [mark=*, mark size=1] coordinates{(2,-4,3) }; 
    \node [left] at (3,-4,1) {$B$};
    \draw plot [mark=*, mark size=1] coordinates{(3,-4,1) }; 
    \node [below] at (1,-4,2) {$C$};
    \draw plot [mark=*, mark size=1] coordinates{(1,-4,2) };
    
    % ray
    \node [above right] at (0,0,0) {$P_0$};
    \draw plot [mark=*, mark size=1] coordinates{(0,0,0) };
    \node [above right] at (0.25,-0.5,0.25) {$\vv{r}$};
    \draw [black, thick] (0,0,0) -- (2,-4,2);
    \draw [blue!50, thick, |-|] (0.1,0,-0.1) -- (2.1,-4,1.9);
    \node [below left] at (1,-2,1) {$t_A$};
    \node [black, above left] at (2.5,-4,2) {$P(t_A)$};
    \draw [black, thick, -{Stealth[width=3mm, length=3mm]}] (0,0,0) -- (0.5,-1,0.5);
    \draw plot [mark=*, mark size=1] coordinates{(2,-4,2) };
  \end{tikzpicture}
  \caption{Viser princippet bag perspektiv projektion af et punkt på et billedplan.}
  \label{fig:perspektiv_projektion}
\end{figure}

Hvis strålen ikke er parralel med planet kan vi nu finde et punkt i planet som strålen skære. Alle vektore i planet er ortogonale på normal vektoren, dermed kan vi opstille ligning \ref{eq:vektor_in_plane} til at beskrive betingelsen for et punkt strålen som også ligger i planet. Ved at substituere linjens ligning (ligning \ref{eq:point_on_ray}) ind og isolere $t_A$ kan vi så bruge til at finde punktet på strålens linje og er $t_A > 0$ så er punktet også foran strålens udgangspunkt $P_0$

\begin{align}
  \label{eq:point_on_ray}
  P(t) &= \vv{r} \cdotp t + \vv{P_0} \\
  \label{eq:vektor_in_plane}
  ( P(t_A) - A ) \bullet \vv{n} &= 0 \\
  \label{eq:test3}
  (\vv{r} \cdotp t_A + \vv{P_0} - A) \bullet \vv{n} &= 0 \\
  \label{eq:test4}
  t_A \cdotp \vv{r} \bullet \vv{n} + (\vv{P_0} - A) \bullet \vv{n} &= 0 \\
  \label{eq:test5}
  t_A &= -\frac{(\vv{A} - \vv{P_0})\bullet \vv{n}}{\vv{r} \bullet \vv{n}}
\end{align}


[FIND IF POINT IS INSIDE TRIANGLE]()

\begin{figure}[H]
  \centering
  \tdplotsetmaincoords{60}{130}
  \begin{tikzpicture}[tdplot_main_coords,thick,scale=2, every node/.style={transform shape}]
    \coordinate (Pt) at (2,-4,2);
    \coordinate (A) at (2,-4,3);
    \coordinate (B) at (3,-4,1);
    \coordinate (C) at (1,-4,2);
    % dashed line through plane
    % triangle and plane
    \path[fill=gray!10, draw=gray!20] (0.8,-4,0.8) -- (3.2,-4,0.8) -- (3.2,-4,3.2) -- (0.8,-4,3.2) -- (0.8,-4,0.8);
    \path[fill=gray!30, draw=gray!60] (A) -- (B) -- (C) -- (A);
    % triangle
    \draw [blue!50, thick, -{Stealth[width=3mm, length=3mm]}] (A) -- (B);
    \draw [blue!50, thick, -{Stealth[width=2mm, length=2mm]}] (A) -- (Pt);
    \draw [blue!50, thick, -{Stealth[width=3mm, length=3mm]}] (B) -- (C);
    \draw [blue!50, thick, -{Stealth[width=2mm, length=2mm]}] (B) -- (Pt);
    \draw [blue!50, thick, -{Stealth[width=3mm, length=3mm]}] (C) -- (A);
    \draw [blue!50, thick, -{Stealth[width=2mm, length=2mm]}] (C) -- (Pt);
    \node [above] at (2,-3,3) {$\vv{n}$};
    \node [above] at (2,-4,3) {$A$};
    \draw plot [mark=*, mark size=1] coordinates{(2,-4,3) }; 
    \node [left] at (3,-4,1) {$B$};
    \draw plot [mark=*, mark size=1] coordinates{(3,-4,1) }; 
    \node [below] at (1,-4,2) {$C$};
    \draw plot [mark=*, mark size=1] coordinates{(1,-4,2) };
    
    % ray
    \node [black, above left] at (2.5,-4,2) {$P(t_A)$};
    \draw plot [mark=*, mark size=1] coordinates{(2,-4,2) };
  \end{tikzpicture}
  \caption{Viser princippet bag perspektiv projektion af et punkt på et billedplan.}
  \label{fig:perspektiv_projektion}
\end{figure}