\subsubsection{Skæring mellem linje og trekant i rummet}
\label{sec:triangle_intersection}
For at finde om der er en skæring mellem en stråle og en trekant, kan man følge tre trin:
\begin{enumerate}
  \item Påvis at strålens linje ikke er parralel med trekantens plan
  \item Find skæringen med trekantens plan og påvis at punktet er foran strålens udgangspunkt
  \item Påvis at punktet i trekantens plan, ligger indenfor trekanten
\end{enumerate}


[FIND IF RAY PARRALEL WITH TRIANGLE PLANE]()

En stråle er paralel med et plan hvis strålens retning er 90 grader relativt til planets normalvektor, dette er sandt hvis ligning \ref{eq:parralel} er opfyldt.

\begin{equation}
  \label{eq:parralel}
  \vv{r} \bullet \vv{n} = 0
\end{equation}

Planets normalvektor findes ved at krydse to vektore fra en af trekantens hjørner til de to andre (se figur \ref{eq:triangle_normal}). såvidt

\begin{equation}
  \label{eq:triangle_normal}
  \vv{n} = (B - A) \times (C - A)
\end{equation}

\begin{figure}[H]
  \centering
  \tdplotsetmaincoords{60}{130}
  \begin{tikzpicture}[tdplot_main_coords]
    \draw [black, thick, -{Stealth[width=3mm, length=3mm]}] (0,0,0) -- (4,-8,4);
    \path[fill=gray!10, draw=gray!20] (0,-4,0) -- (4,-4,0) -- (4,-4,4) -- (0,-4,4) -- (0,-4,0);
    \path[fill=gray!30, draw=gray!60] (1,-4,2) -- (2,-4,3) -- (3,-4,1) -- (1,-4,2);
    \draw [blue!50, thick, -{Stealth[width=3mm, length=3mm]}] (2,-4
    ,3) -- (3,-4
    ,1);
    \draw [blue!50, thick, -{Stealth[width=3mm, length=3mm]}] (2,-4,3) -- (1,-4,2);
    \draw [blue!50, thick, -{Stealth[width=3mm, length=3mm]}] (2,-4,3) -- (2,-2,3);
    \node [above] at (2,-3,3) {$\vv{n}$};
    \node [above] at (2,-4,3) {$A$};
    \draw plot [mark=*, mark size=1] coordinates{(2,-4,3) }; 
    \node [left] at (3,-4,1) {$B$};
    \draw plot [mark=*, mark size=1] coordinates{(3,-4,1) }; 
    \node [below] at (1,-4,2) {$C$};
    \draw plot [mark=*, mark size=1] coordinates{(1,-4,2) };
    \node [above right] at (0,0,0) {$P$};
    \draw plot [mark=*, mark size=1] coordinates{(0,0,0) };
    \node [above right] at (1,-2,1) {$\vv{r}$};
    \draw [black, thick] (0,0,0) -- (2,-4,2);
    \draw plot [mark=*, mark size=1] coordinates{(2,-4,2) };
  \end{tikzpicture}
  \caption{Viser princippet bag perspektiv projektion af et punkt på et billedplan.}
  \label{fig:perspektiv_projektion}
\end{figure}


[FIND IF PLANE IS IN FRONT OF RAY ORIGIN]()

[NEDENSTÅENDE SKAL OMSKRIVES EN SMULE]()

punktmændgen som opgør planet kan beskrives som de punkter $\vv{p}$ der opfylder reglen $ \vv{p} \bullet \vv{n} = 0$ 
\begin{align}
  P(t) &= \vv{r} \cdotp t + \vv{p_0} \\
  ( P(t_A) - A ) \bullet \vv{n} &= 0 \\
  (\vv{r} \cdotp t_A + \vv{p_0} - A) \bullet \vv{n} &= 0 \\
  t_A \cdotp \vv{r} \bullet \vv{n} + (\vv{p_0} - A) \bullet \vv{n} &= 0 \\
  t_A &= -\frac{(\vv{A} - \vv{p_0})\bullet \vv{n}}{\vv{r} \bullet \vv{n}}
\end{align}


[FIND IF POINT IS INSIDE TRIANGLE]()