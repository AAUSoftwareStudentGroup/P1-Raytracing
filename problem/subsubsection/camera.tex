\subsubsection{Camera}
Når vi begynder at render vores scene skyder vi rays fra kameraet og ud i scenen. For nemt at kunne flytte kameraet og gøre programmet mere læsbart, har vi en række funktioner til at manipulerer kameraet.

\begin{lstlisting}[style=Cstyle, caption=Definition af Camera struct ]
typedef struct _camera {
  Vector up;
  Vector right;
  Vector forward;
  Vector position;
  unsigned int width, height;
  double distance;
} Camera;
\end{lstlisting}

[Forklar om vector up, right og forward]

'Vector position' bruger vi til at definerer hvor kameraet er i vores scene i forhold til (0, 0, 0). I kombination med up, right og forward ved vi hvor vores rays skal skyde hen og kan dermed renderer et billede.

'uint width' og height definerer hvor høj opløsning vores billede skal være i pixels, og hvor tæt vores rays skal skydes.

'double distance' er hvor langt væk vores kamera er fra vores skærm-gitteret. Flytter man f.eks. skærm-gitteret længere væk fra kameraet uden at gøre den større vil man få en mindre FOV\cite{fov}, og få effekten af at "zoome ind".

\begin{lstlisting}[style=Cstyle, caption=Prototyper for funktioner der manipulerer camera-structen]
void camera_look_at_point(Camera *camera, Vector point, double distance, double vertical_angle, double horizontal_angle);
int camera_set_angle(Camera *camera, double vertical_angle, double horizontal_angle);
\end{lstlisting}

camera\_look\_at\_point: tager et camera, et punkt defineret som en vektor, en længde og 2 vinkler i radianer. 'camera' parameteren tager et camera og ændrer dens up, right, forward vektorer til at pege mod punktet defineret af 'point', med vinklen defineret af vertical_angle og horizontal_angle og med afstanden distance fra punktet.

camera\_set\_angle: tager et camera og 2 vinkler som doubles. funktionen nulstiller kameraets vinkel og drejer så kameraet om sig selv med vinklen defineret af de 2 doubles.