\subsubsection{Computergrafik}
\label{sec:computergrafik}
I computergrafik, er en 3D model, en beskrivelse af objekters form og materiale.\cite{computergrafik_introduktion} Computergrafiske metoder kan bruges til at immitere hvordan lys interaktere med modellen og på den måde tegne et billede af modellen. Der eksisterer en mængde forskellige computergrafiske metoder, flere af hvilke kan bruges sammen med andre for, at opnå et mere realistisk eller effektivt resultat. Der findes flere produkter som kan visualisere produkter til salg på websites som f.eks. Cylindo\cite{Cylindo}. Men vi har ikke kendskab til at andre specialisere sig, eller markedsføre sig på nuværende tidspunkt med deres kompetencer med fokus på visualisering af lampers belysning.

\paragraph{Rasterisering}
er en metode til at visualisere miljøer med høj aktiv brugerinteraktion som f.eks. computerspil.\cite{rastarization} Metoden virker ved rent mattematisk at projektere modellen på et billedplan som repræsentere skærmen.\cite{rastarization}. Fordelen ved resterisering er at disse projektioner, kan foretages meget hurtigt af computerens grafikkort, som bygget specielt til formålet\cite{rastarization}. Dette kan dog mindske fleksibiliteten, og muligheden for mere avancerede visualisering, hvor der kræves refleksioner og refraktioner af lys, som ikke passer ind i den proces (graphics pipeline\cite{rastarization}, som de enkelte grafikkort danner billeder ud fra. 

\paragraph{Ray tracing}\cite{raytracing_for_begyndere} forsøger, nøjagtigt at simulere lys i et virtuelt miljø, i modsætning til rasterisering hvor hastighed er den primære faktor. Raytracing bygger fundementalt på at følge stråler af lys og bygge en model for hvordan de stråler interaktere med forskællige objekter og materialer. Der skælnes mellem to typer af raytracing: Forwards raytracing og backwards raytracing.

\subparagraph{Forwards raytracing}\cite{radiosity_by_wpi,radiosity_by_uob} er oftere kaldet radiosity og vil bliver reffereret til som sådan fremhenværende. Radiosity er hvad man kunne kalde en forwards raytracing metode. Her eksistere lyskilder ikke som specielle objekter i en 3D model, hvilket er tilfældet for de andre metoder, men her som objekter uden forskel fra de andre i modellen.

I radiosity modellen er alle flader betegnet med en absorbans faktor og en energi faktor. Absorbansen beskriver hvor meget af lys der rammer fladen der bliver absorberet. Absorberet lys hæver en flades energi og som i virkeligheden, afgives noget af den energi som lys, mens andet bliver omdannet til f.eks. varme. Lyskilder er således blot flader som starter med en mængde energi.

Radiosity er i stor grad blandt de mest tidskrævende metoder eftersom at den laver beregninger som ikke nødvendigvis bliver set i et billede. Dette muliggøre dog at visualisere en scene en gang og derefter at kunne se den fra mange vinkler eftersom at de ekstra udregninger allerede er gjort. Radiosity er derimod ikke designet til at håndtere fænomener som er afhængig af hvor man ser et objekt fra, så som refleksion og refraktion. Dvs. at radiosity ikke kan håndtere metalliske overflader eller semitransparente materialer. Til gengæld er Radiosity rigtig god til at simulere matte overflader og skygger.

\subparagraph{Backwards raytracing} er hvad man i almindelighed kalder raytracing og vil bliver reffereret til som sådan fremhenværende. Raytracing forsøger at simplificerer den fysiske model af lys ved at ignorere det lys som ikke rammer vores øjne. Raytracing er dog alligevel blandt de metoder som kendes for at kunne skabe de mest fotorealistiske renderinger. Dette gøres ved såkaldt \textit{backwards raytracing}, hvorved man følger en stråle fra øjet og ud mod 3D modellen, så tjekkes der for kollisioner mellem strålen og objekterne i modellen. Ved hver kollision kan man vælge at følge yderligere stråler som kan hjælpe med at udregne refleksioner eller komplekse skygger. Denne metoder står i modsætning til hvad man kalder \textit{forwards raytracing} som er den mere fysisk korrekte metode, hvor man følger stråler af lys fra hver lyskilde.

I forhold til rasterisering, tager raytracede billeder væsentligt længere tid at tegne, men komplekse lysfænomener som refleksioner og lys forvrængninger igennem semitransparante medier som vand(kaldet refraktion) er simple at beskrive for en raytracing algoritme, som kan tegne disse med realistisk precision. Nogle fænomener som bløde skygger kan også beskrives men jo flere typer fænomener og jo større realisme der kræves des længere tid tager det at tegne et billede, men raytracing tillader stor fleksibilitet.
