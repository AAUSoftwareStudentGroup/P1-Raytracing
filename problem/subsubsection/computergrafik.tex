\subsubsection{Computergrafik}
\label{sec:computergrafik}
I computergrafik, er en 3D model, en beskrivelse af objekters form og materiale.\cite{computergrafik_introduktion} Computergrafiske metoder kan bruges til at simulere, hvordan lys interagere med modellen og på den måde tegne et billede af modellen. Der eksisterer forskellige computergrafiske metoder, flere af hvilke kan bruges sammen med andre for, at opnå et mere realistisk eller effektivt resultat. Der allerede værktøjer, som kan visualisere produkter til salg på websites som f.eks. Cylindo\cite{Cylindo}. Men vi har ikke kendskab til at andre specialisere sig, eller markedsføre sig på nuværende tidspunkt med deres kompetencer med fokus på visualisering af lampers belysning. Derfor er der herunder beskrivelse af to af de mest anvende metoder inden for computergrafik.

\paragraph{Rasterisering}
er en metode til at visualisere miljøer med høj aktiv brugerinteraktion som f.eks. computerspil.\cite{rastarization} Metoden virker ved rent matematisk at projektere modellen på et billedplan som repræsentere skærmen.\cite{rastarization}. Fordelen ved rasterisering er at disse projektioner, kan foretages meget hurtigt af computerens grafikkort, som er bygget specielt til formålet\cite{rastarization}. Dette kan dog mindske fleksibiliteten, og muligheden for mere avancerede visualisering, hvor der kræves refleksioner og refraktioner af lys, som ikke passer ind i den proces (graphics pipeline\cite{rastarization}, som de enkelte grafikkort danner billeder ud fra. 

\paragraph{Ray tracing}\cite{raytracing_for_begyndere} [DER MANGLER KILDER TIL PÅSTANDE I DETTE AFSNIT]() forsøger, nøjagtigt at simulere lys i et virtuelt miljø, i modsætning til rasterisering hvor hastighed er den primære faktor. Raytracing bygger fundamentalt på at følge lysstråler og bygge en model for hvordan lysstrålerne interagere med forskellige objekter og materialer. 

I forhold til rasterisering, tager det længere tid at tegne, men komplekse lysfænomener som refleksioner og lys forvrængninger igennem semitransparante medier som vand(kaldet refraktion) er simple at beskrive for en raytracing algoritme, som kan tegne disse med realistisk precision. Nogle fænomener som bløde skygger kan også beskrives men jo flere typer fænomener og jo større realisme der kræves des længere tid tager det at tegne et billede, men raytracing tillader stor fleksibilitet.
