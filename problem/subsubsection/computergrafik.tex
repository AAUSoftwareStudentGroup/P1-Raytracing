\subsubsection{Computergrafik}
\label{sec:computergrafik}
I computergrafik, er en 3D model, en beskrivelse af objekters form og materiale \cite{computergrafik_introduktion}. Computergrafiske metoder kan bruges til at simulere, hvordan lys interagerer med modellen og på den måde tegne et billede af modellen. Der eksisterer forskellige computergrafiske metoder, flere af hvilke, kan bruges sammen med andre for, at opnå et mere realistisk eller effektivt resultat. Der er allerede værktøjer, som kan visualisere produkter til salg på e-butikker som f.eks.\ Cylindo \cite{Cylindo}. Vi har ikke kendskab til at andre specialiserer sig, eller markedsfører sig på nuværende tidspunkt med deres kompetencer med fokus på visualisering af lampers belysning. Derfor er der herunder beskrivelse af to af de mest anvendte metoder inden for computergrafik.

\paragraph{Rasterisering}
er en metode til at visualisere miljøer med høj aktiv brugerinteraktion som f.eks.\ computerspil \cite{rastarization}. Metoden virker ved, rent matematisk, at projektere modellen på et billedplan som repræsenterer skærmen \cite{rastarization}. Fordelen ved rasterisering er, at disse projektioner, kan foretages meget hurtigt af computerens grafikkort, som er bygget specielt til formålet \cite{rastarization}. Ulempen ved rasterisering er, at mere avancerede lysfænomener ikke nødvendigvis, passer ind i den proces (graphics pipeline \cite{rastarization}), som de enkelte grafikkort danner billeder ud fra. 

\paragraph{Raytracing} forsøger nøjagtigt at simulere lys i et virtuelt miljø, i modsætning til rasterisering, hvor hastighed er den primære faktor. Raytracing bygger fundamentalt på at følge lysstråler og bygge en model for, hvordan lysstrålerne interagerer med forskellige objekter og materialer \cite{raytracing_for_begyndere}. 

På grund af den mere detaljerede simulering af forskellige lysfænomener, tager det, for raytracing, ofte længere tid at lave et billede af en 3D-model, end det gør ved rasterisering. Fordelen ved raytracing er, at den gør det muligt at beskrive mere komplekse lysfænomener, og kan dermed også opnå en mere realistisk præcision end rasterisering \cite{raytracingvsrastarizatioin}. Nogle fænomener som bløde skygger kan også beskrives \cite{softshadow}. Desto flere lysfænomener der kræves, des længere tid tager det oftest at rendere et billede, men raytracing tillader stor fleksibilitet.