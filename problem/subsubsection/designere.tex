\subsubsection{Designere}
Designere er interesseret i at deres design bliver solgt og er derfor sandsynligvis interesseret i et program, der kan hjælpe dem med at gøre deres design mere populært.

 
Vi har haft kontakt med to forskellige lampedesignere, danske Erik Mortensen, og svenske David Wahl fra IKEA. De sagde følgende:
\begin{center}
\textit{"I alle mine lamper er valg af lyskilde og placering sket på grundlag af test via prototyper. De fleste af mine lamper er prototyper"}.

\textit{"Jeg har i en del år arbejdet med lampedesign. og har derfor mest været optaget af armaturets/lampens skulpturelle udtryk, men da det jo er en lampe skal den selvfølgelig  også opfylde det belysningsmæssige."} Mailen kan ses i bilag \ref{sec:mailErik}.
\end{center}
og
\begin{center}
\textit{"Apart from hand sketching and physical prototypes, we use the 3D modeling application Solid Works in IKEA of Sweden. And for renderings we use either the built in renderer, or photo works, which is also part of solid works."} Mailen kan ses i bilag \ref{sec:mailDavid}.
\end{center}

Her er to eksempler på forskellige måder for designere, at visualisere en lampes lys på. Erik Mortensen benytter sig kun af prototyper, hvorudfra han kan se, hvordan lyset falder, hvorimod David Wahl, udover prototyper, også benytter sig af computerprogrammet Solid Works, til at visualisere lampens lys. 