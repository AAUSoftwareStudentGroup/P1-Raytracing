\subsubsection{Designere}
Designere er interesseret i at deres design bliver solgt og er derfor sandsynligvis interesseret i et program, der kan hjælpe dem med at gøre deres design mere populært.
 
Vi har haft kontakt med to forskellige lampedesignere, danske Erik Mortensen, og svenske David Wahl fra IKEA, for at undersøge hvilke værktøjer de bruger til visualisering, når de designer lamper. Erik Mortensen sagde følgende: 
\begin{center}
\textit{"I alle mine lamper er valg af lyskilde og placering sket på grundlag af test via prototyper. De fleste af mine lamper er prototyper"}.

\textit{"Jeg har i en del år arbejdet med lampedesign. og har derfor mest været optaget af armaturets/lampens skulpturelle udtryk, men da det jo er en lampe skal den selvfølgelig  også opfylde det belysningsmæssige."} Mailen kan ses i bilag \ref{sec:mailErik}.
\end{center}

På baggrund af dette kan man altså uddrage at Erik Mortensen hovedsagligt benytter sig af prototyper, til at visualisere lampens lys på.

David Wahl sagde:
\begin{center}
\textit{"Apart from hand sketching and physical prototypes, we use the 3D modeling application Solid Works in IKEA of Sweden. And for renderings we use either the built in renderer, or photo works, which is also part of solid works."} Mailen kan ses i bilag \ref{sec:mailDavid}.
\end{center}

David Wahl bruger altså, ligesom Erik Mortensen, prototyper, men derudover benytter han sig også af computerprogrammet Solid Works, som består af produkter til 3D-modellering, simulering og visualisering\cite{solidworks}.

Ud fra de to udtagelser fra Erik Mortensen og David Wahl, er der på den ene side \textit{prototyper}, som en teknik til at visualisere lampen, og på den anden side computersoftware, som f.eks\. Solid Works, som visualisere en 3D-model af lampen. Derfor tyder det ikke på at designere mangler værktøjer til visualisering, når de designer lampen, men at problemet om manglende visualisering af lamper og deres belysning må opstå et andet sted.

