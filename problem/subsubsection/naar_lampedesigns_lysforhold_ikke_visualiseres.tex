\subsubsection{Når lampedesigns lysforhold ikke visualiseres}
En lampes funktion er, at afgive lys som kan bruges i forhold til brugerens ønsker. Hvis lampen ikke lever op til disse ønsker eller krav kan den blive tilset som værende dårlig og irreterende. Hvis lampen laver mærkelige skygger eller ujævn belysning vil den oftest være dårlig at læse ved, og øjnene skal derfor kompensere hvilket kan medføre at man får ondt i hovedet\cite{lys_konsekvenser}.

Én ting er lampeskærmen, men lyskilden kan give lige så mange problemer:

Lysstofrør giver ofte et flimrende lys som man i længden kan få ondt i hovedet af, men er tilgengæld billige i drift\cite{videnskab_dk_led}. Glødepærer giver et lys der er næsten tilsvarende dagslys, men har en kortere levetid hvilket medfører at den er dyrere på længere sigt, f.eks.\ koster en glødepærer 10-15 kr og har en levetid på 1000 timer mens en LED koster 50-200 kr og har en levetid på 15000-50000 timer. Ud fra et eksempel på bolius.dk kan man spare 954 kr om året ved at skifte glødepærer ud\cite{pris_glodepoerer}. LED-pærer er en rimelig ny teknologi i lys-pære verdenen, men har potentiale, da de både kan farves, er billige i drift, billige i indkøb og holder meget længere end både glødepærer og lysstofrør\cite{videnskab_dk_led}.