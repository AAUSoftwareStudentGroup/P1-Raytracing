Objektet er det man kan se når billedet bliver renderet af raytraceren. I programmet er objektet beskrevet på følgende måde.

\begin{lstlisting}[style=Cstyle, caption=Structs til objektet]
typedef struct _verticie {
  Vector position;
  Vector normal;
} Vertex;

typedef struct _triangle {
  Vertex *verticies[3];
} Triangle;

typedef struct _object {
  Vertex *verticies;
  int n_verticies;
  Triangle *triangles;
  int n_triangles;
  Pixel color;
  Material material;
} Object;
\end{lstlisting}

Et objekt består af en række verticies (linje 11-12), en række trekanter (linje 13-14), samt en farve på objektet og koefficienterne til det materiale den er lavet af. En verticie består af en positionsvektor og en normalvektor (linje 2-3) og en trekant består af tre verticies (linje 7).