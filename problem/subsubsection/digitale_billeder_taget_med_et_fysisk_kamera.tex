\subsubsection{Digitale billeder taget med et fysisk kamera}
Som beskrevet under afsnit \ref{sec:ehandel}, benytter e-butikker, sig ofte af billeder til at vise kunden deres varer over internettet. Et eksempel på dette er vist på figur \ref{fig:e_handel_lampebilleder}.

\begin{figure}[H]
    \centering
    \fbox{\rule{\textwidth}{5cm}}
    \caption{Billeder af lamper på e-butikken somelampstore.what}
    \label{fig:e_handel_lampebilleder}
\end{figure} 

I det viste tilfælde er visualiseringen skabt ved at tage billeder af lamperne med et kamera fra en bestemt vinkel, i en kontekst, der typisk hænger sammen med lampetypen. 

Fordelen ved denne type af visualisering er, at den giver et virkelighedstro billede af, hvordan lampen ser ud i den kontekst, som billedet er taget i. Ulempen er, at der ofte kun er et begrænset antal billeder til rådighed, hvilket kan medføre, at forbrugeren ikke kan se lampen fra alle vinkler og på den måde ikke kan visualisere lampen for sig. Derudover kan det være svært, at se hvordan lyset udbreder sig fra lampen, da dette til dels afhænger af hvilken vinkel man ser lampen fra. 

Herudfra kan man kortfattet sige, at visualisering af lamper gennem billeder, taget med et fysisk kamera, giver et realistisk billede af lampen, men kun i den kontekst og vinkel billedet er taget i. 
