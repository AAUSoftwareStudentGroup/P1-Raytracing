\subsubsection{Digitale billeder taget med et fysisk kamera}
Som beskrevet under afsnit \ref{sec:ehandel}, benytter e-butikker, sig ofte af billeder til at vise kunden deres varer over internettet.

Fordelen ved denne type af visualisering er, at den giver et virkelighedstro billede af, hvordan lampen ser ud i den kontekst, som billedet er taget i. Ulempen er, at der ofte kun er et begrænset antal billeder til rådighed, hvilket kan medføre, at køberen ikke kan se lampen fra alle vinkler og på den måde ikke kan visualisere lampen for sig. Det er dog en mulighed at tage mange billeder, men med forskellige pæretyper, vinkler og lamper, giver dette hurtigt mange kombinationer, og dermed mange billeder, der skal tages, hvilket gør det til en tidkrævende proces. Manglende billeder kan gøre det svært for kunden, at se hvordan lyset udbreder sig fra lampen, da dette til dels afhænger af hvilken vinkel man ser lampen fra. 

Herudfra kan man kortfattet sige, at visualisering af lamper gennem billeder, taget med et fysisk kamera, giver et realistisk billede af lampen, men kun i den kontekst og vinkel billedet er taget i. 
