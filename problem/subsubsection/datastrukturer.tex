\subsubsection{Datastrukturer}
[Indledning: Forklar at vi bruger datastrukturer for at abstrahere over data.]
\paragraph{Vektor}
En stor del af opgaven bygger på vektorer i rummet. Nedenstående kodeuddrag viser hvordan der er abstraheret over en vektor i programmet. 

\begin{lstlisting}[style=Cstyle, caption=Struct til vektor]
typedef struct _vector {
    double x, y, z;
} Vector;
\end{lstlisting}

På linje 2 i ovenstående kodeuddrag er det vist hvordan en 3D vektors koordinarter er beskrevet med typen double. 

\paragraph{Stråle}
For at benytte sig af ray-tracingmetoden er det naturligvis nødvendigt at have en ray (stråle) som man kan trace (følge). 

\begin{lstlisting}[style=Cstyle, caption=Struct til ray]
typedef struct _ray {
  Vector initial_point, direction;
} Ray;
\end{lstlisting}

På linje to i ovenstående kodeuddrag kan det ses at en ray består af et startpunkt (initial_point) og en retning, begge af disse er af typen Vector, som tidligere er beskrevet til at have et x-, y- og z-koordinat.

\paragraph{Trekant}
Enhver figur i programmet består af mange sammensatte trekanter. I de fleste tilfælde er det så mange trekanter, at det ikke umiddelbart er synligt for det blotte øje. Et hjørne i en trekant udspændes af flere og kaldes en vertex.
    
\begin{lstlisting}[style=Cstyle, caption=Struct til vertex]
typedef struct _verticie {
  Vector position;
  Vector normal;
} Vertex;
\end{lstlisting}

På linje to og tre i ovenstående kodeuddrag kan det ses at en vertex består af et stedvektor, som er kaldet position, og en normalvektor.
    
Som nævnt for ovenstående struct for en vertex, er alle objekter i programmet bestående af trekanter. Nedenstående kodeuddrag viser hvordan trekanter er implementeret i programmet.
    
\begin{lstlisting}[style=Cstyle, caption=Struct til triangle]
typedef struct _triangle {
  Vertex *verticies[3];
  Vector edges[3];
} Triangle;
\end{lstlisting}

Denne struct viser, kort sagt, at en trekant består af tre hjørner, altså tre verticies, af typen Vertex, samt tre edges, der er stedvektorer til de tre verticies.

\paragraph{Point_lights}

En raytracers formål er at følge lysstrålerne, som udskydes fra et bestemt punkt, som i vores tilfælde er fra en pære. 

\begin{lstlisting}[style=Cstyle, caption=Struct til light]
typedef struct _pointlight {
  Vector position;
  Pixel color;
  double intensity;
  double radius;
  int sampling_rate;
} PointLight;
\end{lstlisting}

På linje to i ovenstående kodeuddrag kan det ses, at et lys har en given position, som bestemmes af ét 3D-koordinat. Dette lys har en farve, bestående af en RGB-værdi. Derudover har lyset også en intensitet og radius af typen double, samt sampling_rate af typen int.

\paragraph{Materiale}

I virkeligheden har vi mange forskellige materialer der både syner og føles anderledes end andre. En helt ny bil skinner når man lyser på den, mens en murstensvæg er helt mat. For at illustrere det i raytracing har vi brugt nogen værdier der ofte bruges i forbindelse med 3D-rendering og især raytracing. Vi har valgt at bruge Phong-modellen da den er relativ simpel og giver et godt resultat.

\begin{lstlisting}[style=Cstyle, caption=Typedefinition af Material]
typedef struct _material {
  double ambient_coefficient;
  double diffuse_coefficient;
  double specular_coefficient;
  int smoothness;
  double metalness; 
} Material;
\end{lstlisting}

Ambient_coefficient er værdien der fortæller hvor lyst objektet er selvom der ikke er direkte lys på den. Den bevirker at hvis objektet er i skygge så er det ikke helt sort.
Diffuse_coefficient er skygge-værdien der viser hvor stor indflydelse lys har på objektet. 
Specular_coefficient er værdien der viser hvor meget objektet spejler igen. Denne værdi er høj for en ny poleret bil, men næsten 0 hvis det er en murstensvæg. I vores program gør det at vi får en næsten hvid plet hvis vores lyskilde spejler igen.

\paragraph{AABB}
AABB står for axis aligned bounding boxes, og er en kasse hvis sidder er parallelle med akserne. En boks består af to vektorer som indeholder det laveste og største koordinatsæt for boksen. De to vektorer er en stedvektor til det hjørne på boksen som er tættest på og længst væk fra origo. Nedenstående kodeuddrag viser hvordan dette er gjort i programmet.

\begin{lstlisting}[style=Cstyle, caption=Struct til bounding boxes]
typedef struct _plane {
  Vector low, high;
\end{lstlisting}

De to punkter til boksen er angivet som stedvektorer af typen vector (linje 2).

\paragraph{Plan}
Et plan i rummet er beskrev ved et punkt og en normalvektor til planen. Hvordan dette er gjort i programmet er vist i nedenstående kodeuddrag. 

\begin{lstlisting}[style=Cstyle, caption=Struct til plan]
typedef struct _plane {
  Vector normal;
  Vector point;
\end{lstlisting}

Planens normalvektor er beskrevet med typen vector (linje to). Et punkt til planen er angivet som en stedvektor af typen vector (linje tre).

\paragraph{Skæring}
Det er nødvendigt at vide, hvis og hvor en lysstråle skærer et objekt, for at finde de data (farve, materiale, lokation m.m.) der er i det givne punkt ved skæringen.

\begin{lstlisting}[style=Cstyle, caption=Struct til intersection]
typedef struct _intersection {
  Vector normal;
  Material material;
  Pixel color;
  Triangle *triangle;
  Ray ray;
  double t;
} Intersection;
\end{lstlisting}

På ovenstående struct kan det ses, at hver intersection har en mængde data, som bruges senere til phong. Hver intersection har en 
normalvektor, et materiale, en farve beskrevet ved typen Pixel. Den har derudover en 'triangle' til at finde hvilken trekant i træet, som den skærer i, en 'ray', som beskriver lysstrålen, der skærer i det givne punkt samt en tid t af typen double, der beskriver afstanden fra initialpoint.

\paragraph{Image}

\paragraph{Kamera}
Som vist på figur \ref{...} i afsnit \ref{...} er kameraet beskrevet som dens position og orientering i rummet samt en afstand, d, til kameraet. Derudover har kameraet en højde og en bredde som definerer billedets opløsning. Nedenstående kodeuddrag viser hvordan dette er gjort i programmet.

\begin{lstlisting}[style=Cstyle, caption=Struct til kamera]
typedef struct _camera {
  Vector up;
  Vector right;
  Vector forward;
  Vector position;
  unsigned int width, height;
  double distance;
} Camera;
\end{lstlisting}

Kameraets orientering kan beskrives som tre vektorer af typen vector (linje 2-4). Kameraets position kan angives som en stedvektor af typen vector (linje 5). Opløsningen er angivet som positive heltal af typen unsigned int (linje 6). Afstanden til kameraet er angivet som typen double (linje 7).

\paragraph{Scene}
Scenen er den virtuelle verden, som skal visualiseres. I programmet er scenen beskrevet på følgende måde. 

\begin{lstlisting}[style=Cstyle, caption=Struct til scene]
typedef struct _scene {
  Object **objects;
  unsigned int n_objects;
  PointLight **lights;
  unsigned int n_lights;
  Pixel ambient_intensity;
} Scene; 
\end{lstlisting}

En scene består af en række objekter (linje 2-3), en række lys (linje 4-5), samt den ambiente lys intensitet der bruges i Phong.

\paragraph{Objekt}
Objektet er det man kan se når billedet bliver renderet af raytraceren. I programmet er objektet beskrevet på følgende måde.

\begin{lstlisting}[style=Cstyle, caption=Structs til objektet]
typedef struct _object {
  Vertex *verticies;
  int n_verticies;
  Triangle *triangles;
  int n_triangles;
  Pixel color;
  Material material;
} Object;
\end{lstlisting}

Et objekt består af en række verticies (linje 2-3), en række trekanter (linje 4-5), samt en farve på objektet og koefficienterne til det materiale den er lavet af.
\paragraph{K-dimensionalt træ}
Da programmet er bygget op efter at skulle kunne implementeres på en hjemmeside, er renderingen nødt til at være hurtigt. Dette opnås ved hjælp af optimering, hvor der i programmet er brugt KD-træer.

\begin{lstlisting}[style=Cstyle, caption=Struct til KDNode]
typedef struct _KDNode {
  struct _KDNode *low, *high;
  AABB box;
  Triangle **triangles;
  int n_triangles;
} KDNode;
\end{lstlisting}

Et KD-træ består af en række nodes (linje 2), hvilket er forgreninger, den består af en AABB box (linje 3) og en række trekanter (linje 4-5).