\subsubsection{Sammenligning af detail- og e-handel}
Ud fra ovenstående redegørelse af de to typer for handel, analyseres disse nu med henblik på at finde ligheder og forskelle, hvoraf det kan afgøres i hvilken af de to typer af handel, at problemet er størst. 

Da det initierende problem er, at kunden ikke kan visualisere lampen uden at købe den, er det derfor relevant, at se på i hvor høj grad dette er tilfældet ved de to typer handler.

Fordelen ved detailhandel er, at kunden ofte kan se lampen i butikken, og ud fra dette, vurdere hvilken lampe der opfylder de behov som kunden har \ref{sec:sammenligning_af_e_og_d}. Dog er problemet stadig, at kunden ikke ser lampen i den rette kontekst, dvs.\ i sit eget hjem. Dette kan gøre, at kunden får et godt indtryk af lampen i den kontekst, som butikken præsenterer den i, men at den ikke passer ind i den kontekst, som kunden køber lampen til.

Ved e-handel har kunden ikke muligheden for, at se en fysisk udgave af lampen, men ofte kun billeder. Dette gør at kunden alene kan tage valg ud fra de billeder og informationer som e-butikken præsenterer. Problemet er så, at billederne til dels ikke er interaktive, dvs.\ brugeren ikke kan se lampen fra flere vinkler end dem som billederne er taget i, samt at billederne ikke er taget af lampen i den kontekst, som kunden ønsker at købe lampen til. 

Med hensyn til konteksten er fordelen ved e-handel, at kunden kan sidde derhjemme, i den kontekst, hvor lampen skal indgå, og sammenligne med de informationer, der er tilgængelige på e-butikken. I modsætning til dette er detailbutikker, hvor kunden står i butikken, og måske har problemer med at huske eller blot forestille sig alle detaljerne ved den kontekst, som lampen skal indgå i.

Ud fra denne sammenligning, er der på den ene side detailhandel, hvor det er svært at visualisere konteksten, men hvor man kan se lampen. På den anden side er e-handel, hvor man kan sidde derhjemme i konteksten, men har svært ved at visualisere lampen. 

For at afgøre hvilken type handel denne rapport vil fokusere på, skal der derfor svares på om det er mangel på visualisering af lampen i kontekst ved detailhandel eller mangel på visualisering af lampen ved e-handel, som er det største problem.

Da kunden omgås og ser den kontekst, som lampen skal indgå i f.eks.\ et kontor, køkken og bad, må man kunne antage at kunden har en forestilling om, hvordan denne kontekst ser ud selvom kunden ikke står i den, når der handles i en fysisk butik. Derfor er dette ikke et lige så stort problem, som hvis kunden ikke kan visualisere lampen når der handles via e-handel. Derfor vil fokuset i denne rapport være at forbedre kundens evne til at visualisere lamper under e-handel.