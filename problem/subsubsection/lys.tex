\subsubsection{Lys}
\label{sec:lys}
Formålet med dette afsnit er at forsøge at definere lys, og beskrive hvilken type af lys, som rapporten vil tage udgangspunkt i. Derudover redegøres der for farvetemperatur samt den optimale placering af lyset ift.\ anvendelsen.


Der er forskellige opfattelser af hvad lys indebærer. Hvis vi tager udgangspunkt i Karsten Rottwitt, som er professor ved DTU fotonik, definerer han lys som:


\textit{"Lys er andet end synligt lys. For mig er lys et elektromagnetisk felt, som har en høj frekvens"
- Karsten Rottwitt \cite{def_lys}.}

Han mener også, at der er en hårfin grænse for hvornår lys kan betegnes som lys, denne grænse er dog først i spil når vi snakker om UV-lys og infrarødt lys \cite{def_lys}. 
Andre er ikke enige med Karsten Rottwitt om hvordan lys defineres. Tager vi nu udgangspunkt i Britannica \cite{britannica_lys}, så betegnes lys, som magnetiske stråler, som det menneskelige øje kan opfange, også kaldet synligt lys. 


Det er denne definition, som rapporten vil tage udgangspunkt i. Dette er valgt, da det er oplagt at kombinere synligt lys og lamper.

\paragraph{Farvetemperatur}
Det lys, som kommer fra en lampe, kan have forskellige farvetemperaturer. Ordet beskriver hvilken farve lyset fra et sort legeme, med en bestemt temperatur vil have \cite{farvetemp}. Dette er f.eks.\ om det er varmt eller koldt. Farvetemperaturen måles i kelvin, og der er forskel på, hvor man bør benytte pærer med forskellige farvetemperaturer. Integral-LED er et firma med over 25 års erfaring \cite{integral_led}, som har opstillet nogle foretrukne steder at bruge de forskellige typer af lys:

\begin{enumerate}
\item Varm(2700 Kelvin) til varm hvid(3000 Kelvin): anbefaler de at bruge steder som i stuen, soveværelset og i entréen \cite{varm_kold}.
\item Hvid(4000 Kelvin) til kold hvid(5000 Kelvin): anbefaler de at bruge steder som i køkkenet, kontoret, badeværelset og i større skabe \cite{varm_kold}.
\end{enumerate}

Denne rapport vil altså tage udgangspunkt i synligt lys, hvor farven af dette lys kan beskrives med en farvetemperatur. Ud fra ovenstående er det nu også beskrevet hvordan lysets farvetemperatur er en faktor, når der skal afgøres hvilket lys der passer bedst ind i et bestemt rum.
