\subsubsection{Krav til løsningen}
\label{sec:krav}

I forbindelse med vores projekt har vi fået nogle krav fra universitetets side. Disse er, at programmet skal skrives i programmeringssproget C, derudover er der også tidsmæssigt pres da hele projektet kun varer ca.\ to og en halv måned (Svarende til 10 ECTS). 
Vi er også begrænset af vores egen viden indenfor emnet, da raytracing er en fremmed teknik, som kun få af os har haft tidligere erfaringer med. 

Ud fra afsnittet \ref{sec:skitse_loesning} og med baggrund i underspørgsmålene i problemformuleringen (afsnit \ref{sec:problemformulering}) har gruppen opstillet følgende krav til løsningen:
\begin{enumerate}
    \item Løsningen skal gøre det muligt at vise et billede af en lampe og dens belysning.
    \item Det skal være muligt at ændre, hvilken synsvinkel lampen ses fra.
    \item Det skal være muligt at bestemme hvilken farvetemperatur, som pæren i lampen visualiseres ud fra.
    \item Løsningen skal fremskaffe et billede på en tid så den er praktisk anvendelig.
    \item Løsningen skal kunne implementeres sådan at billederne af lampen kan vises for kunderne på en e-butiks hjemmeside.
\end{enumerate}

Kravene hører alle under det overordnede spørgsmål i problemformulering i afsnit \ref{sec:problemformulering}, men hvor ovenstående krav 1-3 specifikt knytter sig til hhv.\ underspørgsmål 1-3 i problemformulering.