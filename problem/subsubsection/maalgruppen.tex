\subsubsection{Målgruppen}
Som beskrevet i forrige afsnit, er det både designere, producenter, butikker, kunder og brugere, der påvirkes af problemet. Det er nu relevant at afgøre hvem problemløsningen retter sig mod, da dette danner grundlag for, hvordan løsningen skal udvikles og hvem der kan indrages i løsningen og udarbejdelsen af løsningsforslaget.

Som illustreret på \ref{fig:interessenter} er det lampebutikkerne, som har den direkte kontakt til kunderne og via kunderne en forbindelse til brugerne. Designere er fravalgt, da vi ud fra mails fra Wahl og Mortensen kan uddrage, at de allerede har værktøjer til at visualisere lys fra lamper, og som der ses på skitsen har designeren ikke nogle direkte kontakt til kunden eller brugeren. Man kunne forestille sig, at problemet kunne løses allerede fra producentens side. Ud fra korrespondance med en belysningskonsulent i en lampebutik er dette dog blevet afvist \ref{sec:mailbelysning}. Vi er blevet informeret om, at producenterne ikke er interesseret i at bruge ressourcer på at løse problemet, da det ikke gavner producenterne direkte. Derudover er det ikke producentens opgave at vejlede kunder til det bedste køb af lamper, dette er derimod lampebutikken opgave.

Hvis man retter problemløsningen mod lampebutikker, og laver en løsning der gør det muligt for kunder at visualisere lamperne bedre, er det sandsynligt at kunderne vil være mere tilfredse med deres lamper, da de har mulighed for at se lampens belysning inden købet. For lampebutikker kan dette betyde, at kunden ikke returnerer lige så mange lamper, og dette vil bidrage til øget kundetilfredshed, som i sidste ende gavner både lampebutikker og kunderne. Hvis kunden er i stand til at købe en lampe som passer ind i den korrekte kontekst vil dette tilfredsstille brugeren. 