\subsubsection{Kildehenvisning}
Rapportens brug af kildehenvisninger er baseret på nummermetoden. I nummermetoden anføres kilderne i fortløbende nummerorden, svarende til hvilket nummer, de har i teksten. To identiske kilder har samme nummer. Herunder ses et eksempel på hhv. en internetkildehenvisning og en artikel- eller bogkildehenvisning:

Interneteksempel med kilde[1].

[1] Titel på emne eller kort forklaring på emnet, hjemmesidenavn. Set DD-MM-YYYY. URL på hjemmeside.


Bogeksempel med kilde[2].

[2] Titel på bog, udgavenummer, forfatter(e), udgivelsesår. ISBN/ISSN-nummer.


Hvis en kilde har yderligere relevante informationer (såsom sidetal, copyright mm. angives disse også i kilden.


Figurhenvisning foregår på samme måde som med andre kilder, dog med en forklaring under selve figuren. Hvis en figur ingen kilde har, er figuren fremstillet af gruppen.
