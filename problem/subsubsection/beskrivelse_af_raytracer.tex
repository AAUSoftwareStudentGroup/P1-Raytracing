\subsubsection{Beskrivelse af raytracer}
I dette afsnit vil de væsentlige funktioner til raytraceren blive beskrevet. Formålet med afsnittet er, at få en forståelse for hvordan den tidligere beskrevet teori, \ref{sec:teori}, er anvendt og håndteret i programudviklingen.


\paragraph{Rendering}
For at rendere et billede af lampen og dens belysning, er der lavet en funktion raytracer\_render, der modtager scenen, dvs. samlingen af alle 3D-objekter og lys, samt modellen for det kamera, som billedet skal dannes ud fra. Funktionen er vist herunder.

\begin{lstlisting}[style=Cstyle, caption=Funktionen der rendere billedet af scenen med et kameras perspektiv]
Image *raytracer_render(Scene *scene, Camera *camera) {
  int x, y;
  Image *image;
  Ray ray;

  image = new_image(camera->width, camera->height);
  
  /* For each column: */
  for(x = 0; x < camera->width; x++) {
    /* For each pixel in column: */
    for(y = 0; y < camera->height; y++) {
      /* Calculate ray */
      ray = raytracer_calculate_ray(x, y, camera);
      
      /* Trace ray and assign result to pixel */
      image->pixels[x][y] = raytracer_trace(ray, scene);
    }
    printf("%.1f\n", ((double)x + 1) / camera->width * 100);
  }

  return image;
}
\end{lstlisting}

Funktionen vist herover, danner en lysstråle for hver pixel i billedet. lysstråle sendes videre sammen med scenen til funktionen raytracer\_trace, som returnere hvilken farve den pågældende pixel på billedet skal have. Til sidst returneres så det endelige billede.

\paragraph{Tracer}
Funktionen raytracer\_trace er den funktion som starter raytraceren samt returnerer en pixelfarve hvis en ray skærer med et objekt i scenen. Funktionen er vist herunder.  

\begin{lstlisting} [style=Cstyle, caption=Funktionen raytracer\_trace]
Pixel raytracer_trace(Ray ray, Scene *scene) {
  Intersection intersection = create_intersection();
  Pixel pixel = {0, 0, 0};
  
  /* If ray intersects with scene: */
  if(raytracer_scene_intersection(ray, scene, &intersection)) {
    /* Shade pixel */
    pixel = raytracer_phong(intersection, scene);
  }
  
  return pixel;
}
\end{lstlisting}
Funktionen initialiserer en skæring med værdien -1 ved at kalde funktionen create\_intersection (linje 2), og en pixel initialiseres til at indeholde RGB-værdien for farven sort (linje 3). Der checkes efterfølgende om den pågældende ray skærer med et objekt i scenen (linje 6), hvis den gør det så tildeles der en RGB-værdi (linje 8), som returneres til sidst i funktionen. 

\paragraph{Skæring med scene}

Funktionen raytracer\_scene\_intersection tjekker om en ray skærer med et objekt.

\begin{lstlisting}[style=Cstyle, caption=Funktionen raytracer\_scene\_intersection]
int raytracer_scene_intersection(Ray ray, Scene *scene, 
                                 Intersection *intersection) {
  int i;
  Intersection temporary_intersection;

  temporary_intersection = create_intersection();

  /* For each object in scene: */
  for(i = 0; i < scene->n_objects; i++) {
    /* If ray intersects with object: */
    if(raytracer_object_intersection(ray, scene->objects[i], 
       &temporary_intersection))
      /* Reassign intersection if current intersection is closer */
      if(temporary_intersection.t < intersection->t || intersection->t == -1)
        *intersection = temporary_intersection;
  }
  return intersection->t > 0;
}
\end{lstlisting}

If-sætningen på linje 11 assigner temporary\_intersection til en intersection hvis rayen rammer et objekt. If-sætningen på linje 14 sørger for at det er den intersection der er tættest på der bliver sendt tilbage. 


\paragraph{Skæring med objekt}

Funktionen raytracer\_object\_intersection er en funktion, der undersøger om der er en skæring mellem lysstrålen og objektet. Funktionen ses herunder:

\begin{lstlisting}[style=Cstyle, caption=raytracer\_object\_intersection]
int raytracer_object_intersection(Ray ray, Object *object, Intersection *intersection) {
  double i, j;
  
  /* if ray intersects with object's aabb: */
  if(intersection_ray_aabb(ray, object->root.box, &i, &j) && 
     raytracer_kdtree_intersection(ray, &(object->root), intersection)) {
    intersection->color = object->color;
    intersection->material = object->material;
  }
  return intersection->t > 0;
}
\end{lstlisting}

På linje fem i ovenstående kodeuddrag kan man se at funktionen tjekker først om lysstrålen skærer med aabb, og hvis dette er sandt tjekker den så om lysstrålen skærer i træet. Hvis begge af disse tilfælde er sande, får pixlen i skæringspunktet en farve bestemt af, hvor og hvordan lysstrålen rammer objektet, samt bliver værdierne af materialet gemt, som kan bruges senere. Herefter returnerer funktionen true, hvis tiden, som beskriver afstanden fra initialpoint til skæringspunktet, er over 0, og false hvis den ikke er.

\paragraph{Skæring med KD-træ}

\paragraph{Skæring med trekant}

\paragraph{Phong pixel farve}



