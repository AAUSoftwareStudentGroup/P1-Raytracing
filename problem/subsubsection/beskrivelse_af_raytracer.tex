\subsubsection{Beskrivelse af raytracer}
[Indledning]
\paragraph{Rendering}

\paragraph{Tracer}
[Indledning: Nævn funktionen der omtales, og beskriv med én sætning, kort og præcist hvad funktione gør]

[Indsæt funktionen hvis den fylder mindre end ca. en halv side, ellers henvis til bilag]

[Beskriv det væsentlige i funktionen ved at henvise til listingen ovenover, ligesom i under datastrukturer]
 
\paragraph{Skæring med scene}

Funktionen raytracer\_scene\_intersection tjekker om en ray skærer med et objekt.

\begin{lstlisting}[style=Cstyle, caption=Structs til objektet]
int raytracer_scene_intersection(Ray ray, Scene *scene, 
                                 Intersection *intersection) {
  int i;
  Intersection temporary_intersection;

  temporary_intersection = create_intersection();

  /* For each object in scene: */
  for(i = 0; i < scene->n_objects; i++) {
    /* If ray intersects with object: */
    if(raytracer_object_intersection(ray, scene->objects[i], 
       &temporary_intersection))
      /* Reassign intersection if current intersection is closer */
      if(temporary_intersection.t < intersection->t || intersection->t == -1)
        *intersection = temporary_intersection;
  }
  return intersection->t > 0;
}
\end{lstlisting}

If-sætningen på linje 11 assigner temporary\_intersection til en intersection hvis rayen rammer et objekt. If-sætningen på linje 14 sørger for at det er den intersection der er tættest på der bliver sendt tilbage. 


\paragraph{Skæring med objekt}

\paragraph{Skæring med KD-træ}

\paragraph{Skæring med trekant}

\paragraph{Phong pixel farve}



