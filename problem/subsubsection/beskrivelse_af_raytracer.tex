\subsubsection{Beskrivelse af raytracer}
[Indledning]
\paragraph{Rendering}
For at rendere et billede af lampen og dens belysning, er der lavet en funktion raytracer\_render, der modtager scenen, dvs. samlingen af alle 3D-objekter og lys, samt modellen for det kamera, som billedet skal dannes ud fra. Funktionen er vist herunder.

\begin{lstlisting}[style=Cstyle, caption=Funktionen der rendere billedet af scenen med et kameras perspektiv]
Image *raytracer_render(Scene *scene, Camera *camera) {
  int x, y;
  Image *image;
  Ray ray;

  image = new_image(camera->width, camera->height);
  
  /* For each column: */
  for(x = 0; x < camera->width; x++) {
    /* For each pixel in column: */
    for(y = 0; y < camera->height; y++) {
      /* Calculate ray */
      ray = raytracer_calculate_ray(x, y, camera);
      
      /* Trace ray and assign result to pixel */
      image->pixels[x][y] = raytracer_trace(ray, scene);
    }
    printf("%.1f\n", ((double)x + 1) / camera->width * 100);
  }

  return image;
}
\end{lstlisting}

Funktionen vist herover, danner en lysstråle for hver pixel i billedet. lysstråle sendes videre sammen med scenen til funktionen raytracer\_trace, som returnere hvilken farve den pågældende pixel på billedet skal have. Til sidst returneres så det endelige billede.

\paragraph{Tracer}
[Indledning: Nævn funktionen der omtales, og beskriv med én sætning, kort og præcist hvad funktione gør]

[Indsæt funktionen hvis den fylder mindre end ca. en halv side, ellers henvis til bilag]

[Beskriv det væsentlige i funktionen ved at henvise til listingen ovenover, ligesom i under datastrukturer]
 
\paragraph{Skæring med scene}

\paragraph{Skæring med objekt}

\paragraph{Skæring med KD-træ}

\paragraph{Skæring med trekant}

\paragraph{Phong pixel farve}



