\subsubsection{Beskrivelse af raytracer}
I dette afsnit vil de væsentlige funktioner til raytraceren blive beskrevet. Formålet med afsnittet er, at få en forståelse for hvordan teorien i afsnit \ref{sec:teori}, er anvendt og håndteret i programudviklingen.


\paragraph{Rendering}
For at rendere et billede af lampen og dens belysning, er der lavet en funktion raytracer\_render, der modtager scenen, dvs.\ samlingen af alle 3D-objekter og lys, samt modellen for det kamera, som billedet skal dannes ud fra. Funktionen er vist herunder.

\begin{lstlisting}[style=Cstyle, caption=Funktionen, der rendere billedet af scenen med et kameras perspektiv]
Image *raytracer_render(Scene *scene, Camera *camera) {
  int x, y;
  Image *image;
  Ray ray;

  image = new_image(camera->width, camera->height);
  
  /* For each column: */
  for(x = 0; x < camera->width; x++) {
    /* For each pixel in column: */
    for(y = 0; y < camera->height; y++) {
      /* Calculate ray */
      ray = raytracer_calculate_ray(x, y, camera);
      
      /* Trace ray and assign result to pixel */
      image->pixels[x][y] = raytracer_trace(ray, scene);
    }
    printf("%.1f\n", ((double)x + 1) / camera->width * 100);
  }

  return image;
}
\end{lstlisting}

Funktionen vist herover, danner en ray for hver pixel i billedet. Rayen sendes videre sammen med scenen til funktionen raytracer\_trace, som returnerer, hvilken farve den pågældende pixel på billedet skal have. Til sidst returneres det endelige billede.

\paragraph{Tracer}
Funktionen raytracer\_trace er den funktion som starter raytraceren samt returnerer en pixelfarve, hvis en ray skærer med et objekt i scenen. Funktionen er vist herunder.  

\begin{lstlisting} [style=Cstyle, caption=Funktionen raytracer\_trace]
Pixel raytracer_trace(Ray ray, Scene *scene) {
  Intersection intersection = create_intersection();
  Pixel pixel = {0, 0, 0};
  
  /* If ray intersects with scene: */
  if(raytracer_scene_intersection(ray, scene, &intersection)) {
    /* Shade pixel */
    pixel = raytracer_phong(intersection, scene);
  }
  
  return pixel;
}
\end{lstlisting}
Funktionen initialiserer en skæring med værdien -1 ved at kalde funktionen create\_intersection (linje 2), og en pixel initialiseres til at indeholde RGB-værdien for farven sort (linje 3). Der tjekkes efterfølgende om den pågældende ray skærer med et objekt i scenen (linje 6), hvis den gør det så tildeles der en RGB-værdi (linje 8), som returneres til sidst i funktionen. 

\paragraph{Skæring med scene}

Funktionen raytracer\_scene\_intersection tjekker om en ray skærer med et objekt.

\begin{lstlisting}[style=Cstyle, caption=Funktionen raytracer\_scene\_intersection]
int raytracer_scene_intersection(Ray ray, Scene *scene, 
                                 Intersection *intersection) {
  int i;
  Intersection temporary_intersection;

  temporary_intersection = create_intersection();

  /* For each object in scene: */
  for(i = 0; i < scene->n_objects; i++) {
    /* If ray intersects with object: */
    if(raytracer_object_intersection(ray, scene->objects[i], 
       &temporary_intersection))
      /* Reassign intersection if current intersection is closer */
      if(temporary_intersection.t < intersection->t || intersection->t == -1)
        *intersection = temporary_intersection;
  }
  return intersection->t > 0;
}
\end{lstlisting}

I funktionen tjekkes der om en ray skærer et objekt, hvis dette er tilfældet så returneres der information om denne skæring (linje 11). Der tjekkes efterfølgende om den nye skæring er tættere på udgangspunktet end de førhenværende skæringer og den tætteste skæring gemmes (linje 14 - 15). Herefter returnerer funktionen sandt, hvis tiden, som beskriver afstanden fra startpunktet  til skæringspunktet, er over 0, og falsk hvis den ikke er.


\paragraph{Skæring med objekt}

Funktionen raytracer\_object\_intersection er en funktion, der undersøger om der er en skæring mellem en ray og et objekt. Funktionen ses herunder:

\begin{lstlisting}[style=Cstyle, caption=raytracer\_object\_intersection]
int raytracer_object_intersection(Ray ray, Object *object, Intersection *intersection) {
  double i, j;
  
  /* if ray intersects with object's aabb: */
  if(intersection_ray_aabb(ray, object->root.box, &i, &j) && 
     raytracer_kdtree_intersection(ray, &(object->root), intersection)) {
    intersection->color = object->color;
    intersection->material = object->material;
  }
  return intersection->t > 0;
}
\end{lstlisting}

På linje fem i ovenstående kodeuddrag kan man se, at funktionen først tjekker om en ray skærer med AABB, og hvis dette er sandt tjekker den om rayen skærer i træet. Hvis begge af disse tilfælde er sande, får pixlen i skæringspunktet en farve bestemt af, hvor og hvordan rayen rammer objektet. Derudover bliver værdierne af materialet gemt, som kan bruges senere. Herefter returnerer funktionen sandt, hvis tiden, som beskriver afstanden fra startpunktet til skæringspunktet, er over 0, og falsk hvis den ikke er.


\paragraph{Skæring med kd-træ}

Funktionen raytracer\_kdtree\_intersection bruges til at undersøge om rayen skærer med kd-træet og i så fald hvor. Formålet med denne funktion er at optimere programmet, og undgå at lave unødvendige udregninger. 

Hele funktionen kan findes på elektronisk bilag i filen kdnode.c

\begin{lstlisting}[style=Cstyle, caption=if(leaf)]
  if(kdnode_is_leaf(node)) {
    /* test intersection with triangles */
    for(i = 0; i < node->n_triangles; i++) {
      if(raytracer_triangle_intersection(ray, node->triangles[i], &temporary_intersection)) {
        /* if first intersection or closer than current intersection: */
        if(temporary_intersection.t < intersection->t || intersection->t == -1) {
          *intersection = temporary_intersection;
          intersection->triangle = node->triangles[i];
        }
      }
    }
\end{lstlisting}

På ovenstående kodeuddrag undersøger funktionen, om skæringen med træet sker i et af bladene, dette ses på linje 1. Et blad er 'bunden' af træet, altså det punkt, hvor en kasse ikke længere kan inddeles i flere kasser. Der er tre krav til, hvornår vi ender vores inddeling af flere kasser, og istedet slutter og kalder det for et blad. Første krav er, at hvis antallet af trekanter i en kasse er mindre end eller lig med to er noden et blad. En kasse med to eller færre trekanter er nemt og hurtigt for computeren at udregne, og derfor behøves den ikke at inddeles yderligere. Det andet krav er, hvis over halvdelen af to kassers trekanter ligger i begge kasser. Hvis dette er tilfældet kaldes begge kasser for blade, da mindst halvdelen af begge kassers fælles trekanter er i begge kasser og der derfor ikke er brug for at inddele i flere yderligere kasser. Det tredje, og sidste, krav er, hvis der er mere end 30 niveauer. Dette krav er nødvendigt for at programmet ikke kører i en evighedsløkke, og bliver ved med at tjekke, selvom det ikke er nødvendigt. En tilfælde, hvor det tredje punkt kunne være nødvendigt er, hvis ét sted i kassen indeholder 10 punkter, og dermed ikke kan opdeles, så ville programmet uden tredje krav blive ved med at lave den samme opdeling igen og igen.


Hvis roden (hovedkassen) er et blad, findes de nærmeste skæringer først som vist på linjerne 6-8, og der findes skæring med samtlige trekanter som vist på linje tre, medmindre nogle trekanter ligger helt i skygge for andre. 

\begin{lstlisting}[style=Cstyle, caption=else (recursive)]
else {
    /* test intersection recursively */
    retl = intersection_ray_aabb(ray, node->low->box, &t_minl, &t_maxl);
    reth = intersection_ray_aabb(ray, node->high->box, &t_minh, &t_maxh);

    if(retl && reth) { /* Intersecting both sub-nodes */
      if(t_minh < t_minl) { /* low node is hit first */
        if(!raytracer_kdtree_intersection(ray, node->high, intersection)) {
          raytracer_kdtree_intersection(ray, node->low, intersection);
        }
      } else if(t_minh == t_minl) {
        raytracer_kdtree_intersection(ray, node->low, intersection);
        raytracer_kdtree_intersection(ray, node->high, intersection);
      } else { /* high node is hit first */
        if(!raytracer_kdtree_intersection(ray, node->low, intersection)) {
          raytracer_kdtree_intersection(ray, node->high, intersection);
        }
      }
    } else if(retl) { /* only low node is hit */
        raytracer_kdtree_intersection(ray, node->low, intersection);
    } else { /* only high node is hit */
        raytracer_kdtree_intersection(ray, node->high, intersection);
    }
  }
  return intersection->t > 0;
}
\end{lstlisting}

Hvis roden ikke er et blad og hvis ray skærer i begge kasser, undersøges hvilken kasse, der er nærmest og funktionen kaldes rekursivt for denne kasse, dette kan ses på linje 6-18 i ovenstående kodeuddrag. Hvis rayen går gennem den nærmeste kasse uden at ramme et objekt tjekkes for den fjerneste kasse. Hvis rayen kun skærer én af kasserne, tjekkes simpelt kun for den kasse, som set på linje 19-24.

Funktionen returnerer afstanden i tid, til skæringen med trekanten.

\paragraph{Skæring med trekant}

Funktionen raytracer\_triangle\_intersection bruges til at undersøge om der er skæring mellem en ray og en trekant. Funktionen ses herunder

\begin{lstlisting}[style=Cstyle, caption=Funktionen der finder skæring med trekant]
int raytracer_triangle_intersection(Ray ray, Triangle *triangle, Intersection *intersection) {
  double denominator, time;
  Plane plane;
  Vector triangle_normal;
  time = -1;

  triangle_normal = vector_normalize(vector_cross(triangle->edges[0], 
                                                  triangle->edges[1]));

  plane = create_plane(triangle->verticies[0]->position, triangle_normal);

  // If triangle on front of camera: check if point is inside triangle
  if(intersection_ray_plane(ray, plane, &time) && time > 0) {
    int i;
    Vector p = ray_get_point(ray, time);
    for(i = 0; i < VERTICES_IN_TRIANGLE; i++)
      if(vector_dot(triangle_normal, vector_cross(triangle->edges[i], 
         vector_subtract(p, triangle->verticies[i]->position))) < 0)
        return 0;
    intersection->t = time;
    intersection->ray = ray;
    if(vector_dot(ray.direction, triangle_normal) > 0)
      triangle_normal = vector_scale(triangle_normal, -1);
    intersection->normal = triangle_normal;
    return 1;
  }

  return 0;
}
\end{lstlisting}

Funktionen finder en normaliseret normalvektor ud fra to vektorer, som beskriver siderne i trekanten (linje 7). Derefter findes trekantens plan ud fra en stedvektor og en normalvektor (linje 10). Der undersøges nu, om der er skæring mellem trekant og plan (linje 13), hvis der er skæring tjekkes der om punktet ligger i trekanten (linje 13 - 22). Hvis punktet ligger indenfor trekanten (linje 22) skaleres der med -1 for, at få normalvektoren til trekanten til at vende ud mod kameraet (linje 23) og værdien gemmes i en output parameter (linje 24).

\paragraph{Phong pixelfarve}
Når skæringen er fundet ved ovenstående funktioner, returneres skæringen og tilhørende data gemt i en Intersection til funktionen raytracer\_trace, som kalder raytracer\_phong med scenen og en Intersection, som parametre. Prototypen for funktionen er vist herunder.

\begin{lstlisting}[style=Cstyle, caption=prototypen til funktionen der beregner pixelfarven på baggrund af data fra skæring med scenen.]
Pixel raytracer_phong(Intersection intersection, Scene *scene)
\end{lstlisting}

I funktionen raytracer\_phong initialiseres først de variable til phong-modellen, som er beskrevet i afsnit
 \ref{sec:fra_model_til_billede}.

\begin{lstlisting}[style=Cstyle, caption=Initialisering af variabler i raytracer\_phong.] 
m_a = intersection.material.ambient_coefficient;
m_l = intersection.material.diffuse_coefficient;
m_s = intersection.material.specular_coefficient;
m_sp = intersection.material.smoothness;
m_sm = intersection.material.metalness;
vN = intersection.normal;
pC = intersection.color;
pA = scene->ambient_intensity;
diffuse = create_pixel(0.0, 0.0, 0.0);
specular = create_pixel(0.0, 0.0, 0.0);
\end{lstlisting}

Herefter beregnes \textit{ambient} lys efter formlen i udtryk \ref{eq:ambient_formel} under afsnit \ref{sec:fra_model_til_billede}.

\begin{lstlisting}[style=Cstyle, caption=Beregning af ambient lys i raytracer\_phong.] 
ambient = pixel_multiply(pixel_scale(pC, m_a), pA);
\end{lstlisting}

Efter det \textit{ambient} lys, beregnes vektorer til beregning af \textit{diffuse} og \textit{specular} lys.

\begin{lstlisting}[style=Cstyle, caption={Beregning af skæring, vektor $\protect\vv{U}$ og pixel S i raytracer\_phong.}] 
intersection_point = ray_get_point(intersection.ray, intersection.t);
vU = vector_scale(intersection.ray.direction, -1.0);
pS = pixel_add(pixel_scale(pC, m_sm), pixel_scale(create_pixel(1.0, 1.0, 1.0), (1 - m_sm)));
\end{lstlisting}

Via en for-løkke summeres \textit{diffuse} og \textit{specular} lys fra alle lys i scenen, som vist herunder.

\begin{lstlisting}[style=Cstyle, caption=Beregning og summering af diffuse og specular lys fra scenens lys., label=summmering_diffuse_specular]
for(i = 0; i < scene->n_lights; i++) {
  sampled_light_intensity = raytracer_get_light_intensity(scene->lights[i], intersection_point, scene);

  pI = pixel_scale(scene->lights[i]->color, sampled_light_intensity);
  vI = vector_normalize(vector_subtract(scene->lights[i]->position,
                          intersection_point));
  vR = vector_normalize(vector_add(vector_scale(vI, -1),
                        vector_scale(vN, vector_dot(vI, vN) * 2)));

  /* diffuse light =  m_l * MAX(vI * vN, 0) * pC * pI*/
  diffuse = pixel_add(diffuse, pixel_multiply(pixel_scale(pC,
                      m_l * MAX(vector_dot(vI, vN), 0)), pI));

  /* specular light = m_s * MAX(-vR * vU, 0) ^ m_sp * pI * pS */
  specular = pixel_add(specular, pixel_multiply(pS, pixel_scale(pI,
                         m_s * pow(MAX(vector_dot(vR, vU), 0), m_sp))));
}
\end{lstlisting}

I kodeuddrag \ref{summmering_diffuse_specular} linje 2 indlæses intensiteten fra den pågældende lyskilde. Lysintensiteten beregnes i funktionen raytracer\_get\_light\_intensity, ved at teste, hvor stor en procentdel af et bestemt antal tilfældige rays fra lyskilden, som ikke afskærmes på vej mod skæringspunktet ved objektet. Dette muliggør bløde skygger, da hver lyskilde kan have en radius, i stedet for blot et bestemt punkt. Ud fra lysintensiteten og lyskildens farve beregnes $I$ (Linje 4). Herefter beregnes $\vv{I}$ og $\vv{R}$, som er hhv.\ vektor mod lyset og refleksionsvektoren. I linje 11-16, lægges de pågældende \textit{diffuse} og \textit{specular} lys til de samlede summer. Til sidst i raytracer\_phong returneres summen af \textit{ambient}, \textit{diffuse} og \textit{specular} lys, som er den farve det pågældende skæringspunkt på objektet får, når billedet renderes. 
\begin{lstlisting}[style=Cstyle, caption={summen af ambient, diffuse og specular lys returneres fra raytracer\_phong.}]
/* return ambient + diffuse + specular */
return pixel_add(ambient, pixel_add(diffuse, specular));
\end{lstlisting}

\paragraph{Placering af kamera}

Funktionen camera\_look\_at\_point er en funktion, der sørger for, at vi nemt og hurtigt kan bestemme en position og vinkel til kameraet hvis vi vil kigge på et bestemt objekt. 
For at kigge på et punkt skal vi vide hvor langt væk kameraet skal være fra punktet, hvilken vinkel vi vil se punktet fra og punktets position. Disse variabler er givet ved en vektor, der indeholder punktets position og tre doubles, der definerer de to vinkler og afstanden fra punktet til kameraet.
Vi sætter kameraets position til at være vores afstand ud af y aksen, og 0 ud af x og z aksen. Så roterer vi punktet ved hjælp af vores camera\_set\_angle funktion som drejer kameraet om sig selv for at vi kigger den rigtige vej.
Så drejer vi kameraet de angivne radianer om x og z aksen og til sidst lægger vi punktet vi vil se på, til kameraets position.

\begin{lstlisting}[style=Cstyle, caption=Kode-uddrag fra camera.c: camera\_look\_at\_point]
void camera_look_at_point(Camera *camera, Vector point, double distance, double vertical_angle, double horizontal_angle) {
  /* Resetting camera position to prepare rotation */
  camera->position = (Vector){0, -distance, 0};

  /* Rotate camera around itself */
  camera_set_angle(camera, -vertical_angle, horizontal_angle);
  
  /* Rotate camera around x and z axis and add to point we wanna look at */
  camera->position = vector_rotate_around_xz(camera->position, -vertical_angle, horizontal_angle);
  camera->position = vector_add(camera->position, point);
}
\end{lstlisting}