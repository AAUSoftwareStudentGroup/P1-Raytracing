\subsubsection{Objekt}
For at kunne visualisere et objekt er vi nødt til at definere hvad et objekt består af. Dette gøres vha.\ tre structs. Første struct beskriver en vertex, som er et endepunkt mellem to vektorer, en vertex kan derfor betegnes som et punkt der udgør et hjørne dannet af vektorer. Anden struct beskriver en trekant som består af tre verticies, dvs.\ tre hjørner. Til sidst har vi en struct for objektet. Et objekt består af n antal verticies, n antal trekanter, en farve og et materiale. 

\begin{lstlisting}[style=Cstyle, caption=Objekter]
#define VERTICES_IN_TRIANGLE 3

typedef struct _vertex {
  Vector position;
  Vector normal;
} Vertex;

typedef struct _triangle {
  Vertex *verticies[VERTICES_IN_TRIANGLE];
  Vector edges[VERTICES_IN_TRIANGLE];
} Triangle;

typedef struct _object {
  Vertex *verticies;
  int n_verticies;
  Triangle *triangles;
  int n_triangles;
  Pixel color;
  Material material;
  KDNode root;
} Object;


Object *new_object(void);
int free_object(Object *object);
\end{lstlisting}

Funktionen new\_object allokerer plads til objektet og returner den frigivne plads. Funktion free\_object frigiver den allokerede plads. 
\begin{lstlisting}[style=Cstyle, caption=Funktionerne]
Object *new_object(void) {
  Object *object = (Object*)malloc(sizeof(Object));
  return object;
}

int free_object(Object *object) {
  if(object->root.low != NULL)
    free_kdnode(object->root.low);
  if(object->root.high != NULL)
    free_kdnode(object->root.high);

  free(object);
  return 1;
}
\end{lstlisting}



