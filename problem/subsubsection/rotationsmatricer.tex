\subsubsection{Rotationsmatricer}
\label{sec:rot_matricer}
Hvis vi vil rotere et punkt eller en vektor omkring en akse i et koordinatsystem kan vi bruge en rotationsmatrix \cite{rotationsmatricer}.
\begin{align} 
  R_x(\theta) = 
  \begin{bmatrix}
  \label{eq:rotate_around_x}
    1 & 0 & 0\\ 
    0 & cos \theta & - sin \theta\\ 
    0 & sin \theta & cos \theta
  \end{bmatrix}\\
    R_y(\theta) =
  \begin{bmatrix}
    cos \theta  & 0 & sin \theta\\ 
    0           & 1 & 0\\ 
    -sin \theta & 0 & cos \theta
  \end{bmatrix}\\
    R_z(\theta) = 
  \begin{bmatrix}
    cos \theta & - sin \theta & 0\\ 
    sin \theta & cos \theta & 0\\
    0 & 0 & 1
  \end{bmatrix}
\end{align}
Vi indsætter den vinkel som vi vil dreje vektoren med i radianer og multiplicerer dem sammen som angivet i udtryk \ref{eq:rotate_around_x}. Vektoren bliver drejet omkring nul-punktet med netop den mængde radianer, som er angivet.
Nedenstående eksempel illustrerer princippet ved at dreje en vektor i rummet.

\begin{equation}
  {\vv{u}} =
  \begin{bmatrix}
    u_x \\ 
    u_y \\
    u_z
  \end{bmatrix}
\end{equation}
Den roterede vektor $\vv{u}$ kan nu beskrives som set i udtryk \ref{eq:rotation_x}
\begin{equation}
  \vv{v} = R_x(\theta) \cdot \vv{u} = \begin{bmatrix}
    u_x \\ 
    cos(\theta)   \cdot u_y - sin(\theta) \cdot u_z \\
    sin(\theta) \cdot u_y + cos(\theta) \cdot u_z
  \end{bmatrix}
  \label{eq:rotation_x}
\end{equation}
Vektor $\vv{u}$ og $\vv{v}$ er illustreret i nedenstående figur \ref{fig:rotationsmatrix_eksempel}.
\begin{figure}[H]
  \center
  \begin{tikzpicture}
    \coordinate (O) at (0,0) ;
    \coordinate (u) at (2, 1) ;
    \coordinate (v) at (1, 2) ;

    \draw[thick,->] (O) -- (3,0);
    \draw[thick,->] (O) -- (0,3);
    \draw[thick,->] (O) -- (3,0) node[anchor=north west] {y};
    \draw[thick,->] (O) -- (0,3) node[anchor=south east] {z};

    \draw [blue!50, thick, -{Stealth[width=3mm, length=3mm]}] (O) -- (u);
    \draw [blue!50, thick, -{Stealth[width=3mm, length=3mm]}] (O) -- (v);
    \node [below right] at (u) {$u$};
    \node [below right] at (v) {$v$};
    \draw[<-] (0.5, 1) arc (60:20:1);
    \node[] at (1,1)  {$\theta$};
  \end{tikzpicture}
  \caption{Eksempel på rotation af en vektor om x-aksen}
  \label{fig:rotationsmatrix_eksempel}
\end{figure}