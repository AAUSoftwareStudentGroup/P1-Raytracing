\subsubsection{Konvertering fra farvetemperatur til RGB}
\label{sec:temptilrgb}
Farvetemperatur, som også er beskrevet nederst i \ref{sec:lys}, er temperaturen af et udsendt lys og måles i kelvin. Denne temperatur kan bruges til at finde ud af, om et lys er varmt eller koldt. 
RGB-værdien er en værdi for en given farves indhold af rød, grøn og blå. Værdien angives normalt ved et tal mellem 0 og 255, altså er RBG-værdien [0, 128, 255] ingen del rød, en del grøn og fuld blå, hvilket, blandet sammen, giver en blålig farve.
Der findes ingen direkte og 100\% præcis formel for at ’oversætte’ en kelvintemperaturværdi til en RGB-værdi, derfor har rapporten taget udgangspunkt i en algoritme, som er lavet ud fra 400 målinger, men som stadig ikke er præcis nok til videnskabelig brug \cite{tanner_helland}.
Måden hvorpå algoritmen er lavet, er ved at tage disse 400 målinger, og lave en funktion ud fra dem. Der er lavet én måling per 100 kelvin, der starter ved 1000 kelvin og slutter ved 40.000 kelvin \cite{charity_values}. Ved at kigge på funktionen \cite{tanner_helland_chart} har Tanner Helland kunne konkludere  tre ting:

\begin{itemize}
\item Røde værdier under 6600 kelvin er altid 255.
\item Blå værdier under 2000 kelvin er altid 0.
\item Blå værdier over 6500 kelvin er altid 255.
\end{itemize}

Disse tre, forholdsvis simple, konklusioner har hjulpet med at gøre algoritmen meget kortere og mere simpel. Herunder kan udregningerne for hhv.\ rød-, grøn- og blå-værdierne ses matematisk, før de er skrevet om til kode. Matematikken er vist gennem gaffelfunktioner, altså funktioner med forskellige funktionsudtryk for bestemte intervaller.


\begin{displaymath}
   R(k) = \left\{
     \begin{array}{lr}
       255 &1000 <= \text{k $\land$ k} <= 6600\\
       329.698727446*(k-60^{-0.1332047592}) &6600 < \text{k $\land$ k} <= 40000
     \end{array}
   \right.
\end{displaymath} 

\begin{displaymath}
   G(k) = \left\{
     \begin{array}{lr}
       99.4708025861*\ln(k)-161.1195681661 &1000 <= \text{k $\land$ k} <= 6600\\
       288.1221695283*(k-60^{-0.0755148492}) &6600 < \text{k $\land$ k} <= 40000
     \end{array}
   \right.
\end{displaymath} 

\begin{displaymath}
   B(k) = \left\{
     \begin{array}{lr}
       255 &6600 <= \text{k $\land$ k} <= 40000\\
       0 &1000 <= \text{k $\land$ k} <= 1900\\
       138.5177312231 * \ln(k-10) - 305.0447927307 &1900 < \text{k $\land$ k} < 6600
     \end{array}
   \right.
\end{displaymath} 

Grunden til at der oversættes fra farvetemperatur til RGB-værdi, er for at kunne visualisere farverne på en computer. Et billede vist på en computer består af pixels, som alle har en RGB-værdi, derfor kan lyset fra en lampe med en given farvetemperatur visualiseres, hvis farvetemperaturen oversættes til RGB-værdi.

\paragraph*{Opsummering}
Vi har nu set hvordan man kan konvertere en farvetemperatur i Kelvin til RGB-værdier. Derudover har vi set, hvordan man kan rotere en vektor ved hjælp af matricer. Denne viden er nødvendig for at kunne opfylde kravene om at løsningen skal gøre det muligt at vise lampen med forskellige farvetemperaturer og vinkler. 









