\subsubsection{Producenter}
Producenten samarbejder med designeren om at udvikle lampen. Producentens rolle er at fremstille lampen på baggrund af designet. Når lamperne er produceret sendes de ud til lampebutikkerne. Producenten kan have en interesse i, at kunderne køber deres produkter i lampebutikkerne, da der er mulighed for, at dette vil medføre at lampebutikkerne bestiller flere af deres produkter hjem. Vi har haft kontakt med en belysningskonsulent, som her ønsker at være anonym. Belysningskonsulenten sagde følgende om producenternes interesse i problemet:

\begin{center}
\textit{"Det er blevet en kompliceret proces at producere en lampe ift. EU lovgivning i dag så jeg har svært ved at se at producenterne vil koste endnu flere penge til produkter til privatmarkedet som måske kun køber en lampe til 3000 kr. som ofte kun interesserer sig for den laveste pris og ikke den bedste service og rådgivning. Så producenters incitament til ligge investeringer hos privatkunder er meget begrænset."} Mailen kan ses i bilag \ref{sec:mailbelysning}. 
\end{center}

Belysningskonsulenten mener altså at producenterne ikke vil smide en masse penge ind i privatmarkedet, da det ikke gavner dem økonomisk. 

Producenterne får sandsynligvis flere midler til større produktion af en given lampe, hvis lampen er populær. desto flere solgte lamper til en lampebutik, desto flere penge tjener producenten. Det er derfor et problem for producenterne, hvis en lampe returneres, grundet en utilfreds kunde. Dog er producenterne, som nævnt i citatet fra belysningskonsulenten, ikke nødvendigvis interesseret i at investere penge i rådgivning, og dermed også visualisering af en lampe og dens belysning for kunder. I stedet er dette en opgave, som i større grad ligger hos lampebutikken, som beskrevet i næste afsnit.
