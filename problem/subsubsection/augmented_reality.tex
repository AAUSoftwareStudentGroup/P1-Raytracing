\subsubsection{Augmented Reality}
Der er nu blevet gennemgået visualisering af virkelige objekter ved hjælp af digitale billeder og virtuelle objekter ved hjælp af computergrafik. I dette afsnit beskrives augmented reality, som er en teknologi, der kombinerer virkelige og virtuelle objekter \cite{augmented_reality}.

Augmented reality fungerer ved at tage et billede med et normalt kamera, og herefter ændrer billedet ved at indsætte computergrafik på billedet \cite{augmented_reality}.

Et eksempel på anvendelsen af augmented reality er Artemides Augmented Reality App. Denne app gør det muligt, at visualisere udvalgte lamper i en kontekst som brugeren selv vælger \cite{artemides}. 

Fordelen ved augmented reality, i forhold til løsningsforslaget, er, at den muliggør at se lampen fra flere forskellige vinkler i den kontekst som kunden ønsker. Hvis der ses bort fra tekniske udfordringer ville det også være en fordel hvis kunden kunne visualisere en lampe og dens belysning i den kontekst som kunden ønsker at købe lampen til. \newline Ulempen ved augmented reality i forhold til løsningsforslaget er, at det er en teknisk udfordring, at få en rummelig forståelse for den virkelige kontekst så det virtuelle, der indsættes får en tilstrækkelig realisme. Det vil derfor være svært at simulere lampens belysning hvis man ikke kender til dimensioner eller materialerne i den virkelige kontekst. Hvis det lykkedes, at få en forståelse for de virkelige objekter i billedet, vil det stadig være raytracing eller en anden teknik inden for computergrafik som ville være at foretrække når man skal visualisere lampens belysning. Derfor er udfordringen ved augmented reality dobbelt, da den både skal få en rummelig forståelse for de virkelige objekter i billedet, samt lave en realistisk computergrafik som passer ind i billedet.

