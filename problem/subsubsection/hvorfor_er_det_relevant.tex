\subsubsection{Argumentation for problemets relevans}
\label{sec:hvorfor_relavant}
For at svare på, hvorfor problemet er relevant antages nu, at det initierende problem eksisterer. Det er svært at bevise, at denne antagelse er korrekt, men antagelsen er lavet efter en diskussion med Lars Peter Jensen, lektor på AAU, som netop havde erfaret, at han havde svært ved at visualisere hvordan lys fra en bestemt lampe ville se ud i sit hus, før han havde installeret lampen. Antagelsen understøttes af en udtalelse fra en belysningskonsulent for en dansk lampebutik som siger følgende: "Der er overraskende mange, der gerne vil se lyset inden de køber lamper"(se bilag \ref{sec:mailbelysning}). Dette er en erfaring som flere i gruppen også har gjort sig. Ud fra en diskussion, har gruppen valgt at arbejde videre med antagelsen, da det formodes at andre mennesker har haft lignende problem. Denne formodning ønskede gruppen at teste yderligere ved en spørgeskemaundersøgelse i IKEA, hvor tanken var at spørge kunderne om de kunne visualisere hvordan lys udbredte sig fra de lamper som de så i butikken. Gruppen henvendte sig derfor til IKEA i Aalborg for at høre om det var muligt at møde op i deres butik for at uddele spørgeskemaer, men gruppens henvendelse blev afvist. Spørgeskemaet er vedhæftet i bilag \ref{sec:skema}.

På baggrund af antagelsen formoder vi, at mennesker har svært ved at visualisere hvordan lys udbreder sig fra en lampe. Hvis kunden mangler visualisering af lampen og dens belysning, kan der opstå fejlkøb af lamper. Hvis kunden oplever fejlkøb af en lampe, kan det medføre økonomiske konsekvenser, hvis lampen ikke kan returneres. Hvis fejlkøb resulterer i købet af en lampe med dårlig belysning, kan det medføre konsekvenserne nævnt i afsnit \ref{sec:konsekvenser}.