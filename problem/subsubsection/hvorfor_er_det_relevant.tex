\subsubsection{Hvorfor er det relevant?}
\label{sec:hvorfor_relavant}
For at svare på, hvorfor problemet er relevant, opstilles følgende antagelse: "Mennesker har svært ved at visualisere hvordan lys udbreder sig fra en lampe." Det er svært at bevise, at denne antagelse er korrekt for alle mennesker, men antagelsen er lavet efter en diskussion med Lars Peter Jensen, lektor på AAU, som netop havde erfaret, at han havde svært ved at visualisere hvordan lys fra en bestemt lampe ville se ud i hans hus, før han havde installeret lampen. Antagelsen understøttes af en udtalelse fra en belysningskonsulent for en dansk lampebutik som siger følgende: "Der er overraskende mange der gerne vil se lyset inden de køber lamper"(se bilag \ref{sec:mailbelysning}). Dette er en erfaring som flere i gruppen har gjort sig. Ud fra diskussionen og egne erfaringer har gruppen valgt at arbejde videre med antagelsen, da det må formodes at andre mennesker har haft lignende problem. Denne formodning ønskede gruppen at teste ved en spørgeskemaundersøgelse i IKEA, hvor tanken var at spørge de handlende om de kunne visualisere hvordan lys udbredte sig fra de lamper som de så i butikken. Gruppen henvendte sig derfor til IKEA i Aalborg for at høre om det var muligt at møde op i deres butik for at uddele spørgeskemaer, men gruppens henvendelse blev afvist.

På baggrund af antagelsen må vi formode, at fordi mennesker har svært ved at visualisere hvordan lys udbreder sig for en lampe, sker det at der købes lamper, som ikke lever op til de forventninger som kunden havde da lampen blev købt. Lamper med dårlig belysning er lamper, der ikke lever op til de belysningsmæssige krav som brugeren stiller til at dække sine behov. Som nævnt i \ref{sec:konsekvenser} er konsekvenserne ved dårlig belysning blandt andet træthed og mindsket produktivitet.
Ud fra antagelsen kan det uddrages at problemet er relevant, da mangel på visualisering af lys kan føre til fejlkøb , og hvis belysningen fra lampen er dårligt så kan dette bl.a.\ medføre hovedpine, træthed og formindske produktiviteten af ens arbejde.
