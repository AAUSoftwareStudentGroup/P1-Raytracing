\subsubsection{Fra 3D-model til billede}
\label{sec:fra_model_til_billede}
I dette afsnit er formålet nu at vise en model for, hvordan billeddannelsen af objekter i rummet, også kaldt rendering, kan konstrueres. Dette er essentielt, da billeddannelsen danner grundlag for, hvordan 3D-modellen for en lampe omdannes til et billede, der kan vises for kunderne på e-butikken. Til sidst i afsnittet udledes en model for, hvordan belysningen fra en lampe kan simuleres og visualiseres vha.\ raytracing. 

\paragraph{3D-model}
En 3D-model er en matematisk beskrivelse af et tre dimensionelt objekt. For at beskrive et 3D-objekt opdeler man ofte objektet i trekanter. Dette er illustreret på nedenstående figur.

\begin{figure}[H]
\label{fig:kanin}
    \centering
    \includegraphics[width=5cm]{kanin}
    \caption{Eksempel hvor trekanter bruges til at repræsentere et objekt. Kilde: http://www.cs.rpi.edu}
\end{figure}

For at rendere et billede af en 3D-model bestående af trekanter, er man nødt til at have en model for det kamera, som billedet renderes ud fra. Hvordan kameraet kan modelleres er beskrevet i næste afsnit.

\paragraph{Kamera}
Et kamera, i konteksten af at rendere en 3D-model, er typisk en beskrivelse af position, orientation og synsfelt. Andre informationer kan tilknyttes som beskriver opløsning af det renderede billede, samt perspektivet af det resulterende billede. Orientationen fastsættes ved hjælp af et sæt vektorer som beskriver hvilken retning der er f.eks.\ højre og op for kameraret.

\begin{figure}[H]
  \centering
  \tdplotsetmaincoords{60}{130}
\begin{tikzpicture}[tdplot_main_coords]
\path[fill=gray!20, draw=gray!40] (-3,-4,-3) -- (-3,-4,3) -- (3,-4,3) -- (3,-4,-3) -- (-3,-4,-3);
\draw (0,0,0) -- (0,-4,0);
\draw[blue!50, thick, -{Stealth[width=2mm, length=2mm]}] (0,0,0) -- (0,-1,0);
\draw[blue!50, thick, -{Stealth[width=2mm, length=3mm]}] (0,-4,0) -- (-1,-4,0);
\draw[blue!50, thick, -{Stealth[width=2mm, length=2mm]}] (0,-4,0) -- (0,-4,1);
\draw[blue!50, thick, -{Stealth[width=2mm, length=2mm]}] (0,0,0) -- (-0.41,-0.8,0.41);
\draw[black, thick, dashed] (-0.41,-0.8,0.41) -- (-2,-4,2);
\draw[black, dashed, thick] (-0,-4,1) -- (-0,-4,2);
\draw[black, dashed, thick] (-0,-4,2) -- (-2,-4,2);
\draw[black, dashed, thick] (-1,-4,0) -- (-2,-4,0);
\draw[black, dashed, thick] (-2,-4,0) -- (-2,-4,2);
\draw plot [mark=*, mark size=2] coordinates{(0,0,0) }; 
\draw plot [mark=*, mark size=2] coordinates{(-2,-4,2) }; 
\node [below] at (0,0,0) {$C$};
\node [below] at (0,-1,-0.1) {$\vv{f}$};
\node [above right] at (0,-2.5,0) {$d$};
\node [above right] at (-2,-4,2) {$B$};
\node [left] at (-2,-4,1) {$y$};
\node [above] at (-1,-4,2) {$x$};
\node [above] at (-1,-4,0) {$\vv{r}$};
\node [right] at (-0.41,-0.8,0.41) {$\vv{s}$};
\node [left] at (0,-4,1) {$\vv{u}$};
\draw[blue!50, thick, |-|] (3.1,-4,3.2) -- (-2.9,-4,3.2);
\draw[blue!50, thick, |-|] (3.2,-4,3.1) -- (3.2,-4,-2.9);
\node [above] at (0,-4,3.2) {$w$};
\node [left]  at (3.2,-4,0) {$h$};
\end{tikzpicture}
  \caption{Visualisering af et punkt på billedplanet som en kombination af tre vektorer $\protect\vv{r}$, $\protect\vv{u}$ og $\protect\vv{f}$, der repræsenterer hhv. højre, op og frem retninger for kameraet.}
  \label{fig:kamera_billede}
\end{figure}

Et punkt $B$ på billedplanet (Se figur \ref{fig:kamera_billede}) kan repræsentere en pixel og kan beskrives som en lineær kombination af en op- og en højre-orientationsvektor, samt en vektor, der beskriver afstanden til billedplanet relativt til kameraet. På figuren er denne vektor $\vv{f}$ skaleret med en faktor $d$.

En stråles retning gennem et pixel koordinat $B$ relativt til kameraets udgangspunkt $C$ kan altså beskrives på følgende måde:
\begin{align}
\label{eq:pixel_til_billede}
  \vv{B} &= C + \vv{f} \cdot d + \vv{u} \cdot \left(y-\frac{h}{2}\right) + \vv{r} \cdot \left(x-\frac{w}{2}\right) \\
  \vv{s} &= \frac{\vv{B}-\vv{C}}{||\vv{B}-\vv{C}||}
\end{align}

Hvor $(x,y)$ er pixelkoordinatet på billedet mellem $(0,0)$ og $(w,h)$.

Afstanden $d$ samt størrelsen af billedplanet bestemmer synsvinklen.

\paragraph{Perspektiv projektion}
\label{sec:perspektiv_projektion}
For at udlede en model for billeddannelsen, tages der udgangspunkt i en perspektiv projektion. Perspektiv projektion er en måde at danne et billede af 3D-objekter ved at projektere objekterne hen på et plan mod et kameras position \cite{perspective_projection}. Princippet bag perspektiv projektion er vist på figur \ref{fig:perspektiv_projektion}.

\begin{figure}[H]
  \centering
  \tdplotsetmaincoords{60}{130}
\begin{tikzpicture}[tdplot_main_coords]
\path[fill=blue!50, draw=gray!20] (2,-8,2) -- (-2,-8,2) -- (0,-8,0) -- (2,-8,2);
\draw (1,-4,1) -- (2,-8,2);
\draw (-1,-4,1) -- (-2,-8,2);
\draw (0,-4,0) -- (0,-8,0);
\path[fill=gray!20, draw=gray!40] (-2,-4,-2) -- (-2,-4,2) -- (2,-4,2) -- (2,-4,-2) -- (-2,-4,-2);
\path[fill=blue!50, draw=gray!20] (1,-4,1) -- (-1,-4,1) -- (0,-4,0) -- (1,-4,1);
\draw (0,0,0) -- (1,-4,1);
\draw (0,0,0) -- (-1,-4,1);
\draw (0,0,0) -- (0,-4,0);

\draw plot [mark=*, mark size=2] coordinates{(2,-8,2) } ; 
\draw plot [mark=*, mark size=2] coordinates{(-2,-8,2) } ; 
\draw plot [mark=*, mark size=2] coordinates{(0,-8,0) } ; 
\draw plot [mark=*, mark size=2] coordinates{(1,-4,1) }; 
\draw plot [mark=*, mark size=2] coordinates{(-1,-4,1) }; 
\draw plot [mark=*, mark size=2] coordinates{(0,-4,0) }; 
\draw plot [mark=*, mark size=2] coordinates{(0,0,0) }; 
\node [above left] at (2,-8,2) {$P$};
\node [above left] at (1,-4,1) {$B$};
\node [above right] at (0,0,0) {$C$};
\end{tikzpicture}
  \caption{Princippet bag perspektiv projektion af et punkt på et billedplan.}
  \label{fig:perspektiv_projektion}
\end{figure}

Som vist på figur \ref{fig:perspektiv_projektion} kan et punkt $P\in \mathbb{R}^3$ projekteres ned på billedplanen $\alpha$, ved at finde skæringspunktet $B$ mellem billedplanen $\alpha$ og en lysstråle $L$, som går fra punktet $P$ mod kameraets position $C$. Gør man nu dette for alle punkter på et objekt i rummet, og tegner skæringspunkterne på billedplanen, dannes et billede af objektet, ved at oversætte punktet på billedplanen, til (x,y)-pixelkoordinater, på det resulterende billede, med udtryk \ref{eq:pixel_til_billede} beskrevet under afsnittet om kameraet.

Udfordringen er så, at afgøre hvilken farve punkterne på billedplanen skal have, da dette afhænger af objektets egenskaber, samt hvilket udefrakommende lys der rammer objektet. 

For at løse denne udfordring, benytter vi i dette projekt raytracing, der, som beskrevet under afsnit \ref{sec:computergrafik}, bygger på at simulere lysstrålers interaktion med forskellige objekter i rummet. Hvordan dette fungerer er beskrevet i næste afsnit, hvor der er beskrevet en model for backwards raytracing.

\paragraph{Backwards raytracing}
I modsætning til en perspektiv projektion af et punkt på et plan, står backwards raytracing, hvor man i stedet for punktet i rummet, tager udgangspunkt i de lysstråler (rays), der danner billedet. Ved backwards raytracing følger man lysstrålerne baglæns og ser på, hvor stor en lysintensitet den pågældende lysstråle har efter den har interageret med objekterne i rummet. Ud fra dette farves det tilhørende punkt på billedet, og på den måde kan man rendere et helt billede. På figur \ref{fig:raytracing_skitse} er det vist hvordan man kan konstruere en lysstråle ud fra et bestemt punkt på billedplanen, hvor lysstrålen er beskrevet ved en retningsvektor og et startpunkt.

\begin{figure}[H]
  \centering
  \tdplotsetmaincoords{60}{130}
  \begin{tikzpicture}[tdplot_main_coords]
  \path[fill=blue!50, draw=gray!20] (2,-8,2) -- (-2,-8,2) -- (0,-8,0) -- (2,-8,2);
\draw (0,0,0) -- (0,-8,1);
\path[fill=gray!20, draw=gray!40] (-2,-4,-2) -- (-2,-4,2) -- (2,-4,2) -- (2,-4,-2) -- (-2,-4,-2);
\draw plot [mark=*, mark size=2] coordinates{(0,-4,0.5) }; 
\draw plot [mark=*, mark size=2] coordinates{(0,0,0) }; 
\node [above right] at (0,-4,0.5) {$B$};
\node [above right] at (0,0,0) {$C$};
\draw [blue!50, thick, -{Stealth[width=3mm, length=3mm]}] (0,0,0) -- (0,-4,0.5);
\node [above right] at (0,-2,0.25) {$\vv{r}$};
\draw plot [mark=*, mark size=2] coordinates{(0,-8,1) }; 
\end{tikzpicture}
  \caption{Viser hvordan der kan opstilles retningsvektor mellem kameraets position $C$ og punktet $P$ på billedplanen, som sammen med startpunktet $C$ beskriver lysstrålen fra trekanten i omvendt retning.}
  \label{fig:raytracing_skitse}
\end{figure}

Retningsvektoren $\vv{r}$ for lysstrålen kan heraf beskrives som følgende.

$$ \vv{r} = \vv{B} - \vv{C} $$

Hvor $\vv{B}$ og $\vv{C}$ er stedvektorer for hhv. punktet på billedplanen $B$ og kameraets position $C$.

Lysstrålen kan på den måde beskrives ved følgende vektorfunktion.

$$ \vv{l}(t) = \vv{r} t + \vv{C}$$

Hvor $t$ er en skalar i $\mathbb{R}$.

For at finde ud af hvilken farve punktet på billedplanen $B$ skal have, ser man hvordan lysstrålen rammer de forskellige objekter der skal renderes.

Der findes flere forskellige modeller for hvordan lysintensiteten for en lysstråle beregnes. En simpel model, er Phong-modellen, som opdeler lys i forskellige kategorier: ambient, diffuse og specular.

\paragraph{Phong-modellen}
Phong-modellen er en såkaldt \textit{shading} funktion, som beskriver lyset fra punkter på et objekt på baggrund af lyskilden, objektet og kameraets synsvinkel \cite{phong_paper}. Der findes flere forskellige variationer af phong-modellen. Da vi som vist i afsnit \ref{sec:temptilrgb} kan arbejde med farvetemperaturer via RGB-værdier, så har vi valgt en variation af phong-modellen, som beskriver lys via RGB-værdier. 

Ud fra modellen \cite{stanford_phong}, kan der skrives følgende overordnede \textit{shading} funktion.
\begin{equation} \label{eq:phong}
  \rho = \rho_a + \sum\limits_{lights} (\rho_d + \rho_s)
\end{equation}
Hvor $\rho_a$, $\rho_l$ og $\rho_s$ er hhv. \textit{ambient}, \textit{diffuse} og \textit{specular} lys der er beskrevet kort herunder.

\subparagraph{\textit{Ambient} lys} repræsenterer det lys, som reflekteres rundt i rummet og rammer objekter ligeligt fra alle sider \cite{stanford_phong}. Formlen til at beregne dette er følgende \cite{stanford_phong}.
\begin{align}
\label{eq:ambient_formel}
	\rho_a &= m_a C A
\end{align}
Hvor $m_a$ er den ambiente konstant, $C$ er overfladens farve som rgb-værdi og $A$ er den ambiente lysintensitet.

\subparagraph{\textit{Diffuse} lys} er det lys, som reflekteres ifølge \textit{Lambert's Cosinuslov}. Ud fra loven kan følgende model anvendes til at udregne \textit{diffuse} lys \cite{stanford_phong}.
\begin{align}
	\rho_l &= m_l C I max(\vv{I}\bullet\vv{n}, 0)
\end{align}
Hvor $m_l$ er den Lambertianske konstant, $I$ er det lyskildens lysintensitet, $\vv{I}$ og $\vv{n}$, er normaliserede vektorer, hhv.\ vektor fra overfladepunktet mod lyskilden og normalvektoren til overfladepunktet.

\subparagraph{\textit{Specular} lys} er det lys der spejles i objektets overflade, givet ved nedenstående formel \cite{stanford_phong}.
\begin{align}
	\rho_s &= m_s S I max(\vv{r}\bullet\vv{u},0)^{m_{sp}} \\
	S &= m_{sm} C + (1 - m_{sm})(1,1,1)
\end{align}
Hvor $m_s$ er den speculare konstant, $m_sp$ er den speculare eksponent som beskriver størrelsen af lys der spejles i objektets overflade, $S$ er en lineær interpolation mellem objektets farve og hvid, afhængig af objektets \textit{metalness} $m_{sm}$. $\vv{u}$ og $\vv{r}$ er normaliserede vektorer, for hhv. vektor fra overfladepunktet mod kameraet og retningsvektor for det reflekterede lys beregnet som følgende.
\begin{align}
	\vv{r} &= -\vv{I} + 2 (\vv{I} \bullet \vv{n}) \vv{n}
\end{align}

De anvendte vektorer til phong \textit{shading}-funktionen er vist på figuren herunder.
\begin{figure}[H]
  \label{fig:phong_skitse}
  \centering
  \tdplotsetmaincoords{60}{130}
  \begin{tikzpicture}[tdplot_main_coords]
  \draw plot [mark=*, mark size=2] coordinates{(0,-4,2) }; 
\node [above left] at (0,-4,2) {$C$};
  \draw (0,-4,2) -- (0,-3,1.5);
  \path[fill=gray!20, draw=gray!40] (-1,-3,1) -- (-1,-3,3) -- (1,-3,3) -- (1,-3,1);

  \path[fill=gray!20, draw=gray!40] (1,-2,0) -- (-1,0,0) -- (1,2,0);
    \draw (0,-3,1.5) -- (0,0,0);
    \draw [blue!50, thick, -{Stealth[width=3mm, length=3mm]}] (0,0,0) -- (0,0,2.24);
    \node [above] at (0,0,2.24) {$\vv{n}$};
    \draw [blue!50, thick, -{Stealth[width=3mm, length=3mm]}] (0,0,0) -- (0,-2,1);
    \node [below left] at (0,-2,1) {$\vv{u}$};
    \draw [blue!50, thick, -{Stealth[width=3mm, length=3mm]}] (0,0,0) -- (0,-1,2);
    \node [above left] at (0,-1,2) {$\vv{r}$};
    \draw (0,0,0) -- (0,2,4);
    \draw plot [mark=*, mark size=2] coordinates{(0,2,4) }; 
    \node [above right] at (0,2,4) {$L$};
    \draw [blue!50, thick, -{Stealth[width=3mm, length=3mm]}] (0,0,0) -- (0,1,2);
    \node [right] at (0,1,2) {$\vv{I}$};
 \draw (0,0.3, 1.1) -- (0,0.35, 1.3);
  \draw (0,0.35, 1.1) -- (0,0.4, 1.3);
  
 \draw (0,-0.2, 0.9) -- (0,-0.25, 1.1);
  \draw (0,-0.15, 0.95) -- (0,-0.2, 1.15);

  \draw (0,-1,2) coordinate (a)
  (0,0,0) coordinate (b)
  (0,0,2.24) coordinate (c)
  pic[draw=black,|-|,angle eccentricity=1.2,angle radius=1cm] {angle=c--b--a};
  \draw (0,0,2.24) coordinate (a)
  (0,0,0) coordinate (b)
  (0,1,2) coordinate (c)
  pic[draw=black,|-|,angle eccentricity=1.2,angle radius=1cm] {angle=c--b--a};
\draw plot [mark=*, mark size=2] coordinates{(0,0,0) }; 
    \node [below right] at (0,0,0) {$P$};
    \draw plot [mark=*, mark size=2] coordinates{(0,-3,1.5) }; 
\end{tikzpicture}
  \caption{Viser vektorer der anvendes i phong-modellen. $\protect\vv{u}$ er vektor mod kameraet med position $C$, $\protect\vv{n}$ er normalvektoren til objektet i punktet $P$, $\protect\vv{I}$ er vektor mod lyskildens position $L$ og $\protect\vv{r}$ er vektor for det reflekterede lys.}
\end{figure}