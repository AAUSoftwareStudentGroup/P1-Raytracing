\subsubsection{Overordnet programstruktur}
Main funktionen er den funktion som fortæller computerens operativ system hvilke kommandoer den skal udføre. Det er således i funktionen main, at programmets funktioner kaldes og eksekveres. Programmets main funktion er vist nedenfor:

\begin{lstlisting}[style=Cstyle, caption=Main]
int main(int argc, char* argv[]) {
  Scene *scene;
  Camera *camera;
  Image *image;

  unsigned long t0 = time(NULL);
  
  if(input_parse(argc, argv, &scene, &camera) == 0) {
    return -1;
  }
  
  image = raytracer_render(scene, camera);
  
  if(argc == 3)
    image_write(image, argv[2]);
  else
    image_write(image, "out.ppm");
  
  printf("%lus\n", time(NULL) - t0);
  return 0;
}
\end{lstlisting}

På linje 1 i ovenstående kodeuddrag, kaldes main, med parametrene argc og argv. Argc bruges til at holde styr på, hvor mange informationer programmet får. I løkken på linje 8 tjekkes det, om inputtet er korrekt, og der angives nogle specifikke informationer til programmet, alt efter hvad inputtet er, som gøres gennem input\_parse-funktionen i input.c. Her kan man f.eks. angive højde, bredde og farvetemperatur. Argv bruges til at holde styr på, hvad programmet har fået som input. På Linje 12 kaldes funktionen raytracer\_render som renderer billedet. Linje 6 og 19 bruges udelukkende til teknisk anvendelse, for at kunne se hvor lang tid programmet har taget om at fuldføre. Dette bruges til at checke renderingstiden af et billede.
